
\chapter{Conclusion and Future Work}
In this thesis, we aimed to show that utilising the local knowledge of an environment in a WSN improves the efficiency of the network by giving it the ability to prioritise sensed data based on the results of in-network processing.  We believe that pushing knowledge out farther towards the edges of the network improves the overall performance. To show this, we have developed a three-tier WSN architecture that uses knowledge-processing capabilities to process sensed data as it is forwarded through the network. Most current WSN implementations deliver data chronologically, or store it on the node to be retrieved by queries. We believe that this knowledge can be used to infer how valuable sensed data is and prioritise that through the network, delivering the most interesting data first. However, resources are still limited in WSNs and our architecture had to utilise these resources effectively, such as battery life and bandwidth, to maximise the network lifetime. Using Data Collection (DC) nodes at the edge of the network, they capture observations and use their limited knowledge-processing capabilities to enrich the sensed data before sending it on. Data Processing (DP) nodes use more powerful knowledge-processing features to attempt to classify observations and prioritise the sending of them to Data Aggregation (DA) nodes. DA node make the data available to users and use their classifications, and input, to dynamically update its knowledge base.

We used a collaboration with a research centre in Malaysia with the aim to implement K-HAS in the Malaysian rainforest, in an area that had experienced logging and contained a diverse range of rare wildlife. Using K-HAS, we wanted to deploy a network that would prioritise images of rare wildlife and only send images of common wildlife when bandwidth was available.  Over the course of 3 visits, we gathered knowledge from the area, researchers and locals to build a knowledge base and create rules that could be used to classify data. We also collected images taken from their current, manual solution to infer patterns and use existing classifications for new sensed data.

Current sensing technology means that K-HAS is not ready to be implemented as the architecture dictates. DC nodes are not yet readily available, and this is especially true for image capturing sensor nodes. Because of this, we modified the architecture to use commercial hardware with fewer capabilities that would allow us to actually deploy a sensor network in the Malaysian rainforest that does use local knowledge. We call this architecture LORIS, Local-knowledge Ontology-based Remote-Sensing Informatics System, and this solution combines the DA and DP node to hold all of the knowledge-processing capabilities.

LORIS has shown that even using local knowledge at the base station of a WSN means that sensed data is processed and organised within minutes of being received. This method also means that users are alerted to data that they have subscribed automatically.

Our simulations of K-HAS show that the bandwidth in a WSN is used more effectively when knowledge is pushed out towards the edge of the network, allowing nodes to perform knowledge-processing to make inferences about the contents of the data and prioritise, or delay, its delivery appropriately. We have no reason to think that this would not work in practice.


\section{Summary of Contributions}
In this section, we summarise the contributions detailed in this thesis, focussing on our deployment in Malaysia, the tiered architecture designed for general use and the alignment ontology created to formalise the architecture of K-HAS and the data standard that it uses. 
\subsection{K-HAS}
In Chapter 4, we present a novel tiered architecture, K-HAS, for WSNs that uses local knowledge. We explored existing networks, those related to our motivating scenario in Malaysia and also those that use knowledge or context-awareness, and routing protocols that use the sensed data to determine how it is routed. We explain the purpose of each tier in the network and show how sensed data is enriched and routed as it progresses through the network. Using Darwin Core as a data standard, each node communicates in a common format and metadata is packaged with data in archives that can be read by any node in the network. We also show how rules can be used to infer the content of sensed data and the ability to run rule engines in the network allows sensed data to be prioritised based on its value; not just the time it was captured.
\subsection{Ontology}
In Chapter 5 we explain the aligning ontology created to formally represent the K-HAS architecture and the data standard used. We show how, while there are existing ontologies that join observation-centric and sensor-centric ontologies, K-HAS allows for the representation of knowledge exchange in the network and show nodes performing tasks similar to humans.
We also show how the ontology is extensible and does not need to be specific to K-HAS; it can be used with any WSN that deals with scientific observations. 
\subsection{LORIS}
In Chapter 6, we present the Local-knowledge Ontology-based Remote-Sensing Informatics System (LORIS), a system developed for our motivating scenario when we discovered that current sensing technology means that K-HAS is not ready to be implemented in its current form. DC nodes are not a type of sensor that is readily available, and this is especially true for image capturing sensor nodes. Because of this, we modified the architecture to use commercial hardware with fewer capabilities that would allow us to deploy a sensor network in the Malaysian rainforest that does use local knowledge. This solution combines the DA and DP node to hold all of the knowledge-processing capabilities and then uses commercially available sensors to replace DC nodes. 

This architecture is easier to implement but does not provide in-network processing, although it automates the delivery of sensed data and uses the increased knowledge-processing capabilities of modern PCs to process sensed data on arrival and inform users. We show how our deployment of LORIS was successful and highlighted how sensed data was delivered within minutes of being captured in some cases, and processed shortly after. 
\subsection{Simulations}
In Chapter 7, we explain the implementation and results of our simulations to model an ideal deployment of K-HAS. We model every variable of the network on existing data collected from our motivating scenario and show that the delivery of observations can be reduced by more than four hundred hours, when compared to the current manual solution. We also outline how the network is able to prioritise data that it believes to be interesting, using a priority queue mechanism that delays data it believes to be empty. Comparing K-HAS to different network implementations, where nodes have different levels of knowledge processing capabilities, shows that our network implementation is more power efficient than a network where every node has higher knowledge processing capabilities, yet its performance is similar.

\section{Future Work}
The focus of this thesis is to show that local knowledge can improve the timeliness of interesting data by making inferences based on previously sensed data and knowledge of the environment. We have shown, using simulations, that this can be done with our tiered approach. However, another aim was to deploy such an architecture for our collaborators. We have explained that current technology means that this is not feasible, but sensing, and microcomputer technology has moved forward significantly since the beginning of this PhD and it would be possible with a longer time period. Our deployment in Malaysia was limited due to time constraints and we would haved liked to leave a functional deployment active in Danau Girang for a six month period. 

With more time, we could create custom DC nodes using webcams and micro-computers, like the Raspberry Pi, encased in a watertight enclosure; allowing us to use higher levels of knowledge-processing at the edge of the network. A deployment of this length would also allow nodes to act on sensed data classified by users and update their knowledge base, responding to changes in the network dynamically.

K-HAS uses a combination of open source projects and some software created during the course of this work, but it does require specialist knowledge to be deployed. We would like to create an installation candidate that would be usable by those without any expertise that could provide basic information on the purpose of the network and the installation of all necessary packages would be automated.

On top of this, our software's user interfaces have been tested by researchers at Danau Girang but we have no metrics on the usability of the software. Testing the software on users to determine how they would score the different areas of the software, such as usability, response time and learning curve, would help us to improve the software and ensure that it can be used by those without any technical knowledge of K-HAS or its underlying architecture.

Experiments have shown us that the range of wireless transmissions is often hard to predict and can be heavily influenced by weather and obstacles. If sensors were placed at the edge of range for a neighbouring sensor, then it is not guaranteed that transmissions could be made every time. We believe that using humans as an intermediate hop could be an interesting research opportunity. Using a mobile app to transfer data between nodes as they are within walking range could speed up data transfers and their knowledge could be used to prioritise data without the need to process it.

If a human views a recent observation when they are in range of a node, they could make an instant assessment on whether it is interesting or not. If they mark it as such, then it could be passed through the network with a higher priority and reach a DA node within a short period as it would not require processing, updating knowledge bases for future, similar observations as it is forwarded.

One important goal is to create a deployment of K-HAS for a different purpose than our motivating scenario in Malaysia. There is a need in Malaysia to track hunters in the forest and alert authorities, this would not require any change in how K-HAS is currently implemented but would show how it can dynamically adapt based to changes in its sensing requirements. We would also like to test K-HAS in a situation where power is no longer a limiting factor, but delivery time of sensed data would be. A building security network would be one such example, deploying K-HAS across a number of floors and processing video feeds to alert users within seconds about suspicious activity.

However, using K-HAS for a WSN that, for example, uses text based sensed data to monitor the temperature and lava level of a volcano is a completely different implementation, but the local knowledge could be used to prevent an emergency and predict eruptions. A deployment such as this would show that the benefit of using local knowledge in a WSN is not limited to our motivating scenario, but is versatile enough to benefit almost any network.

Using heterogenous sensor nodes within a K-HAS network would also show how local knowledge can be used in different streams, as well as how they can be combined to make more detailed inferences.  Using motion sensors with microphones could be used to determine what person/animal is near the sensor and this information can then be used to infer patterns as fallback sources when an image based classification cannot be made.

Our simulations in Chapter 7 show how K-HAS can use local knowledge, but it is by no means a complete implementation.  We need to implement a more modular simulation that allows K-HAS to be applied to any form of WSN. We would like to simulate individual knowledge bases on every node and experiment with networks that contain multiple central nodes to visualise the flow of sensed data through the network and see if there is an effect, positive or negative, on the speed of interesting sensed data. A more immediate goal is to test K-HAS when all of the network uses a communication medium with lower range but a higher transfer rate, such as Wi-Fi, to determine how much Zigbee slows the delivery of sensed data when compared with processing.

We would also like to experiment with different ratios of knowledge processing capabilities on nodes to determine if there is an ideal ratio that maximises the flow of sensed data, delivering interesting data first and quickly but also delivering data that has been classified as not interesting within a shorter time period that would allow humans to act on the data if it had been misclassified.
