\chapter{Introduction}
A wireless sensor network (WSN) \nomenclature{WSN}{Wireless Sensor Network}consists of a collection of heterogeneous nodes with sensing and, typically, wireless capabilities. These sensing nodes can be extremely complex and powerful devices with the ability to sense multiple phenomena simultaneously, or they can be simple motes that have limited processing power and are tasked with sensing one thing in their environment.

Upon deployment, these nodes use their wireless capabilities to form links with their neighbours, where a neighbour is any node that is within transmission range. The way that nodes discover, and communicate with, their neighbours is defined by their routing protocol. Routing protocols vary based on the purpose of the WSN, the requirements of data transmission as well as the characteristics of the nodes. Communication between nodes is expensive and drains the available power faster than any other action that a node performs. For example, if a WSN is deployed in a building with consistent power available, then the routing protocol does not need to be adapted to ensure the nodes maximise their battery life by transmitting as little as possible. However, not every WSN has unlimited resources at their disposal and these protocols, as well as the underlying structure of the network, are used to ensure the network is able to perform well for as long as possible.

Each WSN is different and each will have different constraints, a WSN that monitors traffic along a busy road may experience memory limitations, whereas a  WSN that is deployed in the middle of a desert may experience power issues. Typically, however, all WSNs do the same thing: sense one or more characteristics of their environment and forward that data on to a specified endpoint.

\section{The Local Knowledge Problem}
The majority of WSNs do not know what data they are sensing, or have any knowledge of their environment. This means that, unless fixed by a routing protocol or human that deployed the network, data is delivered on a chronological basis and is then filtered at the base station, usually manually. Some WSNs store all of the data on the node and users of the network must use a 'pull' model to query for data from nodes, but this requires some technical knowledge and, while it does increase the battery life of the nodes, it is a manual process again.

The environment that a WSN is deployed is usually rich and the data sensed often contains patterns that can be used to improve the performance of the network. For example, if a node knows that it is has only been triggering between the hours of 6pm and 5am for the past few weeks, it can enter a deep sleep outside of those hours or use that time to transmit data it has been storing while it knows it will be inactive. Alternatively, this knowledge can be used to prioritise data throughout the network so that the most important data is received first, instead of the most recent. An example of this could two camera nodes deployed facing the entry and exit of a building, tasked with looking for intruders between 5pm and 8am. If the camera facing the exit is triggered at 5:01pm and the camera on the entrance is triggered at 5:05pm, then the knowledge that the security guard leaves through the exit between 5:01pm and 5:08 pm will allow the entrance camera to prioritise its capture as more important, as it is an irregular occurrence.

This knowledge can be categorised into \textit{local} and \textit{global}. Local knowledge (LK) \nomenclature{LK}{Local Knowledge} is the knowledge of an area that has been gained through experience or experimentation and global knowledge (GK) \nomenclature{GK}{Global Knowledge}is knowledge that is generally available to everybody. An example of this is someone who has been tasked with deploying a WSN in the Amazon rainforest would use readily available sources, such as the Internet or prior research, to determine the humidity and weather patterns in order to use a node that could withstand such conditions. This would be classed as GK. However, a native to the Amazon may know that three of the locations that the nodes are to be deployed in are flooded for two weeks of the year, rendering their readings useless for that time period and increasing their risk of failure. This is LK, as it cannot be gained without experiencing the flooding in that area, or experimenting with water levels.

We believe that the use of this knowledge can increase the efficiency of the network, as well as prioritise sensed data by its value instead of the time it was recorded. To show this, we have developed a network architecture for WSNs that utilises knowledge from the data it senses, as well as its deployed environment. It is called the Knowledge-based Hierarchical Architecture for Sensing (K-HAS) \nomenclature{K-HAS}{Knowledge-based Hierarchical Architecture for Sensing} and this thesis will show how K-HAS addresses the problem of delivering the most important data first and improving the overall efficiency of the network.

\section{Motivation}
Throughout this thesis, we focus on a scenario motivated by our collaboration with Cardiff University School of Biosciences, who run a research centre in the Malaysian rainforest, in Sabah, known as Danau Girang (DG) \nomenclature{DG}{Danau Girang}. Located on the banks of the Kinabatangan river, DG has been running for more than six years and holds Masters, PhD and Undergraduate students from around the world, studying the ecology and biodiversity of the unique region.

The reason that the rainforest that DG is set in is so unique is that the area was heavily logged until the late 1970s and now serves as a corridor, between large Palm Oil plantations, connecting two separate rainforest lots. The area is now secondary rainforest (rainforest that has grown since being destroyed) and is experiencing a large variety of wildlife using the area as a habitat, or as a path. Some of this wildlife is unique to this area of the world and DG has had sightings of animals that have not been seen in many years.

There is a variety of research projects currently underway in the field centre, looking into fish population, crocodile attacks, hornbill habitats or the movement patterns of small mammals. One project that has been running almost since DG opened, is the \textit{corridor monitoring programme}, a programme that consists of dozens of wildlife cameras deployed in various areas around DG and triggered whenever an animal triggers a break in their infrared (IR) \nomenclature{IR}{Infrared} sensor.

The Kinabatangan is a very humid place, with thick forest, making it very difficult to walk through and even more difficult for hardware to survive the conditions. Cameras are placed along the river and up to 1km into the forest, recording triggers onto SD cards. These SD cards are collected and stored at the field centre, where the images are manually collated and processed. The cameras are designed to have a battery life of three months but, due to the humidity, a battery life of 3 weeks is more realistic. In 2010, twenty cameras were deployed and half of them were inspected every two weeks, on a rotating basis. In that time, each cameras can record more than a thousand pictures and the dynamic nature of the rainforest, such as the sun through leaves, falling trees and reflections in the water can cause the camera to trigger when an animal has not walked past; we call these \textit{false triggers}. False triggers can make up to 70% of the images on an SD card and each of these must be manually processed. 

We have used this scenario to test our hypothesis and implement a WSN that automates the collection, transmission, processing and storage of images, using LK to classify the data and prioritise the flow of information through the network, making more efficient use of the limited power and bandwidth available. 

\section{Research Contributions}

\section{Thesis Structure}
The rest of this thesis is structured as follows. Chapter 2 provides some background on wireless sensor networks and the use of knowledge. Chapter 3 explains some technical decisions we made and the findings when running experiments in the Malaysian rainforest. Chapter 4 introduces the K-HAS architecture we have developed and explains the purpose of each tier. Chapter 5 details the ontology we have developed to support K-HAS and shows how current ontologies do not sufficiently cover all of the concepts involved with a \textit{scientific observation}. Chapter 6 shows how we use rules to determine how valuable a piece of sensed data is and to explain how K-HAS is able to use these rules, and human feedback, in order to inform future classifications. Chapter 7 highlights the technical limitations of implementing K-HAS today and shows our simulations of it running as it was designed. Chapter 8 then concludes this thesis and summarises our contributions and findings, as well as highlighting work that could be undertaken to take this project further.