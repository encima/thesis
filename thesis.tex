% This thesis template has been put together and edited by Diego Pizzocaro
% It's a collection/edit of files produced by Frank C Langbein for his PhD thesis
% and then edited by Ahmed Alazzawi, Hmood Al Dossari, Manar I. Hosny, H.Z. Al-Dossari 
% and kindly shared by them.
%
% Following, the original GNU license chosen by Frank C Langbein, 
% which is the same license for distributing this package.
%
% beautification.tex -- Beautification of Reverse Engineered Geometric Models
%% PhD Thesis by Frank C. Langbein%
% Maintainer: Frank C. Langbein <frank@langbein.org>
% Version: 1.0
% Copyright (c) 2003 Frank C Langbein.
% Permission is granted to copy, distribute and/or modify this document
% under the terms of the GNU Free Documentation License, Version 1.2 or
% any later version published by the Free Software Foundation; with no
% Invariant Sections, no Front-Cover Texts, and no Back-Cover Texts. A
% copy of the license is included in the section entitled "GNU Free
% Documentation License".
%


\documentclass[a4paper,oneside,onecolumn,openright,12pt]{book}


\makeatletter

%
\usepackage{caption}
\usepackage{footmisc} 
\usepackage{paralist}
\usepackage{amsthm}
% \usepackage[utf8]{inputenc}
% \usepackage{subfig}
%\usepackage{algorithm}
%\usepackage{graphicx}
%\usepackage{rotating}
%\usepackage{amssymb} 


% Fonts, encoding, etc.
\usepackage{type1cm}
% \usepackage[latin1]{inputenc}
\usepackage[british]{babel}
\usepackage[T1]{fontenc}
\usepackage{times}
% \usepackage{hyperref}
\usepackage{longtable}
%\usepackage{algorithm} 
\usepackage[noend]{algorithmic}
%\usepackage{graphicx}
%\usepackage{rotating}
\usepackage{multirow}
\usepackage{float}


\usepackage[numbers]{natbib}

\usepackage[refpage]{nomencl} 
\usepackage{graphicx}
\usepackage{listings}
\usepackage{setspace}
\usepackage{xcolor}
\usepackage{caption}
\usepackage{rotating}
\usepackage{gensymb} 
\usepackage{tabularx}
\usepackage{subcaption}
\usepackage[hidelinks]{hyperref}
\usepackage[all]{hypcap}
\usepackage{color}
\usepackage[toc,page]{appendix}
\usepackage{tikz}

\definecolor{sh_comment}{rgb}{0.12, 0.38, 0.18 } %adjusted, in Eclipse: {0.25, 0.42, 0.30 } = #3F6A4D
\definecolor{sh_keyword}{rgb}{0.37, 0.08, 0.25}  % #5F1441
\definecolor{sh_string}{rgb}{0.06, 0.10, 0.98} % #101AF9

\DeclareCaptionFont{white}{\color{white}}
\DeclareCaptionFormat{listing}{%
  \parbox{\textwidth}{\colorbox{gray}{\parbox{\textwidth}{#1#2#3}}\vskip-4pt}}
\captionsetup[lstlisting]{format=listing,labelfont=white,textfont=white}
% \lstset{frame=lrb, xleftmargin=\fboxsep, xrightmargin=-\fboxsep, breaklines=true, basicstyle=\ttfamily\scriptsize}
\lstset {
 frame=ltrb,
 rulesepcolor=\color{black},
 showspaces=false,showtabs=false,tabsize=2,
 numberstyle=\tiny,numbers=left,
 basicstyle= \ttfamily\scriptsize,
 stringstyle=\color{sh_string},
 keywordstyle = \color{sh_keyword}\bfseries,
 commentstyle=\color{sh_comment}\itshape,
 captionpos=b,
 xleftmargin=0.7cm, xrightmargin=0.5cm,
 lineskip=-0.3em,
 float,
 floatplacement=H,
 escapebegin={\lstsmallmath}, escapeend={\lstsmallmathend},
 breaklines=true
}

%\usepackage{subfigure}
% Line spacing
\def\baselinestretch{1.5}
\parindent0cm
\parskip1.5ex\@plus.7ex\@minus.1ex\relax

%Acronyms
 \usepackage{acronym}
 %Inline lists
 \usepackage{paralist}

% Page dimensions
\usepackage{vmargin}
\setpapersize{A4}
\setmarginsrb{40mm}{20mm}{25mm}{30mm}{14.5pt}{8mm}{0pt}{11mm}

% Footer and header
\usepackage{afterpage}
\usepackage{fancyhdr}
\pagestyle{fancy}
\fancyhead{}
\fancyhead[LE,RO]{\thepage}
\fancyhead[LO,RE]{\slshape \leftmark}
\fancyfoot{}
\renewcommand{\chaptermark}[1]{}
\renewcommand{\sectionmark}[1]%
             {\markboth{\thesection\ #1}{\thesection\ #1}}
\renewcommand{\subsectionmark}[1]{}
\fancypagestyle{plain}{%
  \fancyhead{}
  \fancyhead[LE,RO]{\thepage}
  \fancyfoot{}
  \renewcommand{\headrulewidth}{.6pt}
}

% Chapter
\def\@makechapterhead#1{%
  \ \\[-35.5pt]\hbox to \textwidth {%
    \hfill {\vbox{\hbox{\rule[5pt]{140pt}{4pt}}%
        \hbox to 140pt {\hfill\huge\bfseries\slshape \@chapapp\space\thechapter\/}}}}%
  \vskip55\p@%
  {\parindent \z@ \raggedright \normalfont%
    \interlinepenalty\@M%
    \Huge \bfseries #1\par\nobreak%
    \vskip 45\p@%
  }}
\def\@chapter[#1]#2{\ifnum \c@secnumdepth >\m@ne
                       \if@mainmatter
                         \refstepcounter{chapter}%
                         \typeout{\@chapapp\space\thechapter.}%
                         \addcontentsline{toc}{chapter}%
                                   {\protect\numberline{\thechapter}#1}%
                       \else
                         \addcontentsline{toc}{chapter}{#1}%
                       \fi
                    \else
                      \addcontentsline{toc}{chapter}{#1}%
                    \fi
                    \chaptermark{#1}%
                    \addtocontents{lof}{\protect\addvspace{10\p@}}%
                    \addtocontents{lot}{\protect\addvspace{10\p@}}%
                    \addtocontents{loa}{\protect\addvspace{10\p@}}%
                    \if@twocolumn
                      \@topnewpage[\@makechapterhead{#2}]%
                    \else
                      \@makechapterhead{#2}%
                      \@afterheading
                    \fi}
\def\@schapter#1{\addcontentsline{toc}{chapter}{#1}%
                 \markboth{#1}{#1}%
                 \addtocontents{lof}{\protect\addvspace{10\p@}}%
                 \addtocontents{lot}{\protect\addvspace{10\p@}}%
                 \addtocontents{loa}{\protect\addvspace{10\p@}}%
                 \if@twocolumn%
                    \@topnewpage[\@makeschapterhead{#1}]%
                 \else%
                    \@makeschapterhead{#1}%
                    \@afterheading%
                 \fi}
\def\@makeschapterhead#1{%
  \ \\[-35.5pt]\hbox to \textwidth {%
    \hfill {\vbox{\hbox{\rule[5pt]{140pt}{0pt}\rule[5pt]{0pt}{4pt}}%
        \hbox to 140pt {\hfill\huge\bfseries\slshape \ \/}}}}%
  \vskip55\p@%
  {\parindent \z@ \raggedright \normalfont%
    \interlinepenalty\@M%
    \Huge \bfseries  #1\par\nobreak%
    \vskip 45\p@%
  }}

% Table of contents
\def\contentsname{Contents}
\renewcommand\tableofcontents{%
    \if@twocolumn%
      \@restonecoltrue\onecolumn%
    \else%
      \@restonecolfalse%
    \fi%
    \chapter*{\contentsname}%
    \@starttoc{toc}%
    \if@restonecol\twocolumn\fi%
    }

% Bibliography
\renewenvironment{thebibliography}[1]
     {\chapter*{\bibname}%
      \list{\@biblabel{\@arabic\c@enumiv}}%
           {\settowidth\labelwidth{\@biblabel{#1}}%
            \leftmargin\labelwidth
            \advance\leftmargin\labelsep
            \@openbib@code
            \usecounter{enumiv}%
            \let\p@enumiv\@empty
            \renewcommand\theenumiv{\@arabic\c@enumiv}}%
      \sloppy
      \clubpenalty4000
      \@clubpenalty \clubpenalty
      \widowpenalty4000%
      \sfcode`\.\@m}
     {\def\@noitemerr
       {\@latex@warning{Empty `thebibliography' environment}}%
      \endlist}

% Floats
\long\def\@makecaption#1#2{%
  \vskip\abovecaptionskip
  \sbox\@tempboxa{\textbf{#1: #2}}%
  \ifdim \wd\@tempboxa >\hsize
    \textbf{#1: #2.}\par
  \else
    \global \@minipagefalse
    \hb@xt@\hsize{\hfil\box\@tempboxa\hfil}%
  \fi
  \vskip\belowcaptionskip}
\renewcommand{\topfraction}{0.9}
\renewcommand{\textfraction}{0.1}
\renewcommand{\floatpagefraction}{0.9}

% Tables
\usepackage{dcolumn}
\usepackage{hhline}

% Graphics
% \usepackage[dvips]{graphicx}
% \usepackage[usenames,dvipsnames]{color}
\usepackage{rotating}
\usepackage{psfrag}
\usepackage{epic}
\usepackage{eepic}

% Algorithm environment
\newcounter{algorithm}[chapter]
\renewcommand{\thealgorithm}{\thechapter.\@arabic\c@algorithm}
\def\fps@algorithm{t}
\def\ftype@algorithm{1}
\def\ext@algorithm{loa}
\def\fnum@algorithm{Algorithm~\thealgorithm}
\newenvironment{algorithm}{\@float{algorithm}}{\end@float}
\newenvironment{algorithm*}{\@dblfloat{algorithm}}{\end@dblfloat}
\newenvironment{alevel}%
   {\begin{list}{}{%
      \setlength{\topsep}{0pt}%
      \setlength{\parskip}{0pt}%
      \setlength{\partopsep}{0pt}%
      \setlength{\parsep}{0pt}%
      \setlength{\itemsep}{0pt}}}%
   {\end{list}}
\newcommand\listalgorithmsname{List of Algorithms}
\newcommand\listofalgorithms{%
    \if@twocolumn
      \@restonecoltrue\onecolumn
    \else
      \@restonecolfalse
    \fi
    \chapter*{\listalgorithmsname}%
      \@mkboth{\listalgorithmsname}{\listalgorithmsname}%
    \@starttoc{loa}%
    \if@restonecol\twocolumn\fi
    }
\newcommand*\l@algorithm{\@dottedtocline{1}{1.5em}{2.3em}}

% List of figures and tables
\renewcommand\listoffigures{%
    \if@twocolumn
      \@restonecoltrue\onecolumn
    \else
      \@restonecolfalse
    \fi
    \chapter*{\listfigurename}%
      \@mkboth{\listfigurename}{\listfigurename}%
    \@starttoc{lof}%
    \if@restonecol\twocolumn\fi
    }
\renewcommand\listoftables{%
    \if@twocolumn
      \@restonecoltrue\onecolumn
    \else
      \@restonecolfalse
    \fi
    \chapter*{\listtablename}%
      \@mkboth{\listtablename}{\listtablename}%
    \@starttoc{lot}%
    \if@restonecol\twocolumn\fi
    }

% Math symbols, fonts, etc.
\usepackage{amsmath}
\usepackage{amsfonts}
\usepackage{amssymb}

\newcommand{\N}{\mathbb{N}}
\newcommand{\Z}{\mathbb{Z}}
\newcommand{\Q}{\mathbb{Q}}
\newcommand{\R}{\mathbb{R}}
\newcommand{\C}{\mathbb{C}}
\renewcommand{\S}{\mathbb{S}}
\renewcommand{\P}{\mathbb{P}}
\newcommand{\E}{\mathbb{E}}

\newcommand{\Cf}{\mathfrak{C}}
\newcommand{\Pf}{\mathfrak{P}}

\DeclareMathOperator{\sign}{sign}
\DeclareMathOperator{\avg}{avg}
\DeclareMathOperator{\floor}{floor}
\DeclareMathOperator{\ceil}{ceil}
\DeclareMathOperator{\round}{round}

\providecommand{\abs}[1]{\lvert#1\rvert}
\providecommand{\absd}[1]{\left\lvert#1\right\rvert}
\providecommand{\card}[1]{\lvert#1\rvert}
\providecommand{\norm}[1]{\lVert#1\rVert}

% URLs
\usepackage{url}
%% Define a new 'leo' style for the package that will use a smaller font.
\makeatletter
\def\url@leostyle{%
  \@ifundefined{selectfont}{\def\UrlFont{\sf}}{\def\UrlFont{\footnotesize\ttfamily}}}
\makeatother
%% Now actually use the newly defined style.
\urlstyle{leo}



\makeatother


\begin{document}


\frontmatter


\begin{titlepage}

\begin{center}
\vspace*{3ex}
\textbf{\Huge Using Local and Global Knowledge}\\[2ex]
\textbf{\Huge in Wireless Sensor Networks}\\[12ex]
\textbf{\large A thesis submitted in partial fulfilment}\\[1ex]
\textbf{\large of the requirement for the degree of Doctor of
  Philosophy}\\[16ex]
\textbf{\LARGE Christopher Gwilliams}\\
\vfill
\textbf{\LARGE \today}\\
\vfill
\textbf{\LARGE Cardiff University}\\[1ex]
\textbf{\LARGE School of Computer Science \& Informatics}\\[4ex]
\end{center}

\end{titlepage}
\newpage\thispagestyle{empty}\cleardoublepage


\thispagestyle{plain}

\vspace*{6ex}

\textbf{\large Declaration}

This work has not previously been accepted in substance for any degree and is not concurrently submitted in candidature for any degree.\\[2ex]
Signed \dotfill \ (candidate) \hspace*{10em}\\[1ex]
Date\ \ \ \ \ \dotfill \hspace*{18em}

\vfill

\textbf{\large Statement 1}

This thesis is being submitted in partial fulfilment of the requirements for the degree of PhD.\\[2ex]
Signed \dotfill \ (candidate) \hspace*{10em}\\[1ex]
Date\ \ \ \ \ \dotfill \hspace*{18em}

\textbf{\large Statement 2}

This thesis is the result of my own independent work/investigation,
except where otherwise stated. Other sources are acknowledged by
explicit references.\\[2ex]
Signed \dotfill \ (candidate) \hspace*{10em}\\[1ex]
Date\ \ \ \ \ \dotfill \hspace*{18em}

\vfill

\textbf{\large Statement 3}

I hereby give consent for my thesis, if accepted, to be available for photocopying and for inter-library loan,
 and for the title and summary to be made available to outside organisations.\\[2ex]
Signed \dotfill \ (candidate) \hspace*{10em}\\[1ex]
Date\ \ \ \ \ \dotfill \hspace*{18em}

\vfill

\cleardoublepage

\thispagestyle{plain}
\ \vfill{\small
Copyright \copyright\ 2014 Christopher Gwilliams.\\
Permission is granted to copy, distribute and/or modify this document
under the terms of the GNU Free Documentation License, Version 1.2 or
any later version published by the Free Software Foundation; with no
Invariant Sections, no Front-Cover Texts, and no Back-Cover Texts. A
copy of the license is included in the section entitled ``GNU Free
Documentation License''.}\\[3.5ex]
A copy of this document in various transparent and opaque
machine-readable formats and related software is available at
\url{http://yourwebsite}.
\cleardoublepage


%\ \vspace*{1.11cm}
%\markboth{Dedication}{}
%\begin{flushright}
%\textbf{\large To People you care}\\
%\textbf{\large for their patience and support.}
%\end{flushright}
%\newpage
%\markboth{}{}
%\cleardoublepage


\chapter*{Abstract}

%\def\baselinestretch{1}\normalfont

\def\baselinestretch{1.5}\normalfont

Wireless sensor networks (WSNs) have advanced rapidly in recent years and the volume of raw data received at an endpoint can be huge. We believe that the use of local knowledge, acquired from sources such as the surrounding environment, users and previously sensed data, can improve the efficiency of a WSN and automate the classification of sensed data. We define local knowledge as knowledge about an area that has been gained through experience or experimentation. With this in mind, we have developed a three-tiered architecture for WSNs that uses differing knowledge-processing capabilities at each tier, called the Knowledge-based Hierarchical Architecture for Sensing (K-HAS). A novel aligning ontology has been created to support K-HAS, joining widely used, domain-specific ontologies from the sensing and observation domains. We have shown that, as knowledge-processing capabilities are pushed further out into the network, the profit - defined as the value of sensed data - is increased; where the profit is defined as the value of the sensed data received by the end user.

Collaborating with Cardiff University School of Biosciences, we have deployed a variation of K-HAS in the Malaysian rainforest to capture images of endangered wildlife, as well as to automate the collection and classification of these images. Technological limitations prevented a complete implementation of K-HAS and an amalgamation of tiers was made to create the Local knowledge Ontology-based Remote-sensing Informatics System (LORIS). A two week deployment in Malaysia suggested that the architecture was viable and that, even using local knowledge at the endpoint of a WSN, improved the efficiency of the network. A simulation was implemented to model K-HAS and this indicated that the network became more efficient as knowledge was pushed further out towards the edge, by allowing nodes to prioritise sensed data based on inferences about its content.

\chapter*{Acknowledgements}

My eternal gratitude goes to Alun Preece and Alex Hardisty for their guidance, assistance, support, enthusiasm and faith in me throughout the research and writing of this thesis. Their encouragement of testing new ideas, questioning everything and ensuring I step out of the box that it seems all too easy to stay inside. I am also grateful for the skills they have pushed me to gain, from teaching to presenting to professional writing. None of this work would have been possible without you both.

Benoit Goosens, Mike Bruford, Danica Stark and all of those involved with Danau Girang Field Centre have been fountains of knowledge, support and inspiration. At the very least, I thank you for your hospitality each year and to the staff members that put up with a computer scientist that is clearly not well adapted to life in the rainforest with limited connectivity to the outside world. 

I am grateful to the School of Computer Science \& Informatics, staff and students have provided methods of growth, new skills and support throughout my time at Cardiff University. My thanks go to Matt Williams, for challenging my work from the early stages right through to completion and providing relief through films, and Will Webberley, a friend gained through the PhD that provided support, both professionally and personally, and I do not believe I could have completed this work without him. Special mentions must go to Martin Chorley, Jonathan Quinn, Gualtiero Colombo, Konrad Borowiecki, Diego Pizzocaro, Przemyslaw Woznowski, Rich Coombs, Liam Turner, Mark Greenwood and Matthew John for being there in the pub, as well as in academia.

My deepest thanks also go to family and close friends, with special mentions to Natasha Kinsley-York for supporting me throughout the start of the PhD and helping me deal with the stress incurred and Chris Fellows, for his assistance and also being a sounding board for my many rants, ideas and everything in between. Finally, to Katie Lewis, for her ideas, support, patience and all-round awesomeness that has not only helped my work, but ensured I enjoy every second of my down time.

I would like to dedicate this work to my father, who passed away in 2009, there is no way I would be anywhere close to this point without his input, teaching and encouragement pushing me to go further than I ever imagined I could go. Words cannot express the love and gratitude I have for Robert and Gina Gwilliams, two role models that truly set the bar high. 



\tableofcontents

\newpage
\chapter*{List of Publications}

The work introduced in this thesis is based on the following publications.

\begin{itemize} 

\item \cite{gwilliams2012k} - Gwilliams, C., Preece, A., Hardisty, A., Goossens, B., \& Ambu, L. N. (2012). Local and Global Knowledge to Improve the Quality of Sensed Data. International Journal of Digital Information and Wireless Communications (IJDIWC), 2(2), 164--180.

\item \cite{gwilliams2012poster} - Gwilliams, C., Preece, A. D., \& Hardisty, A. R. (2012). Poster: using local and global knowledge in wireless sensor networks. In Proceedings of the 10th international conference on Mobile systems, applications, and services (pp. 515--516).

\item \cite{gwilliams2012local} - Gwilliams, C., Preece, A., Hardisty, A., Goossens, B., \& Ambu, L. N. (2012). K-HAS: An Architecture for Using Local and Global Knowledge in Wireless Sensor Networks. In The International Conference on Informatics and Applications (ICIA2012) (pp. 331--341).

\end{itemize}

\listoffigures

\listoftables

\lstlistoflistings

% \listofalgorithms

\chapter*{List of Acronyms} 
\begin{acronym}
\acro{API}{Application Programming Interface}
\acro{BLE}{Bluetooth Low Energy}
\acro{DA}{Data Aggregation}
\acro{Darwin-SW}{Darwin Core Semantic Web}
\acro{DC}{Data Collection}
\acro{DGFC}{Danau Girang Field Centre}
\acro{DP}{Data Processing}
\acro{DwC}{Darwin Core}
\acro{DwC-A}{Darwin Core Archive}
\acro{EML}{Ecological Metadata Language}
\acro{EPO}{Extension Plug-in Ontologies}
\acro{EXIF}{EXchangeable Image Format}
\acro{FN}{False Negative}
\acro{FP}{False Positive}
\acro{GBIF}{Global Biodiversity Information Facility}
\acro{GEAR}{Geographical Energy Aware Routing}
\acro{GK}{Global Knowledge}
\acro{GSN}{Global Sensor Networks}
\acro{HK}{High Knowledge}
\acro{IFTTT}{If This Then That}
\acro{INSIGHT}{INternet-Sensor InteGration for HabitaT monitoring}
\acro{IoT}{Internet of Things}
\acro{IR}{Infrared}
\acro{JVM}{Java Virtual Machine}
\acro{K-HAS}{Knowledge-based Hierarchical Architecture for Sensing}
\acro{LK}{Local Knowledge}
\acro{LKWS}{Lower Kinabatangan Wildlife Sanctuary}
\acro{LOS}{Line of Sight}
\acro{LORIS}{Local knowledge Ontology-based Remote-sensing Informatics System}
\acro{MCFA}{Minimum Cost Forwarding Algorithm}
\acro{MK}{Minimal Knowledge}
\acro{NK}{No Knowledge}
% \acro{O \& M}{Observations \& Measurements}
\acro{OBOE}{Extensible Observation Ontology}
\acro{OGC}{Open Geospatial Consortium}
\acro{OpenCV}{Open Computer Vision}
\acro{OWL}{Web Ontology Language}
\acro{QoS}{Quality of Service}
\acro{ROI}{Region of Interest}
\acro{RDL}{Ruleset Definition Language}
\acro{SBC}{Single Board Compuetr}
\acro{SDO}{Sensor Data Ontology}
\acro{SensorML}{Sensor Model Language}
\acro{SHO}{Sensor Hierarchy Ontology}
\acro{SOS}{Sensor Observation Service}
\acro{SPIN}{Sensor Protocol for Information via Negotiation}
\acro{SPS}{Sensor Planning Ontology}
\acro{SSN}{Semantic Sensor Network}
\acro{SUMO}{Suggested Upper Merged Ontology}
\acro{SWE}{Sensor Web Enablement}
\acro{TEEN}{Threshold sensitive Energy Efficient sensor Network}
\acro{TN}{True Negative}
\acro{TOVE}{Toronto Virtual Enterprise}
\acro{TP}{True Positive}
\acro{Wi-Fi}{Wireless Fidelity}
\acro{WSN}{Wireless Sensor Network}
\end{acronym}

\mainmatter
%begin each chapter chapter1.tex by \Chapter{chapter1}, for example, then the sections and so on...
\chapter{Introduction}
A wireless sensor network (WSN) \nomenclature{WSN}{Wireless Sensor Network}consists of a collection of nodes with sensing and, typically, wireless communication capabilities. These sensing nodes can be complex and powerful devices with the ability to sense multiple phenomena simultaneously \cite{Maurer, Nachman2008, Sarajevo2014}, or they can be simple motes with limited processing power \cite{Martinez2004, Kays2009, Szewczyk2004b}. A node has one or more sensors attached to it, with a wireless radio that is used to transmit the sensed data to an endpoint, where an endpoint is the base station of the network. Most WSNs use a single base station, providing a single endpoint, but there are networks with multiple endpoints; either used for different types of sensed data or to ensure that the load of the network is spread out.

Upon deployment, these nodes use their wireless capabilities to form communication links with their neighbours, where a neighbour is any node that is within transmission range. The way that nodes discover, and communicate with, their neighbours is defined by their routing protocol. Routing protocols vary based on the purpose of the WSN, the requirements of data transmission as well as the characteristics of the nodes. Communication between nodes is expensive and drains the available power faster than any other action that a node performs \cite{Raghunathan2002}. For example, if a WSN is deployed in a building with constant power availability, then the routing protocol would not need to be modified to ensure nodes sleep to conserve battery or disable their radios for a period of time. However, not every WSN has unlimited resources at their disposal and these protocols, as well as the underlying structure of the network, are used to ensure the network is able to perform well for as long as possible.

Each WSN is different and each will have different constraints, a WSN that monitors traffic along a busy road may experience memory limitations, whereas a  WSN that is deployed in the middle of a desert may experience power issues. Typically, however, all WSNs do the same thing: sense one or more characteristics of their environment and forward that data on to a specified endpoint; sometimes called the base station and typical WSNs only have one endpoint.

\section{The Local Knowledge Problem}
While large in size, it is typical that a WSN would use low cost nodes with limited power, memory and computational capabilities; causing them to lack the ability to be aware of their surroundings or the data that they are sensing \cite{Akyildiz2002}. This means that, unless fixed by a routing protocol or a technician, data is delivered on a chronological basis and is then filtered at the base station, usually manually. Some WSNs store all of the data on the node and users of the network use a `pull' model to query for data from nodes \cite{Sadagopan}, but this requires some technical knowledge and, although it does increase the battery life of the nodes, it is a manual process again.

The environment that a WSN is deployed in is often a rich source of data to be sensed (such as inside a bird's nest to sense temperature, humidity and movement), which often contains patterns that can be used to improve the performance of the network. For example, if a node knows that it is has only been triggering between the hours of 6pm and 5am for the past few weeks, it can then learn to enter a deep sleep outside of those hours or use that time to transmit data it has been storing while it assumes it will be inactive, based on previous days. Alternatively, this knowledge can be used to prioritise data throughout the network so that the most important data is received first, instead of the most recent. An example of this could be two camera nodes deployed facing the entry and exit of a building, tasked with looking for intruders between 5pm and 8am. If the camera facing the exit is triggered at 5:01pm and the camera on the entrance is triggered at 5:05pm, then the knowledge that the security guard leaves through the exit between 5:00pm and 5:08 pm will allow the entrance camera to prioritise its capture as more important, as it is an irregular occurrence.

This knowledge can be categorised as either \textit{local} or \textit{global}. Local knowledge (LK) \nomenclature{LK}{Local Knowledge} is the knowledge of an area that has been gained through experience or experimentation \cite{Joshi2001} and global knowledge (GK) \nomenclature{GK}{Global Knowledge}is knowledge that is generally available to all persons. For example, a researcher who has been tasked with deploying a WSN in the Amazon rainforest would use readily available sources, such as the Internet or prior research, to determine the humidity and weather patterns in order to use a node that could withstand such conditions. This would be classed as GK. However, a native to the Amazon may know that three of the locations in which the nodes are to be deployed are flooded for two weeks of the year, rendering their readings useless for that time period and increasing their risk of failure. This is LK, as it cannot be gained without experiencing the flooding in that area, or measuring local water levels.

We believe that the use of this knowledge can increase the effectiveness of the network, as well as prioritise the transmission of data that is potentially of the highest value instead of the time it was recorded. To show this, we have developed a network architecture for WSNs that utilises knowledge from the data it senses, as well as its deployed environment. It is called the Knowledge-based Hierarchical Architecture for Sensing (K-HAS) \nomenclature{K-HAS}{Knowledge-based Hierarchical Architecture for Sensing} and this thesis will show how K-HAS addresses the problem of delivering the most important data first and improving the overall efficiency of the network.

\section{Motivation}\label{int:mot}
Throughout this thesis, we focus on a scenario motivated by our collaboration with Cardiff University School of Biosciences, who run a research centre in the Malaysian rainforest, in Sabah, known as Danau Girang Field Centre (DGFC) \nomenclature{DG}{Danau Girang Field Centre} \cite{dgfc}. Located on the banks of the Kinabatangan river, DGFC has been running for more than six years and holds Masters, PhD and Undergraduate students from around the world, studying the ecology and biodiversity of the unique region.

The rainforest surrounding the Kinabatangan river is unique because the area was heavily logged until the late 1970s and the river now serves as a corridor, between large palm oil plantations, connecting two separate rainforest lots. The area is now secondary rainforest (rainforest that has grown since being destroyed) and is experiencing a large variety of wildlife using the area as a habitat, or as a path. Some of this wildlife is unique to this area of the world and DGFC has had sightings of animals that have not been seen in many years \cite{Goossens2012}.

There is a variety of research projects currently underway in the field centre, looking into fish population, crocodile attacks, hornbill habitats or the movement patterns of small mammals. One project that has been running almost since DGFC opened, is the \textit{corridor monitoring programme}, a programme that consists of dozens of wildlife cameras deployed in various areas around DGFC and a set of images are taken whenever an animal triggers a break in their infrared (IR) \nomenclature{IR}{Infrared} sensor.

The Kinabatangan is a protected wildlife reserve, with thick and humid forest, making it very difficult to walk through and even more difficult for hardware to survive the conditions. Cameras are placed along the river and up to 1km into the forest, capturing images when triggered and saving them onto SD cards. These SD cards are collected and stored at the field centre, where the images are manually collated and processed. The cameras are designed to have a battery life of three months but, due to the humidity, a battery life of three weeks is more realistic. In 2010, twenty cameras were deployed and half of them were inspected every two weeks, on a rotating basis. In that time, each camera can record more than a thousand pictures and the dynamic nature of the rainforest, such as the sun through leaves, falling trees and reflections in the water can cause the camera to trigger when an animal has not walked past. The events are known as \textit{false triggers}, and they can make up to 70\% of the images on an SD card and each of these must be manually processed. 

Each trigger of the camera results in a set of three images being captured, this number has been chosen by the researchers at DGFC after some experimentation. Three images allows for more than one image of slow-moving animals to be captured and, typically, allows for the middle image in the sequence to capture the body of fast-moving animals; with the first and last capturing the head and rear respectively.

We have used this scenario to test our hypothesis and implement a WSN that automates the collection, transmission, processing and storage of images, using LK to classify the data and prioritise the flow of information through the network, making more efficient use of the limited power and bandwidth available. 

\section{Research Contributions}

Here we outline the main research contributions explained in this thesis. We believe that the two main contributions have been made and our experimental results serve to support these contributions.
Our primary contribution is that we propose a \textbf{novel tiered network architecture where each subsequent tier possesses increased knowledge-processing ability}, K-HAS, that utilises the local knowledge of its surrounding environment, users and previously sensed data to process observations within the network and prioritise the data according to the inferred classification. 

Knowledge is pushed out to the edge of the network to allow the nodes that capture observations to prioritise the data based on its content. The knowledge processing capabilities increase with each tier as the data moves toward the centre of the network, allowing more detailed inferences to be made and data to be given a higher priority. This smart utilisation of bandwidth allows data to be delivered in an order that is more useful than chronological, delivering the most valuable observations first and using human feedback to learn how important these observations were. 

In addition, K-HAS uses a feedback loop to dynamically update the knowledge base on every node throughout the deployment so it is able to react to changes within the data recorded, and the network, in near real-time.

Experimental results from the Malaysian rainforest showed that current long-range sensor technology is not yet at the point where nodes could be deployed in a harsh environment and left without human intervention for months at a time. A modified K-HAS system LORIS (Local knowledge Remote-sensing Ontology-based Informatics System) combined tiers that use knowledge-processing at the centre of the network, with commercially available, wireless cameras. This showed that knowledge-processing automates the handling of sensed data when it is received and can be used to infer patterns for future observations. LORIS was designed because we could not find robust, open sensing nodes that can send data over long distances and handle high humidity and extreme temperatures. Because of this, we used off-the-shelf (OTS) wireless cameras that ran a closed, while providing some access through proprietary software. We then combined two of K-HAS' tiers so that processing only took place at the base station, matching more of a traditional WSN topology; while still making use of local knowledge gained from previously sensed data.

We model the ideal implementation of K-HAS in a simulation environment, along with variations on the knowledge processing capbilities of each node, We use these simulations to show that LK and GK can prioritise sensed data effectively and that this prioritisation increases the efficiency of the network by delivering data based on its importance, rather than chronologically.

Our second contribution is what we believe to be the first \textbf{ontology that combines, and extends, ontologies used in multiple domains}. To formally define the structure of K-HAS and the data standard used, we developed an ontology that combined existing ontologies for sensor networks and scientific observations. This modular ontology can be used as a whole or in parts for WSNs that use some, or all, of K-HAS' architecture. On top of this, we extended these ontologies by adding specific terms relevant to the different node types and users within the K-HAS architecture. The extensibility of the ontology allows parts to be removed and replaced where necessary.


\section{Thesis Structure}
The rest of this thesis is structured as follows. Chapter 2 provides background on wireless sensor networks and the use of knowledge-based technologies in that context. Chapter 3 explains technical decisions we made and the findings when running experiments in the Malaysian rainforest. Chapter 4 introduces the K-HAS architecture we have proposed and explains the purpose of each tier. Chapter 5 details the ontology we have proposed to support K-HAS and shows how current ontologies do not sufficiently cover all of the concepts involved with a \textit{scientific observation}. Chapter 6 details our implementation in the Malaysian Rainforest and the changes we had to make in order for it to be feasible. Chapter 7 describes how we modelled K-HAS, and other scenarios, in a simulation environment. The results are then explained and we explain how different scenarios are suited to WSNs, in a multitude of environments, with different requirements. Chapter 8 then concludes this thesis and summarises our contributions and findings, as well as highlighting work that could be undertaken to take this project further.

\chapter{Background}\label{chap:bg}
Using knowledge in a WSN is related to existing research into sensor networks that utilise context-awareness in order to improve their effectiveness or adapt their sampling rate. An example of a context-aware sensor is an accelerometer attached to a node that is able to determine what the readings of the accelerometer indicate. For example, a smart phone with an accelerometer may use context-awareness to determine if the user is running, walking or going upstairs. Some of these actions may be more important than others and, thus, they can be prioritised. However, researching context-aware WSNs alone would limit our knowledge of WSNs in general and affect decisions we make when designing our own architecture. There is research that is valuable for all types of sensor network, such as: hardware design, routing protocol, tranmission medium choice or middleware used.

This chapter is split into the following sections. Section \ref{bg:wsni} outlines the issues surrounding WSN design and deployment. 
Section \ref{bg:rp} details relevant existing routing protocols for sensor networks that are used for a number of different purposes, from extending network lifetime by scheduling sleep patterns to storing sensed data on nodes and responding to queries. Section \ref{bg:sm} highlights commonly-used sensor middleware. Section \ref{bg:bsn} shows some examples of existing WSNs that are related to our motivating scenario.  Section \ref{bg:lgk} introduces research into local and global knowledge and Section \ref{bg:conc} summarises the results of our research.
 % and Section \ref{bg:rsn} shows related work into WSNs that utilise knowledge or context-awareness.

\section{Wireless Sensor Network Issues} \label{bg:wsni}

WSNs have been used in a number of domains, for a range of different purposes, from habitat monitoring \cite{Szewczyk2004a} to military purposes \cite{Pizzocaro} and healthcare \cite{Otto2006}. While these applications are different, the technology behind each is similar. Each requires the use of nodes with sensors attached and each node requires a power source and storage devices.

According to \cite{Akyildiz2002}, there are at least eight factors that affect the design of sensor networks, but we focus on a subset that are the most relevant to our research problem. Those we have given less consideration to are: production costs, scalability and topology. While these are important, productions costs are not a concern for us in the research stage and scalability has been considered when studying existing networks. We have considered the following points in greater detail:

\subsection{Fault Tolerance}
	WSNs typically contain a large number of nodes and any node can fail for various reasons, from a lack of power, filling its storage capacity, to factors in the environment causing the hardware to fail. While the hardware architecture of sensor nodes is typically similar, the variation between each deployment means that the device itself must be adapted to its environment. For example, \cite{Mainwaring2002} used a custom protective casing for their nodes so that they were able to survive being in the open while ensuring that the transmission range was not affected.


\subsection{Hardware Constraints}
	A sensor node typically consists of: a platform that contains the memory and processing power,  a sensor (or sensors) and a transceiver that uses a wireless standard, such as Wi-Fi or Zigbee. Cost and size are the most significant considerations when designing a WSN. According to \cite{Intanagonwiwat2000}, the expectation of a sensor node is a matchbox-sized form factor. However, ten years ago, there was much research focussed on \textit{smart dust} \cite{Kahn}. Smart dust is small, inexpensive, disposable nodes that can transmit until their power reserve is depleted, \cite{Akyildiz2002a} highlights that it is a requirement for the nodes to cost less than USD10. In \cite{Corke2010a}, a decade on from the first WSN papers, smart dust has not been realised and the focus has instead been on larger, more powerful nodes that have reduced in cost and grown in power. The Gartner Hype Cycle for 2013 \cite{gartner2013} showed that smart dust is still in early innovation stages and may not be fully commercialised for another ten years. To counter this, research has been focussed on using software solutions to maximise the battery life in these more powerful, more expensive nodes, accompanied by the use of renewable energy sources.

\subsection{Energy Constraints}
	 Commonly, sensor nodes do not have access to a constant power supply and must run on a battery that is, generally, a similar size to the node. Larger batteries must usually be contained in a separate enclosure to the node and this makes nodes less compact and their deployment more difficult. This means that the nodes must be as efficient as possible, knowing when to transmit data and when to sleep. The lifetime of a sensor network is highly dependent on the battery life of each node and, unlike other mobile devices, they cannot typically be recharged \cite{Akyildiz2002}. Much work has been done on power efficient routing protocols, as well as the control of which attached devices are active \cite{Hempstead2005, Schurgers, Segal2010a}. 

	The limited resources on the nodes mean that the sensing devices, and transceivers, attached must consume as little power as possible. Some routing protocols implement turning off wireless radios and scheduling a wakeup across the network \cite{Vaidya2004}, but the cost of turning off a device can waste just as much energy as leaving it on and sampling at a lower rate, if not more \cite{Estrin2001}.

	The use of energy in a node is dependent on how active the sensor(s) are, how much it transmits and receives, the transmission medium used as well as the environment it is in.

\subsection{Transmission Medium}\label{bg:trans}
	Widely used, general purpose transmission media, such as Wi-Fi, are viable solutions in WSNs when a high data rate is required and power is readily available. However, research has shown that Wi-Fi is extremely power-hungry and \cite{Lee2007} shows that Wi-Fi consumes almost 9 times more energy, while transmitting, than other standards, such as Zigbee.
	Bluetooth is a more power-efficient standard that is becoming increasingly popular for sensing devices that are part of the `Quantified Self' movement \cite{Swan2012}, with wearable devices that report measurements, such as heart rate, steps taken and calories burned. With the advent of the new low-power Bluetooh 4.0 , also known as Bluetooth Low Energy (BLE), this standard is supposed to allow months of continuous use on a coin-cell battery \cite{gomez2012overview}. However, the theoretical max range is 100m and, using the same frequency, as Wi-Fi (2.4GHz) means that it is as susceptible to path loss and reduced transfer rates.
	\cite{Zennaro} shows that a 2.4GHz Wi-Fi antenna is capable of transmitting up to 350m, while a considerably lower frequency of 41MHz was able to achieve links of 10km. The use of 2.4GHz frequencies in wet conditions have been shown to reduce the performance by up to 28\% \cite{Markham2010}, while humidity in the air can reduce the performance by up to 78\% \cite{Figueiredo2009}.
	New low-power, low-frequency standards have emerged in recent years and allow for a considerably longer range and increased battery life, at the cost of transmission speeds. Digimesh is an example of this and, while it can achieve 250kb/s using 2.4GHz, it has much slower speeds of 125kbps when using the 900MHz spectrum. However, it does offer a range of, up to, 64Km \cite{Bayat2012}.

\subsection{Environment}
	The environment that a node is deployed in can have a great impact on almost all aspects of a WSN, such as range or the operational lifetime of the node itself. Harsh environments that are not easily accessible make it difficult to place nodes and protected environments may limit where nodes can be placed. 
	In \cite{Martinez2004}, nodes were deployed within glaciers and had to survive extreme temperatures, lasting without human intervention, for months at a time. \cite{Martonosi2003} attached collars to Zebras that had to withstand high speed movement, impact, dust and high temperatures. The deployment of any node requires extensive research as to the environment that it will be deployed in and adjustments must be made to ensure it is able to survive for extended periods without continued maintenance.
	Section \ref{bg:trans} also shows that environment does not simply affect the hardware, but humidity can reduce the transmission range significantly, as well as moisture collecting on wireless antenna can reduce the range for days at a time.

	Limited range in a WSN can be addressed by using \textit{intermediate nodes} - nodes tasked with forwarding data from other sensing nodes that would otherwise be out of range - between disconnected sensing nodes. While they do not need to sense the environment directly, they can be vital in ensuring data from all areas of the network are delivered. In the case of \cite{Martonosi2003}, a car was used to drive near to zebras that had not passed close enough to the base station in order to transmit their data. The car acted as the intermediate node by collecting data from the zebra's collar and relaying it back to the base station. In a normal WSN, this has the downside of requiring more nodes, which increases the cost and the chance of a node failing, but it can connect clusters in a geographical region that would otherwise be unable to send sensed data, or they can provide multiple routes through the network to help preserve battery life.

\section{Routing Protocols} \label{bg:rp} 
	Routing protocols specify how nodes in a WSN are organised, as well as how they transmit data throughout the network. In \cite{Akkaya2005}, the more popular routing protocols are surveyed and split into three of the main identified categories: data-centric, hierarchical and location-based. We use the aforementioned categories, as well as flat, to highlight some of the key protocols that are relevant to our work. Flat routing protocols are for WSNs that involve many nodes deployed over a large area, all sending data to a single endpoint. Data-centric is a protocol that focusses on sensed-data, where nodes advertise their contents and query other nodes to fulfill requests made by a user. Location-based protocols use the physical region that nodes are deployed in to request and send data. Hierarchical protocols are for heterogeneous networks that perform more than one task. For example, a network spread of over hundreds of miles could be split into individual networks, or it could follow a hierachical structure where clusters of nodes send data to an gateway node  (also known as a \textit{cluster head}); a node that may not directly sense the environment but acts as an endpoint for a group of nodes \cite{Akkaya2005}. The data then hops across other nodes in order to reach a final endpoint.
	The protocols have the task of ensuring that a network is performing at its best, providing the best lifetime and ensuring reliable and consistent delivery of data. This must reduce \textit{flooding}, where nodes send every message to every link, aside from themselves, effectively flooding the network with unnecessary messages, and find a way to deliver data to the endpoint using the most efficient path possible.

\subsection{Flat}
	Initially, this was the most common structure for a WSN, dozens of nodes spread out over a geographical area, with one or more neighbours, sending observations to a single endpoint.

\subsubsection{MCFA} \label{bg:rp:mcfa}
	The Minimum Cost Forwarding Algorithm (MCFA) is a flat routing protocol that works by assigning costs to each node, based on how many hops they are from the endpoint \cite{Ye2001}. 

Each node has a path-estimate of the cost of transmission from itself to the base station. The base station sends out a broadcast message and it is received by all nodes in range. The message contains a cost from the base station (initially zero) while every node has their cost set to infinity. The cost is stored at the node, incremented and sent on to all nodes in range. If the received cost is less than the current cost stored on the node, then the cost (and the neighbour) is updated and passed on.

Each message has a cost associated with it, which is based on the hops it has completed so far. A node that receives the message forwards it only if it's cost matches the sum of the source node's cost (contained within the message) and the message's current cost. This ensures that all messages are sent through the minimum cost path, without storing explicit path information on each node.
% If the cost in the message, plus the link it was received, is less than the current cost. If yes, the estimate is update on both the node and the message; the message is then passed on to other nodes in range. 

	This approach allows for dynamic reconfiguration of the network, as well as a reduced overhead due to not having to maintain a global routing table on each node. The assumption with MCFA, however, is that the direction of routing is always towards a fixed endpoint.

\subsection{Data-centric}
	Data-centric routing protocols are not like traditional WSNs where nodes are given addresses; they use a method that involves the advertising, or querying, of the data that has been sensed and those with the relevant data can respond to the request.

\subsubsection{SPIN}
	Sensor Protocols for Information via Negotiation (SPIN) \nomenclature{SPIN}{Sensor Protocols for Information via Negotiation} is one of the first data-centric protocols and attempts to address the issue of flooding the network whenever new data is sensed by addressing the data through metadata \cite{Heinzelman1999}. SPIN works on three messages passed between nodes:
	\begin{enumerate}
		\item ADV - A message sent by a node when it has sensed new data, advertising what it has recorded.
		\item REQ - Sent by nodes that received an ADV to request the data.
		\item DATA - Message containing the sensed data.
	\end{enumerate}

When a node has sensed data, it sends an ADV message to all nodes within range. If any of those nodes are interested in the data, then they respond with a REQ message, at which point the DATA message is sent to nodes that responded.

SPIN eliminates the need for a global view of the network topology, as nodes only need to know their single hop neighbours. However, SPIN does not guarantee equal diffusion of data throughout the network as a node may be interested in the data sensed at the other edge of the network, with only nodes that are not interested in between. This would mean that those nodes would not request the data or pass it on.

\subsubsection{SPIN-IT}
	An extension to SPIN, SPIN-IT \nomenclature{SPIN-IT}{Sensor Protocols for Information via Negotiation - Image Transfer} uses a slightly different approach to receiving data and was developed solely for the transfer of images \cite{Woodrow2002}, using metadata to fulfill requests.
	
Nodes use the existing message structure of SPIN, but REQ messages are used as queries, sent to all nodes in transmission range. The receiving nodes keep these requests and generate a new REQ message, thus allowing nodes to store temporal paths. When a REQ reaches a node that has the desired data, it responds with a ROUTE-REPLY message. This message is used because images are large and resource-constrained WSNs would have a much shorter lifetime if a lot of unnecessary transmissions were made. The ROUTE-REPLY is used in case multiple nodes, in range of the requesting node, have the data it has requested and it can then choose the optimal route. As each node keeps a history of REQ messages, these can be used to trace the requested data back through the network, to the originating node, without the overhead of maintaining a global routing table.

\subsubsection{COUGAR}
	A slightly different data-centric approach is the proposed COUGAR protocol, viewing the network as a distributed database. Cougar is similar to SPIN because it does not forward data as soon as it is sensed, instead COUGAR uses a query language that abstracts the underlying network structure and uses received queries to generate a plan that utilises in-network processing to provide an answer based on the sensed data stored on all deployed nodes \cite{Yao2002}. 

	For example, in a building monitoring network, a user could query a base station for offices that are unoccupied. The base station then sends that query to all nodes in range and it is then dispersed throughout the network. Nodes that have data that can satisfy the query send back their results and these are combined as they move back through the network to the base station. The user then receives that data, along with the nodes that have provided it.

Within the network, a \textit{leader} is selected and this node is used to aggregate the data from nodes that were able to fulfil all, or some, of the query. At risk of failure, each query should result in a \textit{leader} being dynamically selected and it must have sufficient resources to be able to satisfy the request. This protocol was only proposed, and much of the technical detail has yet to be completed, but the concept of treating the network as a distributed database is a novel idea and this is one of the first protocols to suggest the use of a query language that could be used by people without specific technical knowledge, which, in this case, is technical knowledge of query languages and how the routing protocol works.

\subsection{Hierarchical}
	Hierarchical networks are WSNs that contain nodes of different classes; nodes at the edge of the network are typically clustered into groups and served by a gateway node. This gateway could be in charge of aggregating the data, processing the data, or simply forwarding it to an endpoint. Clusters of nodes allow the network to be spread out over a wider geographical area and gateway nodes can use a different transmission method to provide long distance links to the base station. Gateway nodes serving a cluster of nodes means that the network can scale easily as well, simply by adding a new cluster to the network.

\subsubsection{TEEN}
	The Threshold sensitive Energy Efficient sensor Network (TEEN) protocol is designed for reactive sensor networks, networks that require instant reactions to changes sensed in their environment \cite{Manjeshwar2001}. TEEN recognises that transmission is the most power hungry action for a node so each node is coded with a hard and soft threshold. The hard threshold is a value that makes nodes transmit the reading to their cluster head. Similarly, the soft threshold is a small change in the value of the sensed attribute that causes further transmissions.

During the initialisation of the network, the base station sends information about the thresholds and sensing attributes to all cluster heads in the network; the cluster heads then forward this on to all nodes in their cluster.  When a node senses data over the hard threshold, it transmits to the cluster and only transmits again when new sensed values are greater than the hard threshold and the difference between the current sensed value and the previous is greater than the soft threshold \cite{Manjeshwar2001}.

Clusters are assigned for a period of time and then new clusters are selected by the base station, at which point new attributes and thresholds are broadcast to all nodes. This kind of protocol allows the network to be dynamic after deployment and allows user input based on the data that has been sensed in the previous cluster times. 

For example, a network could be tasked with sensing humidity in a rainforest but the thresholds have been set such that nodes are transmitting readings that are not of interest. A user can change these thresholds and they will be pushed out to the nodes at the time that the next clusters are chosen, without any need to visit the node or configure them individually.

\subsection{Location-based}
	Instead of using the physical addresses of nodes, or the data they store, location-based protocols are based on the region that nodes are deployed in. Location-based routing relies on the fact that each node is aware of its own location and is also aware of the destination's location.

\subsubsection{Span}
	Span is a protocol where nodes are selected as \textit{coordinators} based on their positions. A node can decide to be a coordinator based on the amount of energy it has and the number of neighbouring nodes it would benefit if they were able to use it as a bridge \cite{Chen2002}.

An example of this would be node B placed between node A and C. C and A are unable to communicate directly so, when node B wakes up, it decides whether it should become a coordinator. It knows that it has sufficient energy levels and it can provide connectivity for a previously disconnected area of the network, so it chooses to become a coordinator, staying awake and routing sensed data to other coordinators, which form the backbone of the network.

Results showed that using Span, in a system that transmits using 802.11, provides an network lifetime increase of more than a factor of 2 over networks that just use the 802.11 protocol.

\subsubsection{GEAR}
The Geographic Energy-Aware Routing protocol (GEAR) is similar to SPAN in that it makes routing choices based on both energy-awareness and location. Each node maintains an \textit{estimated cost} and a \textit{learning cost} of forwarding a packet through its neighbours. The estimated cost is calculated using the distance to the packet destination and the energy remaining on the node whereas the learning cost is the estimated cost that takes holes in the network into consideration \cite{Yu2001}. 

GEAR is designed to perform in two phases: forwarding a packet towards a region and disseminating a packet within a region. When sending a packet towards a destination, GEAR either sends a packet on to the node in range that is closest to the destination or, if such a node does not exist, then a hole is identified. If a hole is identified then the node that minimises a cost is selected.

To disseminate a packet within a geographic area, GEAR uses algorithms based on the density of the network. Recursive geographic forwarding is typically used but this can result in an endless loop if the density of the network means that the region is unable to contact the destination. In that case, restrictive flooding is used.

\subsection{Conclusion}
Routing protocols can define the topology of a network, how data is sent and have an impact on the network lifetime by determining when nodes should sleep, when they should transmit sensed data and when (or if) they should request updates on the network topology. Researching these types of routing protocol allows us to determine the situations in which each category would be used and whether they fit our requirements. The routing protocol must be considered when developing a network architecture, so we have researched the four main categories and the more commonly-used protocols within those categories. If we chose a flat network structure, then MCFA would be the better choice but our architecture changes significantly if we choose to use a data-centric protocol. 

Using this background knowledge, we can select a protocol that fits with our needs, or combine useful aspects of many. Data-centric protocols are useful for battery conservation and networks that do not require real-time reports. Location-centric is useful as it does not require specific node addressing but location-aware nodes are more expensive and power-hungry. Hierarchical, however, allows for in-network data-aggregation or processing but requires the use of a heterogenous network with a more rigid topology of nodes split into clusters.

With the knowledge of the requirements of a WSN architecture, we can pick the routing protocol that best suits these, or combine traits from multiple protocols in order to develop a hybrid protocol that fits our requirements exactly.
\section{Sensor Middleware} \label{bg:sm}
	Acting as a bridge between the hardware and the user, sensor middleware is software that abstracts the underlying network from the user and provides a means of accessing sensed data and administrating how the network performs \cite{Hadim2006}. These middlewares must not be specific to a single network and provide support for as many different sensor nodes as possible. In \cite{Yu2004}, a middleware is said to provide standardised services to many applications and perform operations that make effective use of limited system resources.

	In \cite{Wang2008c}, a middleware should include four major components: programming abstraction, system services, runtime support and Quality of Service (QoS) mechanisms. In this section, we will discuss the challenges surrounding middlewares for WSNs and highlight some existing middleware that are particularly relevant to our research problem and motivating scenario.

\subsection{Issues}\label{bg:sm:issues}
	WSNs present a range of new challenges to existing middleware, due to their resource constraints, deployment environments and more. However, there has been research into the key issues that must be addressed in order for middleware to be considered suitable.
	While there have been a number of surveys into these challenges \cite{Hadim2006, Rahman, Yu2004}, we will detail those that we believe to be most relevant to our work. Some of those that, while considered, have not been a primary issue are: dynamic network organisation, security and application knowledge.
	\subsubsection{Energy Constraints}
		It is rare that nodes in a WSN would have a constant power source, unlimited memory and a casing that can survive a harsh environment without decaying. In order to ensure that the lifetime of nodes is maximised, middleware needs to offer a power scheduling system that makes efficient use of the hardware on the node, generally disabling the radio after a set interval has elapsed, or use a combination of lower power sensors to provide sensed data of a similar quality \cite{Heinzelman2004}.

		Ideally, a middleware will be able to coordinate nodes through wireless communication, making efficient use of transmissions and dynamically modifying sleep schedules based on the power remaining.
	\subsubsection{Heterogeneity}
		Not every node in a network will have the same capabilities, manufacturer or hardware. WSN middleware needs to provide a standard interface to the applications that are making use of it, whilst at the same time accommodating differences arising from hardware coming from different manufacturers and perhaps having significantly different capabilities. Some middleware have been built for a specific set of hardware \cite{Tengg2007}, however this homogeneity can provide an increase in the performance and efficiency of the network by only supporting a limited number of devices.
	\subsubsection{Real-world Integration}
		WSNs are often tasked with recording phenomena that are time-crucial, so a sensor middleware should provide a real-time interface to the data that it has sensed \cite{F2006}. Ideally this data would be available outside of the network, though the use of an API.
	\subsubsection{Quality of Service}
		This issue is perhaps the most complex as QoS could apply to almost all aspects of the networks, such as efficiently using bandwidth, 100\% uptime for nodes, guaranteed packet delivery or access to data stores. Some of these requirements are managed by the implementation of the routiong protocol, the middleware should be able to monitor deployed nodes and report on their current status, as well as identify failures.

\subsection{Existing Middleware}\label{sec:middleware}
	In this section, we identify existing middleware, explain how they address the issues highlighted in Section \ref{bg:sm:issues} and highlight how they relate to our research. While there are a lot of existing middleware, our research did not show any that used information abouts its environment or knowledge from previously sensed data to process data that is currently being sensed. We did, however, find some middleware solutions that utilise context and rules to administrate the network.

	\subsubsection{GSN}\label{sec:GSN}
		The Global Sensor Networks middleware (GSN) has been developed to manage heterogeneous sensor networks and be suitable for those without any technical knowledge \cite{F2006}.

GSN provides hardware abstraction through the use of  \textit{virtual sensors}, a data stream that abstracts implementation details from the actual sensed data. A virtual sensor can be comprised of many streams and it can even consist of many virtual sensors. 

Virtual sensors are described using XML, with tags that consist of metadata for the sensor, the structure of the incoming data stream, SQL queries for processing the incoming data and querying times. What makes GSN stand out is that virtual sensors do not have to be sensors deployed within your network, or even sensors at all, some examples of GSN show virtual sensors being added that read in data from the weather websites. This also means that the underlying structure of the network is irrelevant to GSN, as well as the physical locations of the nodes. Unlike some middlewares that have an expectation of how data will be routed, GSN is decoupled from the routing protocol, allowing them to act independently.

GSN is completely open source and the Java code can be modified to suit a specific deployment. 

		\begin{figure}[h]
		\centering
		\includegraphics[width=0.8\textwidth]{Chap2/figures/gsn_arch.png}
		\caption{GSN Architecture \cite{F2006}}
		\label{bg:fig:gsn}
		\end{figure}

Figure \ref{bg:fig:gsn} outlines the architecture of GSN, showing that the virtual sensors are stored on a central node and their inputs are managed and stored. GSN also comes bundled with a web interface to show all active sensors and their most recent recordings, as well as the implementation of web services to access the data outside of the interface.

Data from virtual sensors pass through the virtual sensor manager to the storage layer. Once the data has been stored, the query manager is invoked and queries are loaded from the repository and executed by the manager. The results of the queries are then handled by the notification manager and also made available to the web interface. Notifications can be extended to support many different forms of communication, such as SMS, email or web services.

Virtual sensors do not natively support all hardware, although new virtual sensors can be described using XML, and there may be a need to implement an entirely new virtual sensor. In this case, technical knowledge is required, and new sensors can be implemented through use of the Java programming language. This provides more control over the use of XML and allows users to specify how sensed data is stored in a database, use external libraries to receive proprietary data, specify processing workflows before the data is stored or implement new notification methods for the users of the network.

To show the simplicity of a basic virtual sensor, \cite{Aberer2007} describes a temperature sensor that we reproduce here in Listing \ref{bg:lst:gsn:vsensor}. The file is human-readable and does not require specialist knowledge when compared with programming languages, with tags that have self-explanatory names. In this example, the output structure shows that only a temperature reading is received and that the data should be stored permanently. The stream source specifies the content of the stream and the query details the standard query that should be used to extract data from GSN.
\vspace{\baselineskip}
\begin{lstlisting}[caption={Example Virtual Sensor},label={bg:lst:gsn:vsensor}]
<life-cycle pool-size="10" />
<output-structure> 
	<field name="TEMPERATURE" type="integer"/> 
</output-structure>
<storage permanent-storage="true" size="10s" /> 
<input-stream name="dummy" rate="100" > 
<stream-source alias="src1" sampling-rate="1" storage-size="1h">
<address wrapper="remote"> <predicate key="type" val="temperature" /> <predicate key="location" val="bc143" /> </address> 
<query>select avg(temperature) from WRAPPER</query>
</stream-source>
<query>select * from src1</query> 
</input-stream>

\end{lstlisting}

The modularity and flexibility of GSN makes it different to existing middleware as it has not been designed for any specific hardware and modules of the middleware can be replaced, such as the database. 

	\subsubsection{FACTS}
		One such example is the FACTS middleware, an approach that uses a fact repository to coordinate nodes. Rules can then be implemented to process sensed data and fired when certain conditions are met \cite{Terfloth2006}. More traditional sensor middleware controls the network and manage sensed data but this rule based approach allows for more flexibility, where rules can control the transmissions and process the data upon receipt.

		\begin{figure}[h]
		\centering
		\includegraphics[width=0.8\textwidth]{Chap2/figures/facts_architecture.jpg}
		\caption{FACTS Architecture \cite{Terfloth2006}}
		\label{bg:fig:facts}
		\end{figure}

		Figure \ref{bg:fig:facts} shows the FACTS architecture, with the middleware holding the rule-sets and a distributed fact repository. Data within the network is stored as facts, providing a standard data format throughout the network and hardware abstraction. When new facts are received, usually because of new sensing data, the rule engine checks to determine whether any rules should be fired. The rule-set definition language (RDL) is used here and each rule-set contains a group of relevant rules. Each rule is given a priority so that, if more than one rule is triggered by a fact, then the higher priority rules are fired first.

		Listing \ref{bg:facts:rule} shows a FACTS rule, written in Haskell, that determines which geographic areas are covered by nodes. The name of each rule has no prefix, statements are prefixed with `->' and conditions are prefixed with `<-'. The \textit{sendRange} rule runs after a timer expires and removes that timer. Lines 4 to 12 set a \textit{rangeFact} fact that contains the range it expects to be able to cover. The rest of the rule sends the fact to all nodes within range and sets a new fact to show that it has sent its range information. 

		The \textit(xyMinCovered) is a simplified rule that, upon receipt of range information from neighbouring nodes, checks whether the range of the neighbouring node overlaps with its own range and the final \textit{determineCoverage} rule then stores that coverage information in a fact as knowledge of whether or not it is the only node to cover a particular geographic region. This can then be used to inform future routing and sleeping decisions.
\vspace{\baselineskip}
		\begin{lstlisting}[caption={Coverage Algorithm in FACTS Rules}, label={bg:facts:rule}]
 sendRange
 <- Exists Timer.expiredSlot
 -> Retract Timer.expiredSlot 
 -> Define "rangeFact"
 -> Set ("rangeFact" "xMin")
	(posXSlot - System.txRadiusSlot)
-> Set ("rangeFact" "xMax") 
	(posXSlot + System.txRadiusSlot)
-> Set ("rangeFact" "yMin")
	(posYSlot - System.txRadiusSlot)
-> Set ("rangeFact" "yMax")
	(posYSlot + System.txRadiusSlot)
-> Send 0 System.txPowerSlot 
	("rangeFact" [(("rangeFact" "owner")
	== nodeIDSlot)])
-> Define "rangeSendFact"

xyMinCovered
<- Exists "rangeSendFact"
<- Eval ((posXSlot - System.txRadiusSlot)
	< ("rangeFact" "xMin"))
<- Eval ((posYSlot - System.txRadiusSlot)
	< ("rangeFact" "yMin"))
-> Define "xyMinCoveredFact" 

determineCoverage
<- Exists xyMinCoveredFact"
-> Define "coveredFact
		\end{lstlisting}

		While FACTS itself does not utilise any local knowledge, the repository is used as a source for all previously sensed data and would be an excellent source of knowledge to assist with the classification of future readings. Also, the ability to add new rule-sets, without technical knowledge of the hardware of each node means that users of the network have the ability to add knowledge in the form of less technical, high-level rules.

	\subsubsection{ITA Sensor Fabric}
	The ITA Sensor Fabric is a collaboration project between IBM, the US Army and the UK Ministry of Defence. Sensor Fabric, or Fabric, is a two-way messaging bus and set of middleware services connecting network assets to each other and users \cite{Wright2009b}.

The core difference between the Fabric middleware and others is that not every node is sensing all of the time, sensor nodes are tasked when there is a requirement and they stop as soon as that task has been fulfilled. Similar to sinks in a traditional WSN, Fabric utilises Fabric nodes, which run the following three pieces of software:
\begin{enumerate}
	\item Message Broker - Provides the communication infrastructure.
	\item Fabric Registry - Holds information about the current deployment, such as all nodes deployed, all assets, routing information and tasks. Deployed in the form of a database.
	\item Fabric Manager - The main service on the node to track the status of connected sensors, establish communication channels, provide a container for processing, plug-ins and to extends the capabilities of the Fabric.
\end{enumerate}

Fabric runs on a Publish/Subscribe model, a sensing requirement is sent to a messaging broker as a subscription and this is distributed through all Fabric nodes and, thus, all sensor nodes. Sensor nodes then publish their data and the relevant data is sent to all applications that have subscribed to the data. 

The plugin structure of Fabric makes it stand out from existing middlewares, allowing its functionality to be extended through web interfaces.

Because Fabric has been developed for military purposes that cross countries, policy enforcement has been implemented to restrict access to the granularity of sensed data but these access levels do not simply apply to a military context. Using our motivating scenario, researchers and professors should see animal images whereas the Sabah Wildlife Department should see images of hunters and people in the forest.

	\subsubsection{Sensor Web Enablement}
		The Open Geospatial Consortiums's (OGC) Sensor Web Enablement (SWE) is a set of standards to allow developers to make sensors and sensed data repositories accessible via the Internet \cite{botts}. While this is not a middleware, these standards can work with existing middleware in order to publish their sensed data to the Internet. The SWE framework consists of:

		\begin{itemize}
			\item Observations \& Measurements (O\&M) - General models and XML encodings for observations and measurements.
			\item Sensor Model Language (SensorML) - Standard models and XML schema for describing the process within sensor systems.
			\item PUCK - Defines a protocol to retrieve a SensorML description and other information from a device, enabling automatic sensor installation, configuration and operation.
			\item Sensor Observation Service (SOS) - Open interface for a web service to obtain observations and sensor descriptions for one or more sensors.
			\item Sensor Planning Service (SPS) - An open interface for a web service that a user can determine the feasibility of collecting data from one or more sensors and submit collection requests.
		\end{itemize}

		These components can be integrated into existing middleware, clients and sensors and not all need to be used, some middleware, such as MufFIN \cite{Valente2011}, have just used SOS and O\&M but 52North have developed an open source implementations of all components of SWE \cite{broring2011new}.

	\subsubsection{Internet of Things}
		In \cite{giusto2010internet}, the Internet of Things (IoT) is described as the ``concept of pervasive things... which, through unique addressing schemes, are able to interact with each other and cooperate with their neighbours to achieve common goals.'' The number of devices that are part of the IoT grows daily, from thermostats and smoke detectors \cite{nest} to activity monitors \cite{fitbit}. Some of these devices have their data siloed away and it is only accessible through a web interface or a mobile app, but most have an Application Programming Interface (API) that allows the raw data to be access and queried. Because of this, sites like If This Then That (IFTTT) \cite{ifttt} have been created to link together various services and use simple if statements to act on changes in the data. IFTTT uses \textit{recipes} that follow the rule of: \lstinline{IF Service1 changes THEN perform action on Service2}. For example, \lstinline{IF I take a picture using my phone THEN upload it to Twitter} or \lstinline{IF a new item appears on my RSS feed THEN add it to my reading list}. While, IFTTT is not a traditional middleware, it does monitor user's connected services, some of which may be sensors around the home or attached to their person.

		A more traditional middleware for the IoT is Xively, a cloud platform that allows devices to upload their sensor readings to an endpoint and view readings from others \cite{xively}. Feeds can be public or private and Xively can be used with almost any Internet-enabled devices, such as webcams, temperature sensors or alarms. Other sites, such as Dweet.io \cite{dweet} and Open Sen.se \cite{opensense}, have also been in operation for a few years that provide a similar service to Xively. Open Sen.se combines the recipes of IFTTT with the storage of Xively to create a middleware that supports actions being performed when certain sensed data is received. For example, a temperature sensor could trigger an alert on Open Sen.se that would alert a user via e-mail, or through social media. This combination of devices with online services and applications allows a middleware to do more than simply store data, it can execute rules over the data, provide visualisations and create links between devices separated by hundreds, or thousands, of miles.


\section{Biodiversity and Environmental Monitoring Sensor Networks} \label{bg:bsn}
	In this section we will cover existing WSNs that are related to our motivating scenario or, more specifically, biodiversity focussed WSNs that have been deployed to monitor wildlife and/or the environment. WSNs for habitat, and wildlife, monitoring are especially important because these are areas that often need to be untouched by humans. Areas with high human disturbance can influence the abundance of species and some habitats, i.e. underground burrows, may be impossible to monitor without destruction. 
	
	One of the most well known WSNs to monitor habitat is the network deployed on Great Duck Island, an island off the coast of Maine, USA. A hierarchical network of 32 nodes was deployed to monitor a bird, known as the leach's storm petrel \cite{Mainwaring2002}. This network used a clustering approach for groups of nodes to send data to a gateway node, which would then route it back to the base station. The base station, located a few kilometres away on the island, has internet access and uploads the data to allow users to browse and process the data.

	A multihop approach was used here as they found that, for sufficient coverage, single hop connectivity would not cover all of the island. Acrylic enclosures were developed to ensure the nodes were weatherproofed for the conditions of the island, while maintaining the functionality of each sensor and not impeding transmission range. While the nodes, their casing and their sensors have been designed specifically for the deployment on Great Duck Island, the success of the network, running for 123 days in the early stages of WSN research \cite{Szewczyk2004c}, shows that this approach can be used elsewhere with similar effects; allowing hard to monitor and/or inaccessible areas to be continuously monitored.

	On a smaller scale, INternet-Sensor InteGration for HabitaT monitoring (INSIGHT) is a single-hop WSN that allows remote access for data and reconfiguring of nodes \cite{Demirbas}. Using off the shelf hardware, their findings show that their nodes could survive for 160 days on a single battery, supporting their claim that a single hop network allows for a longer network lifetime. 
	
	The key feature of this network is the ability for humans to remotely set reporting thresholds for sensor nodes. This means a user can prolong the lifetime of nodes by limiting the threshold they report on, as well as the fact that these thresholds are a way for users to add knowledge, albeit primitive, into a network.

	While there is research on cameras used to monitor animals \cite{Kays2009, Ahumada2011a}, these networks are generally cameras deployed with their memory cards manually retrieved and processed. In recent years, however, the use of wireless technologies and image-based WSNs has increased, \cite{Garcia-Sanchez2010b} uses wireless cameras to monitor the movement of animals between roads. Using commercial hardware and controlled sleep scheduling, this solution employs the use of nodes to detect movement and wake up more power-hungry camera nodes. While the nodes are wireless, the distance of the network from civilisation means that the data does still need to be collected manually and uploaded to a computer.

	Due to the advent of smartphones and tablets, as well as the improvements in 3G technology, projects taking advantage of more modern technologies have grown in popularity. Using 3G enabled cameras, \cite{ZSL} have deployed a number of devices in locations all over the world, such as: Kenya, Indonesia and the USA. The images captured are transmitted to a server and a website allows the general public to not only see the images in near real-time, but to classify the images as well. This crowd-sourcing of collective knowledge lets people, that may not have domain knowledge, vote on an image and those votes are used to make classification easier.

	Over the past fifteen years, WSNs have grown from a concept to a real solution for monitoring the habitats, movements and eating habits of wildlife all over the world. Whether it is using GPS collars to monitor the movement of cattle \cite{Guo2006}, monitoring animal habitats on a remote island or using cameras to capture the animals themselves, the popularity of these networks has grown considerably and advances in technology have allowed these networks to be deployed in places that humans cannot access with ease.
	
		There are a number of situations where we may want to record data in an environment that is not safe for humans, such as a volcano \cite{Werner-Allen2006}, or that may be difficult for electronics to survive. In these cases, special considerations must be made during the design and deployment of the WSN, in order to ensure maximum  network lifetime with reliable readings.
	\subsubsection{GLACSWEB}
	Glacsweb is a sensor network to monitor the rate at which glaciers are melting \cite{Martinez2004}. Deployed in Norway, specially designed sensors have been drilled into glaciers to monitor pressure, temperature, orientation and strain. Due to the high pressure and exposure to moisture, a polyester casing was used so that, once bonded, the node inside would be protected from its environment, but also preventing it from being recoverable.
	\subsubsection{Reventador}
	Volcano Reventador is in northern Ecuador and a WSN has been deployed on there to monitor eruptions. Similar to Glacsweb, weatherproof enclosures were used to prevent ash and moisture from breaking the sensors and long-range external antenna mounted to a pole was used to achieve communications over large distances when wireless communications have proven to be difficult \cite{Werner-Allen2006}.
	\subsubsection{Rainforest}
	There have been many studies on how the rainforest affects the range and quality of wireless links \cite{Figueiredo2009, Wark2008, Rahman2008}. In \cite{Figueiredo2009}, the humidity was shown to reduce 802.11 range by up to 78\% and \cite{Wark2008} explains that periods of rainfall reduce the link quality up to 100\% in some cases, resulting in the loss of a hop and this could prevent data from reaching the endpoint.


These examples highlight some of the difficulties of deploying WSNs in harsh environments. Environments vary from freezing glaciers to humid rainforests to dry deserts with extreme temperature variation and the sensors cannot disturb their surroundings, be too conspicuous or require regular human attention that would disturb the area and affect the wildlife. 

% \section{Local and Global Knowledge} \label{bg:lgk}
% 	The environment of a sensor network is rich and varied and we believe that knowledge of this environment and patterns in the data sensed can be used to inform the network on decisions surrounding the transmission and processing of newly sensed data. As our research began, we simply called this knowledge but, as our work continued, it became apparent that it could be categorised further.

% While we believe that we are the first to use the concept of local and global knowledge within the wireless sensor network domain, the terms have been around for many years. In 1999, a book that referred to local knowledge as \textit{indigenous knowledge} defined local knowledge as `systematic information that remains in the informal sector, usually unwritten and preserved in oral traditions rather than text' \cite{LadislausM.Semali}. 

% Over the past twenty years, local knowledge has been used in various contexts, from researching lending and the credit market \cite{Stiglitz1990} to extracting local knowledge from natives to improve farming techniques \cite{DEWALT}. This research, as well as work that will be covered later, showed us that there are two kinds of knowledge: global and local. 

% It was from agriculture research that we were able to refine our definition of local knowledge, \cite{Joshi2001} defines local knowledge as `knowledge that farmers have derived locally through experience and experimentation'. They also say that indigenous knowledge is different in that it is culturally specific. From this definition, as well as our work with our motivating scenario, we were able to generalise the definition and expand upon it.

% We now define local knowledge as \textit{knowledge of an area, held by a domain expert, that has been gained through experience or experimentation}. The weather of a region is global knowledge because it can be found through a variety of media, whereas the saturation levels of the soil in a field would be local knowledge as it would require experimentation to gain that knowledge. 
% %Local knowledge could become global if it was readily available but some of these data are held in tribes of people without Internet access.

% Using this definition, we believe that \textbf{utilising local and global knowledge within a sensor network can inform routing decisions to make better use of the bandwidth in resource-constrained WSNs by sending data that the node believes to be important first, rather than chronologically}. Patterns in the data, and knowledge of the environment surrounding a node, can allow a node to infer what data may be valuable, automate the processing of adding context to the sensed data, learn from previously sensed data and utilise global knowledge of ongoing projects within the network to determine what data is thought to be of a higher priority.

% \section{Relevant Existing Networks} \label{bg:rsn}
% 	In this section, we discuss existing networks that are related to our research question and/or motivating scenario. While most of the existing networks relevant to Danau Girang have already been covered in Section \ref{bg:bsn}, there are some WSNs that are not directly related to biodiversity but have been deployed in harsh conditions or involve interdisciplinary collaboration. Interdisciplinary networks are of particular interest because they include the crossover of specialist knowledge from two, or more, domains. For example, biodiversity networks combine computer networking and software development and bioscience while medical networks combine medical knowledge. This means that the networks must be developed with the operation of users that may not possess the same specialist knowledge in mind and, thus, need to be more generally accessible. These examples prove invaluable when designing our own WSN architecture as some middleware and hardware solutions have been created with ease of use as a priority and do not require technical knowledge, allowing them to be used with little training.

% \subsection{Context-Awareness}
% 	While standard WSNs have been prevalent for many decades, new research on the Internet of Things \cite{Atzori2010} has brought about interest in context-aware sensing, in some cases this is for small wearable devices to track fitness but the applications are much broader. Here we will look at WSNs that use context to make informed routing decisions, save power, or prioritise the transmission of data. We consider context-aware sensors the step before a network making use of local and global knowledge and the uses of knowledge in these scenarios will only enhance the data further.

	% \subsubsection{Health}
% 		Health monitoring is one of the more obvious choices for context awareness as classifying readings can often help determine the health of someone, rather than their self-reports. AlarmNet is a WSN that uses context to provide long-term health monitoring for people in assisted-living environments \cite{Wood}.
	
% AlarmNet employs context-awareness to learn about the activity levels of the patient and uses that knowledge to determine when changes in the readings may mean that the patient is at risk. Once the AlarmNet system has completed its learning period, deviations, from what it has recorded as the normal activities for the patient, are sent to nurses and doctors, with the idea that this information can assist with a diagnosis.

% As some nodes in the system will be battery powered, AlarmNet also employs a subsystem, called the Context-Aware Power Management System (CAPM), to control the power consumption of a device based on the recorded activities of the patient. The system uses policies, based on the context, to save power for all nodes in the system. For example, the system could put mains-powered nodes into a low-power state and disable all nodes outside of the bedroom when it detects that the patient has gone to sleep.

% This use of context has not only shown that it can assist with the lifetime of a network, but it can also provide valuable insight into sensed data to provide data enriched with semantics. AlarmNet is a particularly important example because of the learning period it employ, using a time period to add context to the sensor readings from a patient so that it is able to identify future activities that do not fit with the day-to-day pattens of the patient.

% 	\subsubsection{Wearable Devices}
% As wearable devices, such as the Fitbit and the Pebble smartwatch, are increasing in popularity, context-awareness is a useful tool to differentiate between the many activities that a person can undertake. The MObile PErsonal Trainer system (MOPET) is a wearable fitness devices that uses context to record data on jogging and fitness exercises \cite{Buttussi2008}. 

% While the project is five years old now, it is one of the earliest proof-of-concept devices created and shows how simple rules applied to sensor readings can be used to apply context. For example, the device consists of a GPS sensor and, when active, it is constantly recording positions. Using pairs of points, it is able to calculate the speed and, if the speed is consistent with jogging, then it records a running exercise.

% Fitness is not the only purpose for wearable computing, the eWatch is an early design for a \textit{smartwatch} that uses a microphone and light sensor to record the locations that the user visits, storing previous recordings in flash memory and matching them with current recordings \cite{Maurer}.

	% \subsubsection{UNETS}
	% 	*UNETS is tiered and context aware

\section{Conclusion}\label{bg:conc}
	In this chapter we have explored the components that make a WSN, as well as some existing deployments that are relevant to our motivating scenario. Sensor middleware have come a long way in the past decade and increased capabilities for sensor nodes have allowed for more intense processing to be carried out before any data is seen by a human. Context is now used on sensor nodes to infer the activity they are undertaking or even to determine when it should wake to sample. 

Local knowledge is not a new concept and it has been used for many years to extract information from indigenous people and in industry to adapt their processes to a local area, such as farming. However, we believe that our work is the first to apply local knowledge in the context of WSNs. Context-aware sensor networks have gone some way to support our hypothesis that some knowledge can increase the functionality of a network and enhance the quality of the sensed data, but we believe that the addition of local and global knowledge will only increase this functionality further.

The two primary components of a WSN that could be injected with local knowledge is the middleware and the routing protocol, each providing different benefits. Existing work has shown that the use of context-awareness in sensor middleware allows them to make dynamic, global changes to the network (Section \ref{bg:sm}), such as power management, whereas routing protocols affect the data that is sampled and sent to the endpoint(s) of the network, such as adaptive thresholds.

One issue in proving this hypothesis is the deployment of a network that utilises local knowledge. The deployment environment of our motivating scenario is not only interdisciplinary, it is in a region that is humid, dynamic and dense. Existing research has shown that the deployment of a WSN in these conditions means that range will be greatly reduced \cite{Figueiredo2009} and changes in humidity can prevent communication altogether, as well as moisture affecting the hardware itself.
To deploy a network, hardware must be adapted and the right medium must be chosen in order to maximise link quality and minimise dropped connections.

In Chapter 3, we describe how we used this research, along with our own findings, to choose suitable hardware for our network. As well as this, we used the results from previous rainforest range tests in the literature, to design  experiments that would aid us in choosing, and verifying, our choice of transmission medium.


\chapter{Technical}\label{chap:technical}
	In this chapter, we explain our motivating scenario in more detail, explore the sensor hardware that we researched and outline the results of experiments we undertook in the Malaysian rainforest. As highlighted in Section \ref{int:mot}, we have been working with Cardiff University School of Bioscience to design and deploy a WSN that utilises local knowledge, using an area of rainforest in Malaysia owned by the Sabah Wildlife Department, called Danau Girang.

The structure of this chapter is as follows. Section \ref{tech:motiv} explains what we aimed to deploy in Danau Girang and what our considerations were. Secion \ref{tech:hw} introduces sensor hardware that is in use today and details the choices we made. Section \ref{tech:wireless} details the transmission medium choices we tried and also shows the results of experiments performed in both the UK and Malaysia. Finally, Section \ref{tech:conc} summarises our findings and explains the choices we made for the sensor nodes we used in DG. 

\section{Danau Girang}
	Based in Sabah, Malaysia, Danau Girang is a field centre located in Lot 6 of the Lower Kinabatangan Wildlife Sanctuary (LKWS), surrounded by secondary rainforest that had been logged up until the 1970s. Experiencing typical wet and dry seasons, the LKWS can receive more than 500mm of rainfall during the rainy season, dropping to lows of around 150mm during the dry seasons \cite{Walsh2009}, and up to 100\% humidity all year round. 

Danau Girang is uniquely situated in a rainforest corridor that joins two areas of rainforest together, with the corridor surrounded by palm oil fields on each side. Because of this, animals use the corridor to move between the rainforest regions and some use it to enter the palm oil plantation for new feeding grounds. This gives Danau Girang insight into the movement patterns of these animals in the corridor as well as in the rainforest itself, with a wide variety of species that are not commonly seen in other tropical regions of the world. Due to the remote nature of the centre, power is provided by a set of diesel generators which, typically, provide power from 10 am to 1 pm and 5pm to 11pm daily. Wireless Internet access is provided by satellite with speeds comparable to that of 56k, although the upload speeds are considerably faster than downloads.

As outlined in Section \ref{int:mot}, the corridor monitoring programme is a scheme that has been in place for more than five years to use wildlife cameras to track the movement of animals through the corridor and to capture species that are rare or unique to South-East Asia, such as the Bornean Clouded Leopard or Sun Bear. Currently, the use of Reconyx Hyperfire HC500 cameras are being used \cite{Reconyx}. These are standalone cameras that store a pre-defined set of images to an SD card on each trigger, which is triggered by an infrared (IR) motion sensor when the beam is broken. 

Images must be collected manually every two weeks from the cameras and the batteries are changed at that point as well, although a typical charge should last three months. The cameras are equipped with watertight casing but, due to the humidity and the opening of the cameras every 2 weeks, silica gel is used to prevent moisture inside the camera. We believe that humidity also reduces the battery life, as the charge drops from three months to around three weeks within the first few months of usage. However, the lack of constant power availability could also negatively impact the charge cycles of the batteries when at the field centre, reducing their capacity. Each unit is secured to a tree and more dangerous sites, such as known elephant paths, have protective cases as well. 

In 2010, twenty cameras were deployed for six month periods and then relocated based on the needs of the projects at that time. As of 2013, there are now ninety 	with a view to expand and dozens of projects within the field centre use the images gathered from these cameras. Initially, it was the job of visiting research students to collect the images but, since the number of deployed cameras has grown, full-time staff have been taken on to maintain them.

Our belief is that we can use the LKWS, and the locations of the existing cameras, to deploy a WSN that uses local knowledge gained from the researchers at DG to automate the collection of images, improve the battery life by not exposing the internals of the camera to the elements so often and, most important, prioritise the flow of data through the network by in-network processing. 
	
Annual visits, lasting three weeks, have been made to DG to test out hardware, software and wireless choices, in an effort to optimise the network. These visits have also been used to extract local knowledge from the area and researchers, by semi-structured interviews and watching them work.

\section{Hardware}
	Before any visits were made to DG, meetings with staff members of the field centre were held in order to gain a better understanding of the environment and the project. This is where we were alerted to the humidity of the region and the fact that the failure rate of the Reconyx cameras has been as high as thirty per cent.

Reconyx cameras have no external interface support and the only way to access the images is through the removable SD card, because of this there is no way of attaching external sensor hardware to the existing cameras. In-network processing is an important requirement for our WSN and this did limit our choices to nodes that are computationally capable than more common sensors, such as the IMote 2 \cite{Nachman2008}.

In this section, we detail our research into suitable sensor hardware that met the following requirements:
\begin{enumerate}
	\item Able to perform processing of images and metadata
	\item Common interface availability (Serial, USB)
	\item Wireless enabled
	\item Battery-powered
	\item Expandable memory
\end{enumerate}
This section also details the modifications we made to the devices in order to ensure they would survive in a humid environment.

\subsection{Pandaboard}

Texas Instruments supported the development of a reference Single Board Computer (SBC) that had specs similar to that of a modern smartphone and was capable of running desktop Linux, known as the Pandaboard \cite{instruments2012pandaboard}. A dual core 1GhZ ARM processor with 1GB of RAM, support for external storage, expansion ports, USB, Wi-Fi and Bluetooth in a board the size of two credit cards can be a powerful addition to a data heavy WSN, especially one that deals with images.

There is no mention of Pandaboards in the literature being used in WSNs but, the low power of the system and advanced capabilities, make it suitable for processing and transmitting data simultaneously. 
	
\subsection{IGEP v2}

The IGEP v2 is another ARM based SBC that uses a 1GhZ single core processor with 512MB RAM and similar connectivity features to the Pandaboard, but around half the size. This does result in a reduced power draw and the device is still capable of running desktop Linux.

Due to the smaller size, and easier commercial availability, the IGEP has been used as a sensor node to record, process and send readings from multiple devices, such as air temperature, GPS and oxygen saturation as part of environmental monitoring \cite{Resch}. In their research, they found that the IGEP achieved 9.1 hours of uptime using a 4000mAH battery, a capacity used in many modern smartphones.

\subsection{Waspmote}
The Waspmote is a general purpose sensing board that is designed to allow plug-and-play connectivity for multiple sensor modules. The node can be programmed with C++, using the prepackaged SDK. The processing power is not comparable to the more powerful SBCs, but the 600mAh battery is reported to last three months and the size is much smaller \cite{Lib}.

One of the benefits of the Waspmote is that they are commercially available with an actively maintained programming environment.

\section{Wireless Range Experiements}
	In this section, we explain the experiments we carried out to test the performance of different transmission media in the UK as well as the Malaysian rainforest. While range is the most important feature, a data rate that can handle hundreds of large readings in a day is a requirement. Our motivating scenario is focussed on the transmission of sets of 3 images for every trigger; where a sensor can trigger hundreds of times in a day.

\subsection{Wi-Fi}\label{tech:wifirange}
Wi-Fi was already available on our initial test platforms and the high data rate made it suitable for sending a large volume of images in a short period. We knew that current cameras deployed in DG were up to 1km apart and we did not expect to cover that range completely, but were did anticipate that coverage with intermediate nodes.

Research, outlined in Section \ref{bg:trans}, showed us that the rainforest could reduce the range by up to 78\% and the ideal maximum range of 2.4GhZ Wi-Fi is 100m \cite{Dhawan2007}. 

We tested Wi-Fi range using two IGEP boards, powered by 4 D Cell batteries and running a lightweight Linux operating system.	The IGEP nodes we used did not have any additional hardware and the nodes were tested without the use of an external antenna. A Java application was written to periodically scan for available networks and store those results in a text file. One IGEP board was set as the base station and attached to a tree, at the same height it would be if it was attached to a camera, and another was walked to specified points around the base station at defined locations. These locations were chosen to include as many distances as possible and as many different forms of obstacle between the searching node and the base station, such as: line of sight (LOS), medium vegetation or thick trees.
			
This experiment was run in a wooded area in the UK and in the rainforest at DG. The specified maximum range of 802.11g is 120m. When considering attenuation and obstacles we were expecting the signal to be reduced by up to 50\% in the UK. However, we found that we received a maximum range of 30m, with LOS. Figure \ref{cardiffsnr} shows the results we experienced, while testing in Cardiff, some of the drops in signal can be attributed to dense foliage and readings that were not LOS, but a maximum range of 31m, with an SNR of 29.5 dBm, is less than we expected, as the UK does not experience high humidity often.
			
			\begin{figure}[!t]
			\centering
			\includegraphics[width=0.45\textwidth]{Chap3/figures/bp_snr.png}
			\caption{Signal-to-Noise Ratio for Wi-Fi in UK Woodland}
			\label{cardiffsnr}
			\end{figure}
			
	The graph does show a drop at 22m, this was due to the dense foliage that restricted the LOS between the base station and the receiving node, with five runs of this test we observed the same results. The primary aim of this experiment was to prove the viability of Wi-Fi and to to ensure our application functioned as intended, which it did. Further experiments could have been run to remove the anomaly but the results of the experiments in Danau Girang were the focus.
		 
	Despite the poor range from the tests in the UK, it was consistent with other studies reporting signal degradation of up to 78\% in areas with moderate foliage. We visited Danau Girang in 2011 to gather the requirements of the network and ensure the hardware is able to survive the humidity. Range experiments were run in the rainforest to see if a more humid environment impacts range any further, Figure \ref{malaysiasnr} shows this.
			
			\begin{figure}[!t]
			\centering
			\includegraphics[width=0.45\textwidth]{Chap3/figures/dg_snr.png}
			\caption{Signal-to-Noise Ratio for Wi-Fi in Malaysian Rainforest}
			\label{malaysiasnr}
			\end{figure}
						
			Comparing figures \ref{cardiffsnr} and \ref{malaysiasnr} shows that the maximum distance to receive a signal is approximately the same in Malaysia as it is in the UK.  There are more signal drops but this seems to be due to denser foliage blocking the line of sight. However, it does suggest that the humid environment of the rainforest does not have a significant impact on the received signal. It is clear that the denser rainforest does impact the signal-to-noise ratio in a much shorter distance from the base station but a link is still made, allowing for a successful transmission of data.
			
Due to the poor results of these experiments we researched alternative methods to increase the range without impacting the environment the network is to be deployed in. We considered using intermediate nodes, not attached to cameras, to account for the lack of range but, because some cameras can be up to 1km apart, we would need more than 30 nodes to create a connection between two locations.
			
We also researched wireless technologies that are more common in sensor networks. This does mean that the data rate is not as high as Wi-Fi and error correction in packet streams is not always as robust, but it is more suited to sensor networks, using less power and providing longer range.
			
Finally, we considered using the researchers or animals at Danau Girang, as `data mules', creating temporary links between nodes while they are in the forest. However, the trip to Danau Girang yielded the information that researchers generally do not cover those distances in the forest and data delivery would be sporadic.
			
Although the range of Wi-Fi is poor, for our requirements, in both Malaysia and the UK, it did show that the results we experience in the UK are very similar to the results in Malaysia. This means that tests run in the UK should be indicative of what we can expect in Danau Girang.

\subsection{Digimesh}\label{tech:digimesh}
		Due to the poor range results of Wi-Fi, we created a second prototype of the network, using Digimesh. Digimesh is a proprietary wireless protocol, based on the 802.15.4 standard and designed for devices with limited power. Using the same frequency as Wi-Fi, Digimesh has been reported to provide 7km of range, with a data rate of 250kbps.
		
In our prototype implementation, we are using Waspmote sensor boards \cite{Waspmote}, a general purpose node that is capable of transmitting through various communication mediums. Our Waspmotes are provided with Digimesh modules and a 2GB SD card to store sensed data.
		
When testing the range of the Waspmotes, we followed a similar method to that which is outlined in Section \ref{sec:wifirange}. One board is in a fixed location and running a C++ application to poll for nodes in the network, once a node is found it sends a message to the node every 10 seconds. The second board is set to scan the network and receive packets as soon as a base node is found, this node is then moved to different locations.
			
The receiving node prints out variables related to the received packet, such as: RSSI, source MAC address and packet ID. However, not all packets are received so the RSSI can display 0 if there are errors reading or if packet collision occurs. We found this to affect the results and have just used the two nodes to identify the maximum distance they can be apart, while maintaining a stable connection.
					
Initial experiments were run in a moderately vegetated area in the UK which yielded 497m of range. Limitations with buildings preventing us from testing any further but the signal strength still proved to be strong.
			
The initial results for the range tests proved positive and Digimesh does seem to be a viable solution to account for the lack of range when using Wi-Fi. As the frequency is the same as 802.11g, thus licensing it for worldwide use, we expected similar results in Danau Girang..
						
Experiments were run in 2 areas of the rainforest around Danau Girang and the results yielded were not the same as we experienced in the UK, and thick vegetation proved to have a significant impact on the range, reducing it by almost 50\%.
			
In more open areas of the rainforest, we achieved 199m on average, more dense regions of the forest reduced this to 102m on average. While these results are not as high as we achieved in the UK, they are still suitable to use Digimesh in the deployment of a WSN. This could be because we were using low-gain antennas and little configuration had been made on the Digimesh radio.

\subsection{Adapting for Harsh Environments}
	All of the hardware that we used for experiments had their components exposed and would have become compromised if moisture came in contact with them. To protect them from this, we used waterproof cases with a protective foam inside, known as Pelican cases, to keep the nodes watertight, but still allowing airflow, shown in Figure \ref{tech:pelican}.
	External antenna can be fed through the lip of the case, using thin cable, ensuring that the range of the transmissions is not affected by the case. 

			\begin{figure}[!t]
			\centering
			\includegraphics[width=0.45\textwidth]{Chap3/figures/pelican}
			\caption{Pelican Waterproof Case}
			\label{tech:pelican}
			\end{figure}

%\section{Software}
%	\subsection{Triton}
%		Triton is an image-processing application that uses 
%	
%	\subsection{ODACH}

\section{Conclusion}
	In this chapter, we have shown the current hardware choices available for more computationally capable sensors and detailed our experimental results on how rainforest environments impact wireless transmissions. Although technologies, such as Wi-Fi, provide a high data rate, their range is limited and not suitable for sparse sensor networks; especially in a humid environment.
	Newer technologies, designed for long-range communication in sensor networks, are becoming increasingly more viable and, while they do have lower data rates and less robust protocols, they are more suitable for a resource constrained WSN that requires minimal power draw when transmitting. 
	Using general-purpose hardware also means that there is no protection against water, humidity and animal or human intervention. We have adapted existing waterproof cases to suit our needs and two week deployments have shown that they are able to prevent any moisture from entering the case.
	 
\chapter{Wireless Sensor Network Architecture Design}\label{chap:arch}
	In this chapter, we explain our proposed network architecture that uses local and global knowledge to make informed routing decisions and to classify sensed data within the network. Our approach, K-HAS, uses a three-tiered, hierarchical approach with each subsequent tier providing increased knowledge processing capabilities. 

	We aim to show that sensors capable of processing knowledge will provide a more efficient network and be able to prioritise the delivery of interesting sensed data. We also believe that human input is a valuable way for such a network to learn about classifications it has made correctly, and incorrectly, and use that knowledge to inform future classifications. To prove this, we have developed an architecture that uses different levels of knowledge processing throughout the network, the Knowledge-based Hierarchical Architecture for Sensing (K-HAS).

	The rest of this chapter is structured as follows. Section \ref{arch:khas} outlines the main aims of K-HAS and what it is capable of that typical sensor networks are not. Section \ref{arch:tech} details the software and data standard choices we made for our implementation of K-HAS. Section \ref{arch:walk} provides a sample walk-through of K-HAS, relating to our motivating scenario and Section \ref{arch:conc} concludes the chapter.

	\begin{figure}[h]
			\centering
			\includegraphics[width=0.65\textwidth]{Chap4/figures/khasarch} 
			\caption{Overview of the K-HAS Architecture}
			\label{arch:khas:arch}
			\end{figure}


\section{The Local Knowledge Problem}
While large in size, it is typical that a WSN would use low cost nodes with limited power, memory and computational capabilities; causing them to lack the ability to be aware of their surroundings or the data that they are sensing \cite{Akyildiz2002}. This means that, unless fixed by a routing protocol or a technician, data is delivered on a chronological basis and is then filtered at the base station, usually manually. Some WSNs store all of the data on the node and users of the network use a `pull' model to query for data from nodes \cite{Sadagopan}, but this requires some technical knowledge and, although it does increase the battery life of the nodes, it is a manual process again.

The environment that a WSN is deployed in is often a rich source of data to be sensed (such as inside a bird's nest to sense temperature, humidity and movement), which often contains patterns that can be used to improve the performance of the network. For example, if a node knows that it is has only been triggering between the hours of 6pm and 5am for the past few weeks, it can then learn to enter a deep sleep outside of those hours or use that time to transmit data it has been storing while it assumes it will be inactive, based on previous days. Alternatively, this knowledge can be used to prioritise data throughout the network so that the most important data is received first, instead of the most recent. An example of this could be two camera nodes deployed facing the entry and exit of a building, tasked with looking for intruders between 5pm and 8am. If the camera facing the exit is triggered at 5:01pm and the camera on the entrance is triggered at 5:05pm, then the knowledge that the security guard leaves through the exit between 5:00pm and 5:08 pm will allow the entrance camera to prioritise its capture as more important, as it is an irregular occurrence.

This knowledge can be categorised as either \textit{local} or \textit{global}. Local knowledge \nomenclature{LK}{Local Knowledge} is the knowledge of an area that has been gained through experience or experimentation \cite{Joshi2001} For example, a native to the Amazon may know that three of the locations in which the nodes are to be deployed are flooded for two weeks of the year, rendering their readings useless for that time period and increasing their risk of failure. This is local knolwedge, as it cannot be gained without experiencing the flooding in that area, or measuring local water levels.

While we believe that we are the first to use the concept of local and global knowledge within the wireless sensor network domain, the terms have been around for many years. In 1999, a book that referred to local knowledge as \textit{indigenous knowledge} defined local knowledge as `systematic information that remains in the informal sector, usually unwritten and preserved in oral traditions rather than text' \cite{LadislausM.Semali}. 

Over the past twenty years, local knowledge has been used in various contexts, from researching lending and the credit market \cite{Stiglitz1990} to extracting local knowledge from natives to improve farming techniques \cite{DEWALT}. This research, as well as work that will be covered later, showed us that there are two kinds of knowledge: global and local. 

It was from agriculture research that we were able to refine our definition of local knowledge, \cite{Joshi2001} defines local knowledge as `knowledge that farmers have derived locally through experience and experimentation'. They also say that indigenous knowledge is different in that it is culturally specific. From this definition, as well as our work with our motivating scenario, we were able to generalise the definition and expand upon it.

We now define local knowledge as \textit{knowledge of an area, held by a domain expert, that has been gained through experience or experimentation}. The weather of a region is global knowledge because it can be found through a variety of media, whereas the saturation levels of the soil in a field would be local knowledge as it would require experimentation to gain that knowledge. 
%Local knowledge could become global if it was readily available but some of these data are held in tribes of people without Internet access.

Using this definition, we believe that \textbf{utilising local and global knowledge within a sensor network can inform routing decisions to make better use of the bandwidth in resource-constrained WSNs by sending data that the node believes to be important first, rather than chronologically}. Patterns in the data, and knowledge of the environment surrounding a node, can allow a node to infer what data may be valuable, automate the processing of adding context to the sensed data, learn from previously sensed data and utilise global knowledge of ongoing projects within the network to determine what data is thought to be of a higher priority.

To show this, we have developed a network architecture for WSNs that utilises knowledge from the data it senses, as well as its deployed environment. It is called the Knowledge-based Hierarchical Architecture for Sensing (K-HAS) \nomenclature{K-HAS}{Knowledge-based Hierarchical Architecture for Sensing} and this thesis will show how K-HAS addresses the problem of delivering the most important data first and improving the overall efficiency of the network.


% We believe that the use of this knowledge can increase the effectiveness of the network, as well as prioritise data based on its value as opposed to when it was recorded. 
	
	\section{K-HAS}\label{arch:khas}
		K-HAS has been designed as an architecture for WSNs that addresses the above problem by using knowledge to classify sensed data and adapt to changes in the structure of the network. By pushing knowledge bases out to the edge of the network, all nodes in the network have some awareness of the data they are sensing, as well as how important it is, based on the current projects that the network is involved in. This is achieved by using rules with different levels of granularity based on the knowledge processing capabilities of that tier. Figure \ref{arch:khas:arch} shows a high level overview of K-HAS, showing the flow of data from each tier.
	
	
	\subsection{Data Collection}\label{khas:dc}
	The data collection (DC) tier is very similar to sensor nodes commonly used in a WSN, using hardware that has similar, limited processing power and storage to nodes such as: the I-Mote and Waspmote. These DC nodes are deployed at the edge of the network and tasked with sensing their environment, pre-processing the sensed data and using each other to relay data to the next tier.

	DC nodes are capable of performing processing on data, such as the time it was recorded and its size, but their limited knowledge processing capabilities allow them to have an increased battery life and reduced size, making them suitable for a variety of deployments as they can be integrated easily with existing devices (i.e. cameras) and be easily disguised, or hidden.

	\subsubsection{Knowledge Base}
	Reduced knowledge processing capabilities and low memory restrict the knowledge that these nodes can hold and they are limited to a static knowledge base that is encoded at the time of 	deployment. DC nodes run an operating system designed for embedded devices, the size of the operating system is minimal when compared to an operating system used on modern desktop computers and provides much more limited functionality, such as sleep scheduling, basic file system access and the capabilities to run a variation of popular languages. 

	For example, TinyOS \cite{levis2005} is an operating system designed for sensor nodes and uses a dialect of the C language, called nesC, that is optimised for the memory constraints of most sensor nodes. The operating system exposes components that allow basic commands, such as reading from the sensor interface or sending a message. Commands are executed as a request to perform some action and events signal the completion of that action. Intensive tasks, such as data processing, can be scheduled to run at a later time, ensuring that the low-power node remains responsive and preventing any calls from blocking other events.

	The Waspmote nodes, explained in Section \ref{tech:waspmote}, do not run any operating system and any event handling, networking or memory management is performed solely by the single file application that is uploaded to the node through the boot-loader. Performing intensive tasks, such as local data processing, can drain the battery of the node, cause it to fill the available memory or even cause the node to go down. DC nodes do not have the safeguards in place that most operating systems and applications do, operational code must be lightweight, responsive and ensure that events do not run for longer than intended. 

	Some nodes within this class do not have the capability to process text files within the file system and, therefore, must have the knowledge added to their operational code. DC nodes only perform simple operations on the properties and content of the data that they sense, such as the time it was recorded, the location and its size. For more complex data, such as images and video, DC nodes do not possess the computational power required to process them and instead use the metadata associated.
	Unlike modern rule engines, these static rules do not use forward chaining and the outcome of one rule does not cause the rules to be fired again. Listing \ref{kb:dcrule} shows an example of some of the rules in the knowledge base and highlights that the core structure of this file is simply a list of if statements that can be executed one after the other. Forward chaining is not used here because processing times need to be kept to a minimum and a chain of rules being fired could easily overwhelm the node. If any rules are fired that suggest the observation may be interesting, it is prioritised through the network and potential classifications are encoded within the observation before it is sent on.

\begin{lstlisting}[breaklines=true, caption={Pseudocode DC Node Rules}, label={kb:dcrule}]
if month of observation is JUNE AND time of observation is between 17:00 and 19:00 and active otter project is TRUE
	add classification to observation(`Potential Otter sighting')
	prioritise observation as interesting
if temperature is 37 AND time of observation is between 01:00 and 05:00 and active leopard project is TRUE
	add classification to observation(`Potential Leopard sighting')
	prioritise observation as interesting
\end{lstlisting}

When the data is recorded by the DC node, the knowledge base is fired and inferences are made about the contents of the data. Each DC node has a static knowledge base loaded onto it before it is deployed, which is based on the local knowledge of the area that it is to be deployed in.  For example, a node deployed on the bank of a river would have a different knowledge base to a node deployed in the fields of a plantation.

Once a trigger has been processed, the data is packaged and then sent on to the Data Processing (DP) node.

	\subsection{Data Processing}\label{khas:dp}
	DP nodes act as cluster heads of the network, serving a subset of all deployed DC nodes. When data is sensed, it is forwarded through all DC nodes to the DP node that is tasked with serving the originating DC node. These nodes have more knowledge-processing capabilities than a DC node and do not perform any direct sensing. 

	Due to the greater capabilities, DP nodes have a much shorter battery life than DC nodes and a network typically consists of fewer DP nodes. These capabilities, such as increased memory and higher processing power, allow DP nodes to run a rule management system (such as Drools \cite{proctor2005drools}), that would not be possible on nodes that run on an OS for embedded systems, that is able to handle more complex rules and process more than just files, they can perform the same tasks as most modern computers, such as: image processing, audio processing or reading metadata from files that requires extra libraries. When a DP node receives data, it processes everything associated, this includes metadata, the data itself and the inferences made by the DC node. If the DC node has inferred that the data is of a higher priority, then this data is processed first. This is done by prioritising the data at two stages: once it has been received and when it is about to be sent.

	In our current design, DP nodes use two different radios, a Zigbee radio to allow long range communication from DC nodes and a Wi-Fi radio that provides short range communication that allows for higher data rates.

	\subsubsection{Knowledge Base}
	In our motivating scenario the network is image-based, this means that the DP node would perform image processing, as well as processing the image metadata. The increased knowledge processing capabilities allow DP nodes to run rules dynamically, learning from the sensed data and providing classifications that change based on changes in the environment. For example, if a DP node has not seen an elephant before, and it is not aware of the object in the image, then it will await a human classification. The node will then record the time period that it receives elephant pictures, i.e. June to July, and become more alert the following year. Similarly, the node will know not to look for pictures of nocturnal animals during the day. This local knowledge allows processing power to be saved and, thus, time; this ensures that the processing of sensed data is optimised as much as possible in order to reduce the time it spends in the network.
	
	% The rule engine used in our current implementation is Drools, a Java based rule engine that allows for rules to be defined in \textit{.drl} files and these can be loaded dynamically into a knowledge base. This flexibility allows to be changed on the fly without the need to restart the device, or even require human access, as all of this can be achieved through network communication. 
	The rule engine used should allow for rules to be dynamically inserted into the rule base, so that rules can be updated through network communication.

	Upon receiving sensed data from a DC node, the rule base is fired on the meta data of each file received. If the rules determine that the data is of interest or, in the best case scenario, provides a classification, then the data is packaged and sent on to the Data Aggregation (DA) node.
	
	\subsection{Data Aggregation}
	Placed at the root of the network, these are nodes with the same knowledge-processing capabilities as DP nodes (although they typically have greater memory, processing power and a continuous power source) and would be accessible by users of the network. When DA nodes receive sensed data, it is unpacked and stored with a link to the node that the data originated from. Compared with a standard WSN, these nodes can be compared with a base station, or endpoint.
	
	Any information added by the DP node is parsed and classifications are extracted. If a classification is found, it is stored and the DA node checks for any active projects that contain the classification. If a match is found then all users involved in the project are informed via their preferred method of communication. Using the motivating scenario as an example, the people involved with projects could be researchers and professors and they may be looking for images of leopards, requesting to be informed via Twitter.
	
	All sensed data received, regardless of whether it has been classified, is accessible through a web interface hosted by each DA node. The interface shows all of the sensed data from each deployment, along with the associated classification. More importantly, it allows users to classify the data using a voting system. Users have roles which give them different privileges within the system. Normal users are able to vote and the majority vote is seen to be the current classification. While this is not a necessary feature of a DA node, it does allow the network to utilise the knowledge of human experts to inform future classifications.
	
	However, privileged users are able to confirm a classification and prevent any further votes. Once a classification has been confirmed, it is then sent back to the DP node it originated from. If the classification made by a user is different to the one inferred by the node, then it updates its knowledge base and acknowledges receipt.

	This section of K-HAS is yet to be implemented but we believe that this system will be vital in the early stages of deployment as it will be used to build up a knowledge base. The more user classifications there are, the more accurate the network will be in the future. After a few months of classifications, K-HAS would then be able to use the knowledge base to make more informed classifications, requiring fewer classifications/confirmations from users. The exception to this is sensed data that cannot be matched to the knowledge base. 
	
	Data aggregation nodes should not be hampered by limited battery life that deployed nodes would experience as we expect them to be placed in a base station with power availability and access to the Internet. Therefore, DA nodes would typically be desktop computers with a constant power supply.
	
	\subsubsection{Knowledge Base}
	DA nodes do not typically experience the resource constraints that DC and DP nodes must compensate for. Because of this, they hold a global knowledge that contains a history of all observations made by all nodes, as well as the location and deployment times of all nodes in the network. When nodes are deployed, this is updated, by users, to show the location of the node and when it was placed. 

	Every classification sent from DP nodes is stored in a database and contains all information about an observation: date, time, originating node, route taken within the network, location, sensed data, classification of sensed data. They also hold information for all projects running, for example: a project to track elephant movements within the forest. When sensed data that could be related to the project is received, such as an image with an elephant classification, users associated with that project are informed through email, or other means.
	
	While DC do not store any of the observations they capture, and DP nodes only store part of the observation that can be used in future classifications, DA nodes store the complete observation made by every DC node, as well as any extra knowledge that is added by users upon receiving the sensed data. What is stored by the the DP node is dependent on the deployed purpose of the network. For example, in our motivating scenario, we store: the processed image, the classification, the node it originated from and the date and time it was captured.
	
	As well as this, the functionality of DA nodes can be extended to provide administrative operations on the network, such as the recording of node locations, time of deployment and viewing all active nodes. This allows the DA node to monitor active nodes and alert users if a node has not sent any data in a while. The longer a K-HAS network runs, the more knowledge a DA node gains, both from users that classify observations and from changes in the sensed data. This knowledge is then relayed back to DP nodes, updating their knowledge bases as to what classifications were correct and which need updating for future observations.

	\subsubsection{Feedback Loop}
	The feedback loop is a protocol within K-HAS, that uses human input, and other sources, to update DP nodes. When a DP node classifies sensed data, it stores some of that to assist with future classifications. What is stored depends on the type of sensed data. For example, images would mean that the DP node would store the resulting processed image, its classification and information about the time it was taken, the camera that took it and the location. When that is sent to the DA node, a human would then look at the image and mark the classification as correct, or modify it if it was not. Once that classification has been finalised, the DA node sends either a confirmation or a modification to the DP node that sent it, updating its knowledge base. This protocol seeks to reduce the number of incorrect classifications the longer the network is deployed and allows the nodes to be dynamic and `learn' throughout the lifetime of the network. It also allows nodes to adapt quickly to new data, if a DP node is unable to classify an image of an animal it does not yet have template images for, the knowledge of a human expert can provide those templates and the feedback loop will deliver that knowledge.
	
	In the current design of K-HAS, the feedback loop has only been tested in a development environment and has not been fully implemented, we believe that the actual implementation of the feedback loop will allow the network to adapt dynamically to changes during its deployment.
	
	
	\section{Technological Components}\label{arch:tech}
	In this section, we describe the technologies used in our designs of K-HAS and how they integrate in order to use local knowledge based on their respective knowledge processing capabilities. The majority of components, both hardware and software, used in K-HAS are used so they are applicable for any WSN, but some choices have been made to remain in line with our motivating scenario and, thus, are more specifically suited for the capture of scientific observations.
	
	\subsection{Data Standard}\label{arch:tech:dwc}
		To pass sensed data through the network, we first had to choose a standard format that would allow us to encode the sensed data, as well as enrich it with inferences made through processing. Darwin Core (DwC) is a body of standards with predefined terms that allows for the sharing of biodiversity occurrence information through the means of XML and CSV data files \cite{wieczorek2012}.

The Global Biodiversity Information Facility (GBIF)\cite{gbif} indexes more than 500 million Darwin Core records published by organisations all over the web, allowing datasets that were previously siloed from the public to be accessed by both human and machine. The primary purpose of Darwin Core is to create a common language for sharing biodiversity data that is complementary to and reuses metadata standards from other domains wherever possible \cite{wieczorek2012}.

DwC Archives follow a star file structure, where a record can contain many occurrences, which is the recording of a species in nature or in a dataset. In an occurrence, there is an \textit{event}, a recording of a species in space and time, enriched with other terms such as \textit{identification} and \textit{location}. DwC is the standard set of terms that can be used, while a Darwin Core Archive (DwC-A) provides the structure for data recorded using these terms. The core files in a DwC-A are:
\begin{enumerate}
	\item EML.xml
	\item Meta.xml
	\item Data files
\end{enumerate}
While DwC does not have the extensions available to OBOE, an extensible base ontology designed for ecological observations that is explained in detail in section \ref{OBOE}, it is extremely concise for recording observations within the biological diversity domain and aims to be a standard reference for sharing these observations.

% The standard structure of a Darwin Core Archive (DwC-A) is a star record; an archive of files with a core metadata file that describes the content of all other files within the archive.

The record shown in figure \ref{dwca} represents a DwC-A that conforms to the star schema. The ecological metadata language (EML) document contains all of the details about the project, such as who is involved, the institution code, contact details and the project(s) related to the observation. Listing \ref{dwc:eml} shows a fragment of the EML file and a complete DwC archive can be found in Appendix \ref{appendix:dwc}.
\vspace{\baselineskip}
% \lstinputlisting[language=XML, caption=Darwin Core Ecological Metadata File, label=dwc:eml]{Chap5/figures/dwc_arch/eml.xml}
\begin{lstlisting}[caption=Darwin Core Ecological Metadata File Fragment, label=dwc:eml, breaklines=true, language=XML]
<alternateIdentifier>e71fda1c-dcb9-4eae-81a9-183114978e44</alternateIdentifier>
<title>Images from Danau Girang during the PTY Project 2011-12</title>
<creator>
	<individualName>
		<givenName>Christopher</givenName>
		<surName>Gwilliams</surName>
	</individualName>
	<organizationName>Cardiff University</organizationName>
	<positionName>PhD</positionName>
	<address>
		<city>Cardiff</city>
		<administrativeArea>Cardiff</administrativeArea>
		<postalCode>CF24 3AA</postalCode>
		<country>Wales</country>
	</address>
	<phone>(+44)2920 123456</phone>
	<electronicMailAddress>C.Gwilliams@cs.cf.ac.uk</electronicMailAddress>
<onlineUrl>christopher-gwilliams.com</onlineUrl>
</creator>
<pubDate>2012-07-26</pubDate>
 \end{lstlisting}

The descriptor file is an XML document that contains the column headers in the attached files and the mappings of those headers to DwC terms, an example file is shown in Listing \ref{dwc:meta}. This file shows that this is an archive containing the sighting of an individual, and that the sighting has been split into two files. The largest benefit of Darwin Core is its modularity. Extension files can be added to enrich the data for each occurrence. In this example the extension file is named as \textit{images.csv} and contains image-based evidence to support the observation.

\noindent\begin{minipage}{\textwidth}
\lstinputlisting[language=XML, caption=Darwin Core Descriptor File, label=dwc:meta]{Chap5/figures/dwc_arch/meta.xml}
\end{minipage}

    \begin{figure}
    \centering
      \includegraphics[width=\textwidth]{Chap5/figures/dwca.png}
    \caption{A Darwin Core Star Archive \cite{wieczorek2012}}
    \label{dwca}
    \end{figure}

Listing \ref{dwc:set} shows the comma-separated central file (Basis of Resource) containing the core details of the observation, i.e. the animal observed, and the column headings map to the descriptor file. Other linked files are typically linked by the unique ID of the observation, containing information that extends the observation and provides further context.

\noindent\begin{minipage}{\textwidth}
\lstinputlisting[caption=Darwin Core Occurrence Data, label=dwc:set]{Chap5/figures/dwc_arch/set.csv}
\end{minipage}


% Ecological Metadata Language (EML) is a metadata used by ecologists and the language is used to describe projects and those involved. This file acts as a form of certificate and descriptor as to what the data is related to and who owns it. The XML file, shown in Listing \ref{arch:dwc:eml}, outlines a sample project and users involved in the project.

% \lstinputlisting[language=XML, firstline=0, lastline=21,breaklines=true,label=arch:dwc:eml,caption=Darwin Core: EML.xml]{Chap4/listings/dwc_arch/eml.xml}

% The core file, \textit{meta.xml}, shown in Listing \ref{arch:dwc:meta} lists the files that contains the actual sensed data, as well as the terms used to describe it. Examples include: date, time, location, type of data, filename and species contained.

% \lstinputlisting[language=XML, breaklines=true,label=arch:dwc:meta,caption=Darwin Core: meta.xml]{Chap4/listings/dwc_arch/meta.xml}

% Data files contain the actual sensed data, based on how it is supported, and these files are linked in \textit{meta.xml}. For example, temperature readings or direct human sightings would be stored in a CSV file and linked, however, images or video would require the metadata to be store in a CSV file and a filepath would be referenced in the XML. The structure of the CSV file contains a header line that matches the terms in the meta file and each line would be an observation. The terms are linked to the Darwin Core glossary so the archive can be validated and processed by a DwC archive reader.

All of these files are then archived and sent as a ZIP folder throughout the network. If the sensed data is media based, then the media is included as well. Sofwtare libraries to process DwC archives are included on both DP and DA nodes.

Darwin Core is suited to K-HAS because its use in ecological observations matches our motivating scenario and the archive can be easily created by a DC node, as it does not require any heavy processing and all of the files are commonly used formats.
	
	\subsection{Middleware}
	The knowledge-processing capabilities of DA and DP nodes are the same and this is part of what makes K-HAS different from most other WSNs; both types of node run the sensor middleware, but each for different purposes. DA nodes use the middleware for administrating the network, receiving and archiving sensed data and allowing users to provide classifications. DP nodes use it for the receiving, sending and controlling the flow of processing of sensed data before it is passed on.
	
	Existing suitable middlewares have been detailed in Section \ref{sec:middleware} and our requirements for K-HAS were partially determined by the expertise of the users in our motivating scenario. Below is a list of our three core requirements:
	\begin{description}
		\item[Portability] Heterogeneous WSNs utilise nodes with different architectures and capabilities, if middleware is to be used on the nodes it must be able to run on these varied devices. 
		\item[Usability] Users of K-HAS should not be expected to have knowledge of computer science or the underlying architecture, this network should be usable by almost anyone. The same must be said for the middleware as well.
		\item[Extensibility] A closed-source middleware can be used, but it must then support all sensor nodes and data types, as well as receive regular updates. Open-source, or extensible, middleware can be used to add support for newer nodes.
	\end{description}
	
	GSN is a Java-based open-source middleware. New generic sensors can be added through XML files, while more complex sensors can be added through custom Java classes. GSN is covered in more detail in Section \ref{sec:GSN}. Because GSN can run on any architecture that supports the Java Virtual Machine (JVM), it meets our portability requirements and the web interface to provide administrative functionality makes it usable by those without any domain knowledge. Finally, the ability to add new sensors through XML means that it can be extended by almost any user of the network with very little guidance.
		
	\subsection{Knowledge Capture}\label{arch:kc}
		GSN is packaged with a web interface that allows users to see all nodes deployed and view the latest sensed data received. The web interface is targeted towards users with domain expertise and has limited functionality focussed towards sensor administration. However, it does make GSN accessible by more than one computer, as well as a variety of different architectures. For example, the admin webpage could be accessed on the machine that runs GSN, or from a tablet computer connected to the same network. We used the same approach to develop a web-based tool that provides access to all sensed data, as well as a simple interface for performing tasks, such as uploading new rules or updating the location of nodes.
		
		All sensed data is read from a MySQL database and users can view the metadata from each observation, such as location, date, time and temperature, as well as the data itself. From this, users are able to classify the data based on their role. Shown in the ontology in Chapter \ref{chap:ont}, K-HAS uses roles to control active projects and classifications; there are administrators and researchers. Researchers are involved in projects and receive notifications when relevant data has been received. They have access to the web interface and can vote on classifications for sensed data. Administrators lead projects and can create/complete them, but they also have the ability to finalise classifications. K-HAS follows a knowledge hierarchy (Figure \ref{arch:kno:hier}), with administrators at the top and DC nodes at the bottom. While there are more DC nodes in the network, their knowledge bases are more limited and classifications are trusted less than classifications made by DP nodes. Although there are fewer DP nodes, their knowledge bases are more detailed. Researchers and admins do not share a level on the knowledge hierarchy because we assume that administrators would be more experienced domain experts. For example, researchers could be students at Danau Girang, whereas an admin would be a professor with more experience. When an administrator makes a classification, all prior classifications are ignored and the feedback loop protocol is used to update nodes.

		\begin{figure}[!h]
			\centering
			\begin{tikzpicture}

			\def \h {5};
			\def \f {.7};

			\foreach \y in  {0,1,2,3} {
			    \def \w { \h*\f-\y*\f };
			    \def \v { \y*\f-\h*\f };
			    \draw (\v,\y) -- (\w,\y);
			}

			\draw (-\h*\f,0)  -- (0,\h);
			\draw (\h*\f,0)  -- (0,\h);
			\node at (0,0) [above] {DC Nodes};
			\node at (0,1) [above] {DP and DA Nodes};
			\node at (0,2) [above] {Researchers};
			\node at (0,3) [above] {Admins};
			\end{tikzpicture}
			\caption{Knowledge Hierarchy for K-HAS}
			\label{arch:kno:hier}
		\end{figure}
		
		Figure \ref{kc:loris} shows an observation where users can vote on the contents. An administrator can then confirm that classification and prevent further votes from being cast. This type of moderation means that it does not have to be specialists voting on sensed data that cannot be classified by DP nodes.
		
		\begin{figure}[h]
		\centering
		\includegraphics[width=\textwidth]{Chap4/figures/loris}
		\caption{Web Interface for Observations}
		\label{kc:loris}
		\end{figure}
		
		Viewing data for new observations is useful for gaining classifications and alerting members of a project, but viewing older sensed data allows patterns to be identified in order to create new rules. Figure \ref{kc:loris_data} shows a map of all deployed nodes in the area surrounding the field centre in our motivating scenario. When users select a node, a table is populated with all of the classified observations that it has captured; this can then be used to extract patterns from the data and create rules. For example, the three observations of the Malay civet are only seen late at night, if further observations also showed this, then we could create a rule defining the active hours of the Malay civet and, potentially, list days that it is likely to pass. These rules can then be written and uploaded to the knowledge base.
		\begin{figure}[h]
		\centering
		\includegraphics[width=\textwidth]{Chap4/figures/loris_data}
		\caption{Web Interface for Classified Sensed Data}
		\label{kc:loris_data}
		\end{figure}

	\subsection{Knowledge Base}\label{arch:kb}
	%Discuss drools here and the API developed for access to it from Sitesy.
	The Drools rule engine is a Java-based engine that uses forward chaining inference for the processing of rules \cite{proctor2005drools}, which means that rules are used to make meaningful inferences about data. Unlike DC nodes, which have limited knowledge processing capabilities, Drools is able to chain rules together and a rule that may not have been triggered at the start of processing may be triggered later if another inference is made. For example, a rule that is specifically for small mammals may not be triggered until an inference has been made that the image may contain small mammals based on the time and location of the observation.
	
	Drools is able to dynamically update its knowledge base, adding rules and firing them on observations that have already been loaded, as well as newer ones. This allows DP nodes to adapt to new rules and local knowledge whilst they are deployed. The use of \textit{drl} files use a mixture of Drools and Java syntax to define rules, allowing them to modify, or create, Java objects. For example, a rule could be triggered on the receipt of sensed data and create a DwC object from the received data, process it and perform checks on the result that would trigger different rules based on that result. This is one of the main reasons we chose Drools, as it can work with GSN and DwC Java objects, as well as the ability to run on any architecture that supports Java. In order to add rules to the Drools system, knowledge of the Drools syntax and, ideally, Java is required, which does make it a part of the network architecture that requires some specialist knowledge. 
	
	The functionality of Drools is extensive and the engine is very powerful, however, it does require specialist knowledge to use and manipulate rules. Using a custom developed Drools web interface , detailed in Section \ref{arch:kc}, we created a simplified interface that uses a custom REST API for Drools, allowing users to create sessions, add rules, load data and fire rules, returning the output to the interface. Users can view, and load, existing drl files, shown in Figure \ref{kc:loris_drl}. I
	
		\begin{figure}[h]
		\centering
		\includegraphics[width=\textwidth]{Chap4/figures/rules}
		\caption{Web Interface for Drools Operations}
		\label{kc:loris_drl}
		\end{figure}
	
	Once a file has been selected, users can view the rules and use the controls on the webpage to perform common operations. The \textit{Load Darwin Core} button loads all DwC archives that are stored in the MySQL database into the current Drools session and the \textit{Fire} button runs all loaded rules on the loaded data. If any of the rules trigger, then the output is presented to the user in the same page, allowing them to act on the results. For example, the location of observations could provide a narrowed down list of potential classifications, allowing an administrator to remove votes that do not match the list.

	Users can write new rules based on patterns gleaned from observations received during a K-HAS deployment, which can be found using the web interface and querying the database. These patterns can be encoded as rules into a \textit{drl} file and uploaded to have an immediate effect on the active knowledge base in the network. 
	
	Currently, these implementations have been developed for our motivational scenario, but much of these tools are general enough to be repurposed in order to apply to a variety of different WSNs. The Drools API can be used for any kind of sensed data and the web interface would require minor changes to be extensible.

	\subsection{Routing Protocol}\label{arch:routing}
		The routing protocol we have selected for K-HAS is not fixed for every deployment but, from our research, we modified the commonly used Mininum Cost Forwarding Algorithm (MCFA), outlined in Section \ref{bg:rp:mcfa}, as it allows for changes in the topology of the network and does not require every node to have a global view of the network.

		On deployment of all nodes, a DA node sends out a packet containing the number zero, representing the number of hops to the DA node. Nodes in range receive the packet, store it along with the identifier of the originating node and then send it once it has been incremented by one. The next nodes receive that packet and do the same until the edge of the network is reached. If a node has a number stored that is higher than the one it receives then it is replaced and sent on until the nodes at the edge of the network are reached.

		When a node wants to send data, it queries neighbouring nodes and sends to the node with the lowest hop count to the root. We modified this to run in accordance with our tiered architecture as we expect the topology to remain the same for much of the deployment.

		In MCFA, nodes do not store path information and messages are broadcast to all nodes when they are sensed.

		Firstly, our protocol runs in two modes: configuration and running. During the configuration mode, we use MCFA. Whereas MCFA does not store path information, we use the method described above to store both the hop count and the nearest neighbour. This process is carried out until all nodes of the network have a hop count (and neighbour) and a final pass is made by all nodes to find, and store, their neighbour with the lowest count. The main difference is that, unlike MCFA, not all nodes process all packets. If a packet originates from a DA node, then it is only stored by DP nodes but DP node packets are stored by both DC and DP nodes. DC nodes store it and send all data through their nearest DP node and DP nodes store other DP neighbours to delegate processing to, should a situation arise where they have too much data to process and cause a bottleneck. Configuration mode can then be run at a set interval throughout the deployment of the network, or initiated manually.

		Whilst in running mode, nodes do not query for the neighbour with the lowest hop count, or broadcast the sensed data to all nodes in range, they send to the node stored as their nearest neighbour. If that fails, then a query is sent out to find other available nodes in range and then sent to the one with the lowest hop count. If the nearest neighbour node is unavailable for more then three attempts, then it broadcasts a request to run the configuration mode again.

	
	\section{Walk-through}\label{arch:walk}
		In this section we will explain the steps involved in the capture, and processing, of an observation when using the K-HAS architecture. Each tier is responsible for performing different actions upon the observation to ensure it is received by the DA node with an inference as to what it may contain.
		
		Not all of the features described in this section have been implemented within K-HAS and some features have only been tested on a small set of observations. For example, the image classification down to species level is a concept that we have not implemented but template matching is a commonly used practice within image processing \cite{fast1995}.

		\subsection{Scenario}\label{arch:scen}
			This walk-through will use our motivating scenario and the type of sensed data will be images of animals in the Malaysian rainforest. In this example, we have a collection of wildlife cameras, with nodes attached to them, deployed in the forest. Projects for the rare clouded leopard and sun bear are currently active at Danau Girang. The clouded leopard is a nocturnal carnivore that uses existing paths and hill trails to travel through the rainforest and the sun bear is the smallest bear in the world and sightings are rare. It is also nocturnal and claw marks can be seen on trees that they have climbed. All of this information has been encoded onto the DP nodes and DC nodes know that images taken at night will be of a higher priority, as well as to prioritise further images at night from DC nodes that are deployed on ridges or existing trails.

			In order to explain the K-HAS architecture, we need to show the planned topology for the network in Danau Girang, which is based on the positions of the cameras in 2010. Figure \ref{topol} shows a section of the proposed topology around the field centre, with the rest of the network spreading out along the river on both sides. The triangle icon shows the location of the DA node (at the field centre), with DP nodes (square icons) placed near the DA node because of the poor range of Wi-Fi (Section \ref{tech:wifi}). Circle icons represent the DC nodes and they link to the nodes with the fewest hops to a DA node, as long as that route includes a DP node. 

			\begin{figure}[!t]
			\centering
			\includegraphics[width=\textwidth]{Chap4/figures/topology}
			\caption{Section of Proposed Topology for K-HAS in Danau Girang}
			\label{topol}
			\end{figure}
			
		\subsection{Data Collection}
			A DC node is deployed along a ridge in the rainforest and consists of a wildlife camera with a wireless node attached. At 0200, the infra-red sensor detects movement and the camera triggers a set of 3 images to be captured. The DC node creates the DwC archive for the observation. Terms that describe the observation, such as time, date, species identified and location, is added to the meta.xml file and links to CSV files that contain the data for each term. Any field that can be completed, such as time, location and date, is added to the set.csv file. A separate CSV file is created that holds the filename of each image that was taken. The image is shown in Figure \ref{cl2}.
			
			\begin{figure}[!t]
			\centering
			\includegraphics[width=\textwidth]{Chap4/figures/leopard2.JPG}
			\caption{Clouded leopard Image Capture}
			\label{cl2}
			\end{figure}

			\begin{figure}[!t]
			\centering
			\includegraphics[width=\textwidth]{Chap4/figures/leopard_proc}
			\caption{Processed Image of clouded leopard}
			\label{clproc}
			\end{figure}
			
			The DC node runs its rules on the metadata of the images and infers that the image may contain a clouded leopard, this is because the image was taken in the early hours of the morning and the camera is deployed on a ridge. DC nodes run simple, non-chaining rules based on the file metadata and details about the node, such as location, but these rules are fixed from the time of deployment and are not updated until the network is redeployed. When bandwidth is restricted, DC nodes use a queue to prioritise both their own observations and those received from other nodes. Observations marked as interesting are moved to the front of the queue and any others are sent afterwards.
			
			The inference is included in the archive and is compressed. The node then sends it through every DC node between the originating node and the DP node assigned. To achieve the long range communications in the forest, Digimesh is used; the low transfer rate does mean that an archive can take several minutes to send but it allows for a range of up to 1km.

	\subsection{Data Processing}			
			The DP node receives the observation and it is unzipped and processed by the Darwin Core library. If the data has been preprocessed by a DC node and marked as interesting, then the processing is prioritised, otherwise it is added to the queue. In this case, the DC node believes the observation is interesting and may contain a clouded leopard, so it would be processed before other observations that may be queued and not marked as interesting. The images are read from the filenames provided in the CSV and processed using two methods. The EXIF tags in the image are extracted and the images themselves are processed using the Open Computer Vision (OpenCV) library. A unique feature of the DP node is that it uses two radios to allow links to both DA and DC nodes. DC nodes send archives using Digimesh, to achieve long range communication, and DA nodes use Wi-Fi, to provide a faster transfer rate than Digimesh and a more standard method that allows other devices to connect, such as mobile phones or laptops.
			
		\subsubsection{EXIF}
			EXIF (Exchangeable Image Format) tags are written to images at the time of capture. Examples of these tags can be time, date or camera serial number. The capabilities of the camera do affect how detailed the EXIF is, for example, a camera with GPS capabilities will enrich the image with the location. 
			
			Wildlife cameras have more functionality than common digital cameras, with details like moon phase, temperature and/or GPS location. Some devices even include the saturation, brightness and hue of each image. These capabilities allow the EXIF to be extremely detailed and this metadata can be used to find patterns in pictures that, when accompanied by local knowledge, assist with the classification of sensed data. 
			
			In this example, the knowledge base on the DP node is aware that clouded leopards and sun bears are nocturnal, but clouded leopards have previously only been seen when the temperature is between 30 to 35\celsius\ and only when the moon is not full. However, data on sun bear is not as complete and the knowledge base only shows that the bear is nocturnal and can be seen at any time of night in any area of the rainforest. The DP node identifies that the moon phase is not full and that the temperature is 32\celsius, from this it determines that the image could contain either animal and it cannot make a final conclusion.
			
		\subsubsection{Image Processing}
			Our Triton program, described in Section \ref{tech:sf:triton} is run on the set of three images. These images are converted to black and white and combined to build a background model for the complete set. The detected background is then removed and the final image is then searched for objects, where objects in the foreground will be shown with white pixels. The largest object is then found in the image and extracted to create a template, shown in Figure \ref{clproc}.
			
			Processed images of previously sensed images are stored on the DP node and associated with the confirmed classification, confirmed by a human or a node. Although the memory available on a DP node is typically around 32GB, this could easily fill in a matter of months if 3 full HD images were stored for every observation. Storing a single black and white template that contains a portion of the image is much more efficient and can still easily be associated with the classification made. The extracted image is then compared with the existing images, using the knowledge base to prioritise templates for comparison. In this example, nocturnal animals are prioritised and especially nocturnal animals with active projects associated. If the DP node has received an observation from the same DC node recently, then it will check for a classification on that and check for a match there first.

		As explained in Section \ref{tech:sf:triton}, species classification has not been implemented in K-HAS and templates are currently only associated with human classifications but this walkthrough describes how the process would be carried out if classification were to be implemented. These findings are written into the set.csv file of the DwC archive (Listing \ref{dwc:set}), using the identifier of the DP node as the `person' that identified the image and the scientific name for the clouded leopard as the species identified in the image. The archive is then zipped and sent on to the DA node, sending observations of interest first and delaying observations that have been found to contain nothing of interest.
				
		This observation is the first trigger from the DC node in the past few hours, so there are no recent classifications. However, processing of the metadata showed that the image was taken at night, so the DP node would use its knowledge base to match the images to templates of nocturnal animals first. Triton could then find a match to an existing template of a clouded leopard and completes its classification.
	
		\subsubsection{Classification}
		The metadata processing of the image shows that it could be any nocturnal animal that is known to come out when the moon is not full and the temperature is 32\celsius. This is not a complete classification but the image processing has found a match. 		
	\subsection{Data Aggregation}		
			Upon receiving a DwC archive, it is unarchived and processed by Darwin Core libraries, called by the middleware running on the node. The resulting archive is then inserted into a MySQL database and the files themselves are stored in a directory that maps to the DC node that captured the original observation. At the field centre, three users of the system have subscribed to updates for observations of clouded leopards.
			
			As the archive is processed by the library, the species is extracted and this triggers a rule to notify the subscribers. The rule then queries the database and finds their preferred method of communication. In this case, one is a lecturer and wants to be emailed while the two remaining are students and want to be notified via Twitter. An email is generated that contains the time, location and content of the observation, with the images attached, and sent on to the lecturer. The students are sent a short tweet that tells them a clouded leopard has been spotted and a link to the middle image in the sequence is provided; the middle image is used because local knowledge has shown that it is the most likely to contain the full subject in the image. In order to maintain privacy, a `direct message' can be sent on Twitter so this message is not public.
			
			The middleware on the node supports a web interface to allow users to perform administrative functions on the network, such as deploy a new node, on top of this there is a custom made website that shows all observations for every DC node. This allows users to log on and classify the images. In this case, the lecturer receives the email notification, reviews the attached images and clicks on the link to access the website to inform the DP node that the classification was correct. Due to the administrator position the lecturer has on the system, he is able to stop any users voting on the image and to simply confirm the classification.
			
			When a user classifies the observation, they see that a clouded leopard has been spotted in the same area on the same day for the past 5 weeks and they create a rule to automatically classify images from this camera that have a similar time (within an hour) and have an object extracted from them by the image processing. The user can then upload the rules through the same web interface and it will instantly become active on the system.

			If there was no classification, then users would be able to vote on the contents and use the classification with the highest vote, or the classification made by an administrator. Once a classification has been made, it is stored in the database and written to the archive. This triggers the DA node to send that classification on to the DP node that sent the original archive. In this case, the DA node informed the DP node that it has been confirmed as a clouded leopard and the DP node then stores the extracted image in the directory of clouded leopard templates, to assist with future classifications. This updated template causes the rule base of the DP node to be updated so similar data processed in the future would be correctly identified. The longer the network is deployed, the more knowledge DP nodes gain and the more accurate their classifications can be. For example, a change in season could cause a new, previously unknown animal to migrate to the rainforest. With human assistance, the animal can be identified and determined whether it is of interest. This knowledge can then be stored and sent on to DP nodes to prioritise and classify correctly the next time that it is captured. The feedback loop protocol is currently only a design and has been minimally implemented as a proof of concept.
					
	\section{Conclusion}\label{arch:conc}
		In this chapter we have explained the architecture we have developed to allow knowledge to be encoded and utilised within a wireless sensor network. Using tiers of nodes, with varying levels of knowledge-processing capabilities, we can process observations within the network and deliver data that has, where possible, already been classified. Working as more of a subscribe-push method, users do not have to check a DA node for new data, instead it is sent to them if it has been found to be part of a project they are subscribed to. If not, then the data is accessible to all users of the network through a web interface.

		%COMPARE WITH OGC SWE HERE!
		%http://essay.utwente.nl/59473/1/scriptie_R_de_Lange.pdf
		Using GSN as the application middleware allows sensors to be added, modified and maintained by those without technical knowledge and ensure future interoperability. However, it is not a standardised approach and using an architecture that implements the standard OGC SWE\cite{botts} would allow for interoperability through the use of standards not just for sensor networks but the Web as well. 
		
		One of the key features of K-HAS is that it is not a static deployment. The knowledge that the network holds at the time of deployment will rarely be the same as the knowledge held after a few months. Humans enrich the existing knowledge base and the nodes are able to make inferences about the data they are sensing, improving their classifications the longer they are deployed. When developing this architecture, GSN was the most robust architecture of those that we researched and tested, the support for many databases, administrative interface, native support for many widely used sensors meant that it was a better choice than a middleware that adopted the OGC SWE standards, especially as we were not interacting with SWE systems in our motivating scenario. However, while GSN is stable and mature, its large codebase does mean that there are dated features, such as using SOAP instead of REST and an unintuitive web interface that does not utilise web sockets. A middleware with similar automation on receipt of sensor data could be used instead of GSN that did follow the standards set out by the OGC. We believe that this should require few changes to the core K-HAS architecture as it currently stands.
		
		In Chapter 6, we explain how we implemented a variation of K-HAS in our motivating scenario and Chapter 7 shows our evaluation of the K-HAS architecture, but Chapter 5 will first outline the development of an ontology to support the architecture described here.
		
			
			
			
			
			
			
			
			

\chapter{An Ontology for the K-HAS Architecture}\label{chap:ont}
In this chapter, we explain the ontology that we developed to formally the components within K-HAS, the structure of the sensed data and the users involved with the network, as well as the format of the sensed data that is passed through the network.

This ontology is for those wanting to use K-HAS with a representation that is easy to reuse and deploy in a number of different scenarios, or even to select parts of the ontology that meet their requirements and implement those. Terms have been used to map to widely-used ontologies in the domains of scientific observation, sensors and people. For areas of the ontology where terms did not exist, we have used simple, common sensical names to represent them.
\section{Background}\label{bg}

%COMPLETE THE BACKGROUND FOR EXISTING ONTOLOGIES

%MAYBE INCLUDE SUBSECTION FOR DEVELOPMENT METHODOLOGIES (IS THIS 2 DIFFERENT PAPERS?)

K-HAS was developed in order to provide a generic architecture for wireless sensor networks to utilise the local knowledge contained within their environment to process sensed data and, therefore, make more efficient use of the network bandwidth by prioritising sensed data that is deemed to be more valuable. We have defined local knowledge as knowledge of an area that has been gained through experience, or experimentation.

Before we were able to implement K-HAS, we needed to model the flow of knowledge within the network. Developing an ontology for K-HAS means that we can do this, as well as provide a computer-readable model for all of the classes and components used by K-HAS, and the relationships between them.

Making it computer readable has allowed us to reuse classes in the development of software for each tier. For example, we were able to develop a common Darwin Core java library that is used in this ontology and in our GSN middleware to unzip and process received archives.

During the development of the K-HAS ontology, we researched existing ontologies that were commonly used in the domains that K-HAS covered. These included scientific observations and sensor hardware.

Looking into existing ontologies, we found that there had been many surveys on representing sensors in the semantic web: \cite{Compton2009}, \cite{Janowicz2010}, outlines the existing work. These surveys clearly highlighted that these ontologies had been split into two branches; observation-centric and sensor-centric.

Observation-centric ontologies, such as OBOE \cite{Madin2007b} and O\&M \cite{Botts},  focus on the data that is sensed, and its content; whereas sensor-centric ontologies, such as SSN \cite{lefort2011semantic} and SensorML\cite{Botts}, detail the components that make up a sensor and the operations they perform to turn sensed data into an output.

We found no ontologies that linked these concepts together to show the flow of data within the network and the role that each sensor plays in delivering this data. This could be because other WSNs do not integrate structure, hardware and sensed data in the same way as K-HAS or due to the fact that most WSNs do not process their sensed data until they reach a base station, so they would not need to model their data structure within the network. Because of this, we developed an aligning ontology that reuses existing ontologies, where possible, and introduces new classes that allows these ontologies to interlink. The result is an ontology that covers the flow of sensed data from the point of capture to the point it is received, processed, reviewed and stored at the end point of the network. Not only does the ontology cover how the data changes as it flows through the network but also the roles and capabilities of each tier.

K-HAS has been developed to be used with any sensors and is not specific to wildlife cameras, therefore we also looked into sensor-based ontologies that concentrate on the hardware and the individual capabilities of each device within the network.

From this we have categorised relevant existing ontologies into Observation-Centric and Sensor-Centric ontologies.

\subsection{Observation-Centric Ontologies}
%This can be seen as an ontological representation of the OpenGIS Observations and Measurements (O\&M) Standard \cite{botts2008ogc} that provides a framework for representing observations, measurements and procedures within sensor systems.

\subsubsection{OBOE}\label{OBOE}
The Extensible Observation Ontology (known in reverse as OBOE) is a popular suite of ontologies used to represent scientific observations \cite{Madin2007b}. Initially starting as base ontology for ecological observations, it has now grown into a suite of extensions that make it suited for chemistry, bioinformatics, anatomy and others. OBOE is represented by OWL-DL \cite{McGuinness2004} and allows the characteristics of a generic scientific observation to be linked to domain specific characteristics.

OBOE focuses on the concept of an \textit{observation}, which is made up of an entity, a measurement and a characteristic \cite{Madin2007}. An example of an observation could be a researcher observing an animal (the entity) and recording the gender (the measurement) as male (the characteristic). A single observation could then consist of multiple observations within it, such as gender, location, species and the number of species observed. 

    \begin{figure}
    \centering
        \includegraphics[width=\textwidth]{Chap5/figures/OBOEcore.JPG}
    \caption{The Core of an OBOE Observation}
    \label{oboe}
    \end{figure}

Figure \ref{oboe} shows the basic structure of a core OBOE observation, outlining the five key classes that are linked by seven properties. While this is the core structure of OBOE, domain specific extensions have been implemented by utilising `extension points' that are part of OBOE's core. OBOE follows the O\&M Standard (below) very closely, providing extensions to the core classes that allows more information about each to be encoded, adding context and enhancing the value of an observation.

The primary benefits of OBOE are that it is generic enough to cover almost all types of scientification observation and domain extensions allow for more specific details to be stored.

\subsubsection{O\&M}
The OpenGIS Consortium (OGC) Observations and Measurements Standard aims to provide a framework suitable for recording any observation made by a sensor, regardless of the domain \cite{Botts}. 

The key difference with O\&M, when compared to other ontologies, is that \textit{observation} and \textit{measurement} are not just classes. They also denote an action.
An observation is an action that causes a result, yielding a value and a measurement is a set of operations that provide some result(s).

O\&M provides a conceptual model, as well as XML encoding for observations and measurements. The listing \ref{omxml} shows an observation of a vehicle in a given time and place. Similar to the encoding of a measurement with the standard and protocol used in OBOE, O\&M provides support for the recording of the procedure used to gain the measurement for the observation. 

\begin{lstlisting}[caption=An Observation of a Vehicle encoded in O\&M, label=omxml, breaklines=true, language=XML]
 <?xml version="1.0" encoding="windows-1250"?>
 <om:GeometryObservation gml:id="geom1610" 
 xmlns:om="http://www.opengis.net/om/1.0" 
 xmlns:xsi="http://www.w3.org/2001/XMLSchema-instance" 
 xmlns:xlink="http://www.w3.org/1999/xlink" 
 xmlns:gml="http://www.opengis.net/gml" 
 xsi:schemaLocation="..Specialization_override.xsd">
   <om:samplingTime>
     <gml:TimeInstant>
       <gml:timePosition>2009-09-16T17:22:25.00</gml:timePosition>
     </gml:TimeInstant>
   </om:samplingTime>
   <om:procedure xlink:href="urn:ogc:object:procedure:ifgi:GPS"/>
   <om:observedProperty xlink:href="urn:ogc:def:phenomenon:OGC:Shape"/>
   <om:featureOfInterest xlink:href="urn:ogc:object:feature:vehicle"/>
   <om:result>
     <gml:Point srsName="urn:ogc:crs:epsg:4326">
       <gml:pos>40.7833 -73.9667</gml:pos>
     </gml:Point>
   </om:result>
 </om:GeometryObservation>
 \end{lstlisting}
 
 The listing shows that an observation centres around a \textit{feature of interest} that can be a physical object and the measurement is the detection of the vehicle at the recorded location. Figure \ref{oandm} shows a basic diagram of the ontology. While the event of a feature is linked, it is clear that the main focus is on the observation and the measurement associated with it.

    \begin{figure}[h]
    \centering
	\includegraphics[width=\textwidth]{Chap5/figures/o&m_ontology.png}
    \caption{The OGC O\&M Ontology \cite{Probst}}
    \label{oandm}
    \end{figure}
 
The structure of an observation within O\&M is focussed on the action and the result, this makes it suited for sensor networks across many domains that perform a wide range of observations. Because the users of the O\&M standard are spread across many domains, each with their own terms and definitions, the creation of an ontology for the standard aimed to remain as generic as other observation-centric ontologies. 

\subsubsection{Darwin Core SW}\label{bg:dwc}

In order to represent DwC occurrences, covered in Section \ref{arch:tech:dwc}, in an ontological format, work has been done to represent Darwin Core terms, as an ontology, in OWL. Darwin-Semantic Web (Darwin-SW) \cite{dwc_sw} is the project that aims to do this and many of the core terms associated with an occurrence have already been formalised.

    \begin{figure}[h]
    \centering
  \includegraphics[angle=90,width=\textwidth, height=\textheight, keepaspectratio]{Chap5/figures/dsw.jpg}
    \caption{Darwin-SW}
    \label{darwin-sw}
    \end{figure}

While Darwin-SW does not represent all of the classes within the DwC namespace, it contains the core classes required to record an occurrence, to a more-detailed level than OBOE. The downside of this is that the specificity of the terms limits DwC to ecological observations, making it far less generic than OBOE and other alternatives. There are positive aspects that, for those that need to record ecological observations, DwC allows for a greater level of detail and  can be mapped to OBOE with ease.

\subsection{Sensor-Centric Ontologies}
Sensor-centric ontologies are more focussed on the structure of the sensor, the network and the sensing processes involved. 

\subsubsection{SensorML}
The Sensor Markup Language (SensorML) Standard has been developed by the OGC, and complements the O\&M Standard, to enable the discovery and tasking of internet-connected sensors \cite{Botts}.

SensorML provides an XML schema for describing a sensor, its capabilities and the processes available. At the core, SensorML comprises of:
\begin{enumerate}
  \item Component - A physical process that transforms information from one form to another.
  \item System - Model of a group of components.
  \item Process Model - Atomic processing block used within a Process Chain.
  \item Process Chain - Composite block of Process Models.
  \item Process Method - Definition of the behaviour of a Process Model.
  \item Detector - Atomic part of a Measurement System.
  \item System - Array of components, relates a Process Chain to the real world.
  \item Measurement System - Specific type of System, mainly consisting of sampling devices and detectors.
  \item Sensor - Specific type of System that represents a complete Sensor.
\end{enumerate}

These definitions outline the core concepts of SensorML, a \textit{system} that performs one (or more) process(es) and is comprised of a group of \textit{components} \cite{Robin2006}. A SensorML document allow for a general, formal specification of a \textit{sensor} and its capabilities. The document describes a \textit{component}, outlining what data it reads in and the output once it has been processed. Several of these \textit{components} can the be used to create a \textit{system} and the primary goal of SensorML is to describe the process of how an observation came to be, focussing on the technical featuresof the node.

%\begin{lstlisting}[caption=SensorML Sample Document, label=sensormldoc, breaklines=true]
%<Component>
%  <keywords>
%    <KeywordList>
%      <keyword>weather station</keyword>
%  ...
%    </KeywordList>
%  </keywords>
%  <identification>
%    <IdentifierList>
%      <identifier name="uniqueID">
%        <Term definition="urn:ogc:def:identifier:OGC:uniqueID">
%          <value>urn:ogc:object:feature:Sensor:IFGI:thermometer123</value>
%        </Term>
%      </identifier>
%  ...
%      </identifier>
%    </IdentifierList>
%  </identification>
%  <classification>
%    <ClassifierList>
%      <classifier name="sensorType">
%        <Term definition="urn:ogc:def:classifier:OGC:1.0:sensorType">
%          <value>thermometer</value>
%        </Term>
%      </classifier>
%    </ClassifierList>
%  </classification>
%  <capabilities>
%    <swe:DataRecord definition="urn:ogc:def:property:capabilities">
%      <swe:field name="status">
%        <swe:Text definition="urn:ogc:def:property:OGC:1.0:status">
%          <gml:description>System operating values.</gml:description>
%          <swe:value>active</swe:value>
%        </swe:Text>
%      </swe:field>
%    </swe:DataRecord> 
%  </capabilities>
%  <inputs>
%    <InputList>
%      <input name="atmosphericTemperature">
%        <swe:ObservableProperty definition="urn:temperature"/>
%      </input>
%    </InputList>
%  </inputs>
%  <outputs>
%    <OutputList>
%      <output name="temperature">
%        <swe:Quantity definition="urn:ogc:def:property:OGC:1.0:temperature">
%          <gml:groupName codeSpace="ObservationOffering"> Weather </gml:groupName>
%          <swe:uom code="Cel"/>
%        </swe:Quantity>
%      </output>
%    </OutputList>
%  </outputs>
%</Component>
%  \end{lstlisting}

\subsubsection{SSN}
The Semantic Sensor Network (SSN) Ontology is the most fitting ontology for our requirements, as it covers systems processes and observations. Developed by the W3C after an extensive review of existing ontologies \cite{lefort2011semantic}, the SSN ontology is designed to allow focus on a variety of perspectives, such as the sensor within the network or the data that has been observed.

    \begin{figure}
    \centering
        \includegraphics[angle=90,width=\textwidth, height=\textheight, keepaspectratio]{Chap5/figures/ssnont.jpg}
    \caption{SSN Ontology}
    \label{ssnont}
    \end{figure}
    
Figure \ref{ssnont} shows the model for the SSN ontology, displaying the relationships that connect each class, as well as their associated properties. It also highlights the modular approach that has been taken, separating the system from the process and the observation.

The observation pattern of SSN is centred around `Stimuli-Sensor-Observation', which can be simply described by an event causing a sensor to trigger and create an encompassing observation to store details about the event, as well as the device that recorded the event. While SSN is focussed on sensors, capturing the measurement capabilities of each sensing device that makes up a system and specifics on its lifespan, the development of the ontology arose from the review of sensor-centric ontologies as well, recording data about observations is a structure very similar to O\&M.

\subsubsection{SUMO}
The Standard Upper Ontology (SUMO) is an ontology that links sensor hardware and sensor data ontologies to assist with searching and evaluating distributed and heterogenous sensor networks \cite{Eid2007}. SUMO combines the Sensor Data Ontology (SDO) and Sensor Hierarchy Ontology (SHO) with the ability to `plug in' extension ontologies. SHO contains knowledge models for the hardware of the network, such as sensors and data processing units, allowing attributes like the accuracy and/or range to be specified. SDO describes the data sensed by the network, including the spatial and/or temporal observations.

SUMO could be classed as both an observation-centric and sensor-centric ontology, because it does link ontologies that model both sensing and sensed data and, while SUMO's development did follow a similar approach to the one described in this chapter, combining existing ontologies, it is more focussed on retrieval of sensed data through queries constructed using both hardware details and properties of the sensed data and does not fulfil all of the requirements we identified when developing K-HAS.

\subsubsection{Findings}
There are many ontologies that are suited for sensor-based scientific observations and ontologies that allow for the description of sensor hardware are very complete. However, in our research, we were unable to find a complete ontology that allowed for a sensor to be described as well as specifics of the observation. SSN was the closest that we found for this, but it was still very hardware focussed and aimed at a technical audience. SSN primarily describes the hardware of each node and their capabilities, whereas our requirements are for both hardware and sensed data to be represented in a single ontology.

We found that many of these ontologies satisfied many subsections of K-HAS completely, meaning that it would be unnecessary to reproduce these concepts. From these findings we have developed an alignment ontology that connects ontologies across multiple domains to support our proposal of K-HAS, creating an ontology that is both sensor-centric and observation-centric.

\section{Method}
% *Outline standard ontology development method
% *Explain our variation
% Reference the various methods outlined

Before we began development, we researched existing methods that have been used to develop current ontologies. From this, we found three well-documented methodologies; the methodology used to develop the Toronto Virtual Enterprise (TOVE) ontolgy, the Methontology and the methodology used to develop the Enterprise Model ontology.

The methodology by Uschold and King \cite{Uschold1995}, used to develop the Enterprise ontology, is a four step method that was most suited to the processes we expected to follow. The steps are not covered in great detail in the original paper but, research since provides greater detail for each step.

\subsection{Identify Purpose}
The aim of this step is to identify why the ontology is being built and what requirements it is supposed to fulfil. This includes considerations, such as the audience, the intended purpose and the specificity of the ontology.

Other methodologies have used more structured methods to informally identify the requirements of the ontology. Gruninger and Fox \cite{Gruninger1995} propose competency questions; these are questions that are used to identify the problems that the developed ontology is developed to solve. These questions can act as a benchmark, showing that an ontology is sufficient if it can solve the questions raised here.

\subsection{Build the Ontology}
Building the ontology can be broken down further into 3 smaller steps: capture, code and integrate existing ontologies.

\subsubsection{Capture}
 % - identify key concepts, provide unambiguous text definitions
The capture stage involves defining the concepts and terms that the ontology will model, as well as how they map to the real world. This can be done in one of three ways:
\begin{enumerate}
\item Top Down - Starting with the core conepts, create the more specialised classes until you have identified all subclasses.
\item Bottom Up - This process begins with the more specialised classes and grouping them into the more general classes towards the top of the heirarchy.
\item Middle Out - A combination of the two, this process involves specialising, and generalising, the classes identified in the middle layer of the hierarchy.
\end{enumerate}
% REFERENCE THE ONTOLOGY 101 PAPER
As for the capture of the knowledge used to identify the classes needed, this is an area that has been documented, but primarily provides recommendations. \cite{Fernandez-Lopez1999} suggests interviews with domain experts, iteratively brainstorming with a group actively involved in the development of the ontology. This stage has often been referred to as the knowledge-acquisition stage.
% and **REFERENCE** recommends the use of Common KADS \cite{Hoog1992}, a knowledge analysis methodology used to develop knowledge-intensive systems.

\subsubsection{Code}
 % - perform the above step in a formal language (i.e. OWL)
This step involves coding the terms identified in the previous step into a formal language, such as the Web Ontology Language (OWL) \cite{McGuinness2004}.

\subsubsection{Integrate Existing Ontologies}
 % - join all ontologies that match/overlap with the identified terms
This step can carried out at the same time as the step above, so that the ontology is developed with existing ontologies in mind. This also allows overlap to be identified, and incorporated, early. Being aware of commonly-used ontologies within the domain, before development begins, is a more logical approach and does avoid the need to create new terms for existing concepts.

Existing ontologies can be integrated by importing them and linking the existing terms to the terms identified in the ontology that is being developed. However, there is also the method of developing the ontology completely, so that it is consistent without the need to rely on existing ontologies and creating `sameAs' relationships between the terms identified and only the required terms in the existing ontologies.

In \cite{Jimenez-Ruiz2008}, it is recommended that, when importing external ontologies, the whole ontology is not used. Rather, one should aim to extract only the required fragments from the external ontology that are relevant to the concepts in the developed ontology.


\subsection{Evaluation}\label{method:eval}
For the evaluation, Uschold and King adopt the definition of \cite{Fernandez-Lopez1997}: “to make a technical judgement of the ontologies, their associated software environment, and documentation with respect to a frame of reference..." The frame of reference may be requirements specifications, competency questions, and/or the real world”.

There are some well documented methods for evaluating ontologies in the literature, \cite{Uschold} proposes using the competency questions, used in the first step, to ensure that they can all be fully answered by the finished ontology. 

\subsection{Documentation}
Although this step is listed at the end of the development cycle, it seems more fitting to document all major aspects of the ontology as it is being developed. Documentation of the ontology should include: any assumptions, all concepts introduced, all ontologies that have been incorporated (by whatever means) and any primitives used for the definitions of concepts.

\section{Results}
This section details our results when using the Uschold and King methodology, accompanied by more recent research for particular steps, to develop the K-HAS ontology. 

\subsection{Identify Purpose}\label{ident}
The need to create the K-HAS ontology was partly due to the reasons outlined by Gruninger and Fox \cite{Gruninger1995}, we had identified a problem as we were developing a sensing architecture that utilised local knowledge: how could we formally represent the flow of knowledge and sensed data throughout a wireless sensor network?

To determine the scope that K-HAS should cover, we identified a set of competency questions that represent what we expect K-HAS to cover within the domain. In order to present this, we used an approach similar to that of \cite{Choi2010}, which can be seen in Table \ref{tab:comp_qs}.

\begin{table}[h]
\centering
\begin{tabular}{ p{10cm}}
\hline
\textbf{Competency Questions}\\ 
\hline
Find all \textbf{Occurrences} of an \textbf{Individual} \\
Find all \textbf{Occurrences} of an \textbf{Individual} at a \textbf{Location}\\
Find all \textbf{Occurrence} within a specified \textbf{Date} and \textbf{Time} range \\
Find all \textbf{Sensors} that have recorded an \textbf{Occurrence} of an \textbf{Individual}\\ 
Find all \textbf{Locations} of an \textbf{Individual}  \\
Find the storage location of a \textbf{Project} \\ 
Find all \textbf{Projects} containing an \textbf{Individual}\\
Find all \textbf{People} involved in a \textbf{Project}  \\
Find all the \textbf{Evidence} that supports an \textbf{Occurrence}\\
Find all the \textbf{Types} of evidence that supports an \textbf{Occurrence} \\
\hline 
\end{tabular}
\caption{Competency Questions} 
\label{tab:comp_qs}
\end{table} 

These questions allow us to identify the core concepts that the ontology needs to represent, as well as serving as a tool to evaluate the completed ontology.

\subsection{Build the Ontology - Capture}\label{buildcapture}
This part of the development cycle is the identification of the concepts and their implementation. The first step is capture the knowledge that will be used to identify the core concepts within the ontology. 

To capture the knowledge for the ontology, we used a combinatorial approach of those outlined in the literature. We interviewed domain experts, as well as overseeing a basic implementation of a system, and we iteratively brainstormed the concepts throughout the development cycle.

Over the course of eighteen months, which involved 2 field visits and several brainstorming meetings, we identified the core concepts that the ontology would need to contain, as well as the properties that would link them. From this, it would seem that we followed the \textit{Top-Down} approach, but the first visit with domain experts also allowed us to identify some of the more specialist classes early on in the development cycle. Thus, it seems that in practice we followed a more \textit{Middle-Out} approach.

Table \ref{tab:concepts} outlines the core concepts we have identified for K-HAS to be complete, as well as the definitions we have used for the architecture. When integrating existing ontologies, the definitions of similar terms would need to match with our definitions or we would not deem them the same as the K-HAS concepts.

\begin{table}
\centering
\begin{tabular}{p{5cm}|p{2cm}|p{5cm}}
\hline
\textbf{Concept} & \textbf{Subclass Of} & \textbf{Definition}\\ 
\hline
Occurrence & - &  \\
Identification & - &  A text-based recording of the content of an Occurrence\\
Evidence & - & Media to support the Identification, such as a photo or recording.\\
Location & - & The location of the Occurrence.\\ 
Date & - & The date of the Occurrence.\\
Time & - & The time of the Occurrence.\\ 
Project & - & Project(s) that can contain many Occurrences\\ 
Node & - & A device with sensing capabilities.\\ 
Data Collection (DC) Node & Node & Node with limited knowledge-processing capabilities charged with sensing a feature (or features) of its environment. \\
Data Processing (DP) Node & Node & Node with knowledge-processing capabilities charged with serving a subset of DC Nodes and processing their sensed data. \\
Data Aggregation (DA) Node & Node & Node with knowledge-processing capabilities that stores all knowledge and sensed data for the whole network. \\
Person & - &\\ 
Administrator & Person & Person in charge of a project (or projects). \\
Worker & Person & Person involved with a project. \\
\hline 
\end{tabular}
\caption{K-HAS Concepts} 
\label{tab:concepts}
\end{table}

Mapping these concepts, a diagram of the base K-HAS ontology is shown in Figure \ref{khas_base_ont}. The next step is to create the ontology and map the concepts identified to existing ontologies within the same domain space(s).

    \begin{figure}[h]
    \centering
        \includegraphics[angle=90,width=\textwidth, height=\textheight, keepaspectratio]{Chap5/figures/khas_base.JPG}
    \caption{K-HAS Base Ontology}
    \label{khas_base_ont}
    \end{figure}



\subsection{Build the Ontology - Code}
Once the concepts had been identified (and agreed upon), the ontology must be coded. We used Protege 4.2.2 \cite{Noy2000a} and implemented K-HAS in the Web Ontology Language (OWL), creating a core file that could be expanded; should we need to import existing ontologies.

\subsection{Build the Ontology - Integrate Existing Ontologies}
Our proposal for K-HAS is focused on scientific observations. Because of this, the focal point of our existing ontology research is of ontologies that are centred around scientific observations, this allows us to create a more generic ontology that is still able to capture all of the semantic details associated with a wildlife observation.

As explained in Section \ref{bg}, our research of existing ontologies covered two categories: observation-centric and sensor-centric. We identified ontologies, within multiple domains, that satisfied many of our requirements for K-HAS, but not all. Using the core concepts outlined in Section \ref{buildcapture}, we created a minimal ontology and used the results of our research to map K-HAS concepts to those that had already been identified.

During this process we found that we had determined concepts that mapped to existing ontologies, but may not share identical structures, or even the same name. For example, the OBOE ontology describes the concept of an occurrence, which is very similar to our occurrence and the occurrence in the Darwin Core-SW ontology. When we found these mappings, we used \textit{sameAs} relationship to form a link that allows data stored according to these existing ontologies to map directly to K-HAS. However, the structure of an OBOE observation is, as outlined in Section \ref{OBOE}, is more generic for all types of scientific observation and depicts the observation of an entity, containing a measurement of a particular characteristic. Whereas the structure of a Darwin Core observation is more limited to scientific observations of taxa which allows it to have more predefined terms, such as location, species name and the evidence for the recorded individual. Because of this, it was difficult to create a structure that encapsulated both OBOE and Darwin Core due to the generality of an OBOE observation compared to the more specific structure of DwC.

Research showed that Darwin Core maps to K-HAS' requirements more completely, as well as the fact that there has been work to represent Darwin Core Observations in OBOE \cite{dwc_oboe}, K-HAS occurrences map directly to Darwin Core and elements that are similar to OBOE have been linked by the \textit{sameAs} relationship. This does mean that full OBOE observations cannot map directly, but K-HAS, and Darwin Core, occurrences can be converted, if necessary. However, using the terms in our ontology we can create a subset of an observation that does not include all of the terms defined in OBOE but can still map to all three ontologies.

For the sensor hardware of K-HAS, we found that the SensorML ontology maps directly to our concepts and we also realised that we did not need to recreate concepts that may already exist in popular ontologies outside of the domain spaces we researched. For example, the Friend of a Friend (FOAF) ontology \cite{Document2010} is an ontology designed to create machine-readable pages that describe people, so it is more logical that K-HAS reuses existing terms from popular ontologies to allow pre-existing data to be mapped with ease.

% INCLUDE TABLE TO MATCH TERMS TO EXISTING ONTOLOGIES??

\subsection{Extend the Ontology}

Whilst researching widely-used ontologies, we became aware that some classes identified for K-HAS were not satisfied by what is currently available, these are prefixed with khas and shown in Figure \ref{khas_ont}. The final step for creating the ontology was to add these concepts to the linked ontologies, we call these \textit{extension concepts}, concepts unique to K-HAS which do not exist in any other ontology. These terms were linked to the unique layered architecture of K-HAS and we defined them within a new K-HAS namespace, changing our ontology from an alingment of existing ontologies to an extension of these.

The final ontology is shown in Figure \ref{khas_ont} (with the complete code in Appendix \label{appendix:ontology}) and shows how each concept maps to existing ontologies. The figure shows that there were only five extension concepts unique to K-HAS that can be subclassed from the \textit{person} concept in the FOAF ontology and \textit{node} in the SensorML ontology.

    \begin{figure}[h]
    \centering
        \includegraphics[angle=90, width=\textwidth, height=\textheight, keepaspectratio]{Chap5/figures/khas_ontology.JPG}
    \caption{K-HAS Ontology}
    \label{khas_ont}
    \end{figure}
    
\section{Evaluation}

Uschold and King's ontology development method does not explain how to evaluate the created ontology and much of the literature describes a number of methods that can be employed, \cite{Brank2005} outlines a number of different evaluation techniques and separates them into categories. These categories are listed below:
\begin{enumerate}
	\item Comparing the ontology to a `golden standard'.
	\item Using the ontology in an application and evaluating the results.
	\item Comparing the ontology with a collection of documents from the domain to be covered.
	\item Manual evaluation by humans to test if the ontology meets a set of predefined criteria.
\end{enumerate}

The last category mentioned is similar to the method in Section \ref{method:eval} that uses the competency questions to determine the effectiveness of the ontology. Because there is no agreed method,we have evaluated K-HAS by testing it in the application it was developed in (Protege), mapping it to existing documents within the intended domain and ensuring that it can satisfy all of the competency questions we outlined in Table \ref{tab:comp_qs}.

\subsection{Using the Ontology in an Application}
We developed the K-HAS ontology in Protege, a Java based ontology editor, that also provides the functionality to reason over the ontology and check for inconsistencies. Using the Hermit reasoner that is built into Protege, the ontology was determined to be logically consistent.

To confirm the validation provided by Protege, we also used a web-based ontology validation tool, called the OntOlogy Pitfall Scanner (OOPS), that scans an ontology for common errors, such as defining incorrect inverse relationships or using recursive definitions, that can occur during the development phase \cite{Poveda-Villalon2012}. The results of this tool showed that no pitfalls had been detected.

We have a customised build of K-HAS running that receives data from a variety of sources and stores it in a MySQL database that allows users to classify through a web site. The schema of the database maps to our ontology, scientific observations are received, unzipped and stored in the relevant fields in the database. Using real sensed data, we have successfully stored over two hundred occurrences and mapped existing DwC occurrence to K-HAS.

\subsection{Competency Questions}

In Table \ref{tab:comp_qs} we have identified some basic competency questions that we expected K-HAS to be able to satisfy with the concepts we had identified, Table \ref{eval:comp_qs} shows how the original questions are satisfied by K-HAS. We have mapped the terms we originally used to the terms used in K-HAS (and the linked ontologies) and shown the concepts involved with each question, as well as the relationships linking them.
\begin{table}[h]
\centering
\begin{tabular}{ p{5cm}p{5cm}p{5cm}}
\hline
\textbf{Competency Questions} & \textbf{Concepts} & \textbf{Relationships}\\ 
\hline
Find all \textbf{Occurrences} of an \textbf{Individual} & Occurrence; Individual & Occurrence occurrenceOf Individual \\
Find all \textbf{Occurrences} of an \textbf{Individual} at a \textbf{Location} & Occurrence, Token, Individual, Location, Identification & Occurrence hasEvidence Token \\ 
& & Token isBasisForID Identification \\ 
& & Identification identifies Individual \\
Find all \textbf{Occurrence} within a specified \textbf{Date} and \textbf{Time} range & Occurrence; Event; Date; Time & Occurrence atEvent Event \\
& & Event hasTime Time \\ 
& & Event hasDate Date \\
Find all \textbf{Data-Collection Nodes} that have recorded an \textbf{Occurrence} of an \textbf{Individual} & DCNode; Individual; Occurrence & Occurrence recordedBy DCNode \\
& & Occurrence occurrenceOf Individual \\  
Find all \textbf{Locations} of an \textbf{Individual} & Occurrence; Individual; Event; Location & Occurrence occurrenceOf Individual \\ 
& & Occurrence hasEvent Event \\ 
& & Event locatedAt Location \\
Find the storage location of a \textbf{Project} & Project, Data Aggregation Node & Project storedBy DataAggregationNode \\ 
Find all \textbf{Projects} containing an \textbf{Individual} & Individual, Project; Occurrence & Individual occurredIn Occurrence \\ 
& & Occurrence heldIn Project \\
Find all \textbf{Persons} involved in a \textbf{Project} & Person; Administrator; Worker; Project & Administrator hasProject Project \\ 
& & Worker hasProject Project \\
Find all the \textbf{Evidence} that supports an \textbf{Occurrence} & Occurrence; Token & Occurrence hasEvidence Evidence \\
Find all the \textbf{Types} of evidence that supports an \textbf{Occurrence} & Occurrence; Token; Type & Occurrence hasEvidence Token \\ 
& & Token isOfType Type \\
\hline 
\end{tabular}
\caption{Competency Questions} 
\label{eval:comp_qs}
\end{table} 

The table shows that all of the competency questions outlined in Section \ref{ident} have been satisfied by the final ontology and each concept used maps quite easily with little to no change.

\subsection{Comparing the Ontology with a Collection of Documents}
Within the domain of scientific observations, Darwin Core is a popular choice for observations of wildlife and plants, it was because of this that we chose to use many of the pre-existing concepts from Darwin Core in K-HAS. 

In the data driven approach suggested by \cite{Brewster2004}, a corpus of documents, related to the domain that the ontology covers, and the keyword content is matched with the terms used in the ontology. Because Darwin Core observations are structured into archives of files, explained in Section \ref{bg:dwc}, we evaluated the K-HAS ontology by combining the approach of comparing a corpus of documents related to the domain with human evaluation of ensuring an ontology met a set of requirements to create a method that ensured existing scientific sensed observations could be mapped to K-HAS with little to no modification.

As explained in Section \ref{arch:tech:dwc}, an archive of Darwin Core files consists of a minimum of 3 files: a metadata file that contains information about the creator of the archive and the project it relates to (eml.xml), a metadata file that describes all of the files that contain the occurrence and the fields within them (meta.xml) and a csv file that contains the data relating to the occurrence itself. Within the archive, the core files that need to be mapped to K-HAS are the meta.xml and eml.xml files as this allows us to store what and who.

We used files from a DwC archive made available online to extract the terms associated with an occurrence and we mapped these terms to K-HAS concepts, the results can be seen in Table \ref{tab:dwc_eval}.

\begin{table}[h]
\centering
\caption{Evaluation of K-HAS against Darwin Core Occurrence Terms} 
\begin{tabular}{ p{5cm}p{5cm}p{5cm}}
\hline
\textbf{Darwin Core Term} & \textbf{K-HAS Concept}\\ 
\hline
occurrenceID & Occurrence \\
basisOfRecord & Basis of Record \\
recordedBy & Person/Node \\
associatedMedia & Token \\
eventDate & Event Date \\
eventTime & Event Time \\
locationId & Location \\
scientificName & Individual \\
identifiedBy & Person \\
dateIdentified & Identification \\
\hline 
\end{tabular}
\label{tab:dwc_eval}
\end{table} 

Each term within a Darwin Core observation maps to K-HAS with only a few changes to the terms, while the definitions do not change. As K-HAS is an alignment ontology, there are extra concepts that do not map to Darwin Core, but can encapsulate each occurrence. For example, a project in K-HAS can contain many occurrences. As previously mentioned, \cite{dwc_oboe} shows how a DWC archive can be represented in OBOE, which means that it would not be too complex to map an OBOE observation in K-HAS.

\section{Conclusion}

In this chapter we have presented an ontology for the K-HAS architecture and described our methodology for development, as well as showing that existing ontologies are not complete enough to cover our requirements for K-HAS. The K-HAS ontology we present is an alignment of ontologies that are spread across multiple domains and provides a complete solution for a sensor network that deals with scientific observations and we believe this is the first extension ontology that combines sensor-centric and observation-centric ontologies. We have also extended this alignment to include concepts that model the unique features of K-HAS, but this ontology should be suitable for any sensor network that deals with observations.

Our evaluation has shown that the ontology is logically consistent and that existing scientific observations, in other formats, can be mapped across to K-HAS with few changes. The competency questions we identified during the design phase of the ontology can all be satisfied by the resulting ontology and our K-HAS network currently stores data with a schema that follows the design of this ontology.

In the future, work would be done to make the Darwin Core terms more modular and users will then be able to `plug in' their occurrence structure of choice. Currently, this ontology has been developed with the scientific observations of our motivating scenario in mind. The modular design allows it to be adapted to suit observations with different structures. The K-HAS, FOAF and SensorML modules can all be reused in almost any WSN but not all networks will be tasked with recording individuals. In these cases, DwC (and some K-HAS) terms would need to be replaced with terms suited to the task of the network, such as flood monitoring.


\chapter{Implementation}
	In this chapter, we discuss the implementation made for our motivating scenario, a deployment where the hardware choices were using the 'cutting edge' technology that was readily available and software packages that can be re-used in any scenario. 
	
	Experimentation in the Malaysian rainforest revealed that our ideal implementation of K-HAS was not feasible for a prototype that could be used by those without any domain knowledge. This was mainly due to the effects of the humidity and vegetation on the range of the DC nodes, but also because of the difficulty creating a DC node with a camera that could be remotely controlled.
	
	The key point of K-HAS' development was to show that local knowledge can improve the efficiency of the network, the quality of the data received and automate the processing of sensed data. In order to implement a network that showed this, we had to modify the architecture of K-HAS to create a network that was designed specifically for its use in Malaysia, as well as using commercial hardware that could be used by anyone. 
	
	While this implementation does not exactly match the architecture of K-HAS, the principles are the same and the implementation is based on the best technology that is currently available. We call this modification to K-HAS, the Local-knowledge Ontology-based Remote-sensing Informatics System (LORIS).
	
	LORIS uses the same approach to knowledge-processing as K-HAS but, as experimentation revealed, the ideal architecture of K-HAS is not currently feasible in a humid rainforest and sensor nodes cannot be deployed 1km apart. To address this, we made a modification to the tiers used in K-HAS and adopted commercial hardware solutions that would overcome the difficulty we experienced with our own sensor solutions.
	
	\subsection{Hardware Used}
		\subsubsection{Data Collection}
			We described the experiments performed on wireless transmissions in Chapter \ref{chap:technical} and outlined that Wi-Fi communications in a humid rainforest was not possible beyond 30m, at least at ground level. We also discovered that Digimesh, despite the reported range of up 60km, was also limited. Taking this into consideration, we looked into a commercial solution that would be able to capture images and transmit over distances of more than a few hundred metres. The limitation of this approach means that no local processing can be performed on the device and it cannot run any custom programs. However, the benefit is that wireless sensing cameras are provided with the necessary hardware and robust software that allows for remote configuration and retrieval of images.
			
			With this in mind, we ordered 3 Buckeye X7D wireless cameras with a reported range of 1 mile.
		\subsubsection{Data Processing}
		\subsubsection{Data Aggregation}	

	\subsection{Software Used}
		\subsubsection{Data Collection}
		\subsubsection{Data Processing}
		\subsubsection{Data Aggregation}	
	
\chapter{Simulations}
In this chapter, we detail simulations that we developed to test our hypothesis with a complete implementation of K-HAS. LORIS was implemented because current technology does not fully support the requirements of the K-HAS architecture, because of this we implemented a simulation of K-HAS that satisfies all of its requirements.

Using RePast, a java-based network simulation tool, we created a network to emulate K-HAS. Repast is an agent-based modelling system that allows for agents to be created and placed on a grid. Ticks denote a period of time and simulations can run for a fixed number of ticks, or until stopped. Ticks can also be used to schedule events, such as searching for neighbours, by calling methods that last for a set number of ticks, or begin at a particular tick. Similar to the chaining of rules we described with Drools, ticks are the core of Repast's scheduling mechanism which can be used to schedule single events, as well as chaining events. For example, a train may take three hundred for it to arrive at its destination. When the simulation reaches three hundred ticks, a scheduled event could run that would open the train doors, make an announcement and so on.

Agents are created, using the RePast SDK, and Java classes are used to manipulate their behaviour. Simple networks may only contain basic agents with only a few variations from those provided by RePast. However, for more complex networks, a hierarchy of agents is required and Java's inheritance can then be used create subclasses of an agent.

A 2D space is used to display the grid and the simulation is run within Repast's own GUI. This GUI provides functionality such as editing the properties of classes, integrating with Matlab, taking screenshots and saving different configurations of the same network.

This chapter is structured as follows. Section 2 describes the implementation of the network. Section 3 outlines the results and Section 4 compares these with LORIS and the current solution in our motivating scenario. Section 5 concludes our findings and highlights areas that require further experimentation.

\section{Implementation}\label{sim:imp}
	While the aim of these simulations was to show the effectiveness of K-HAS over the current solution, we also wanted to determine if it was the optimal solution, in terms of delivery of interesting data and in terms of network lifetime. The ideal solution would be to attach DP nodes to all cameras in the network, however the short battery life means that replacements would be made as often as the current manual solution. Because only one of these simulations matches K-HAS, the terminology used in this chapter is more generic and focussed on nodes that have knowledge but do not match our definitions of DP and DC nodes. Instead we use the term \textit{sensing node} to define a node that is capable of capturing an observation, \textit{routing node} to define a node that processes the data from sensing nodes and \textit{central node} to define the end point of the network, where data is made available to users. While these definitions are similar to sensing, routing and central nodes the key difference is that sensing and routing nodes possess have varying levels of knowledge processing capabilities, explained below:
	
	\begin{itemize}
		\item No Knowledge (NK): The node possesses no knowledge processing capabilities.
		\item Low Knowledge (LK): The node possesses basic knowledge processing capabilities and contains a static rule base.
		\item High Knowledge (HK): The node possesses high knowledge processing capabilities and is able to process data, metadata and use a dynamic rule base.
	\end{itemize}
	
	HK nodes, while more accurate, their battery life is limited to 3 weeks. In contrast, less accurate LK nodes can run for approximately 3 months without human interaction. The scenarios we have implemented cover the possible combinations of HK, LK and NK, in a twenty five node network. These were developed to determine which combination of LK, NK and HK nodes allowed for the greatest network lifetime, as well as the greatest accuracy when delivering interesting sensed data, and they have been outlined below:
	
	\begin{itemize}
		\item NK-ALL: Sensing and routing nodes possess no knowledge processing capabilities.
		\item LK- ALL: Sensing and routing nodes possess low knowledge processing capabilities.
		\item NK-LK: Sensing nodes possess no knowledge processing and routing nodes have low knowledge.
		\item LK-HK (K-HAS): Sensing nodes have low knowledge and routing nodes possess high knowledge.
		\item HK-ALL: Sensing and routing nodes have high processing capabilities.
	\end{itemize}

Before implementing, we designed the agents required based largely on the existing ontology. Using that, we created a hierarchy of nodes inheriting common properties from a node object. As previously mentioned, we had metrics on range and transmission times from previous experiments and the deployment of LORIS. We used these to create properties for each transmission medium that could be used by each node object.

We also needed to create an object to represent the DwC archives, a Drools REST API had been implemented to work with the LORIS web interface, and much of the DwC archive code was reusable within RePast. 

The structure of the simulation is as follows: The \textit{network builder} instantiates all the nodes, places them randomly on the grid and schedules events once the simulation has started. The nodes then use the properties of their transmission medium to find nodes in range and create a connection; depicted by a line between the node. The simulation uses metrics extracted from the images taken at Danau Girang and the chance of an image being captured by a camera is based on the average capture rate of a camera. The fire rate has been calculated by the average number of pictures captured in a day taken by each camera. Figure \ref{fig:sim} shows an example simulation of the LK-HK scenario.


	\begin{figure}[h]
	\centering
	\includegraphics[width=0.70\textwidth]{Chap7/figures/khas_sim}
	\caption{K-HAS Simulation}
	\label{fig:sim}
	\end{figure}


%\begin{itemize}
%\item Network Builder
%\item Node
%	\begin{itemize}
%	\item Sensing
%	\item Routing
%	\item Central
%	\end{itemize}
%\item Darwin Core
%	\begin{itemize}
%	\item Identification
%	\item Location
%	\item Occurrence
%	\item Image
%	\item Species
%	\end{itemize}	 
%\end{itemize}

\subsection{Darwin Core}
The Darwin Core class represents a DwC archive, encapsulating \textit{Identification}, \textit{Location}, \textit{Occurrence}, \textit{Image} and \textit{Species}. The images we have collected from Danau Girang were processed to find details such as the average size when captured at night and day, how often an average camera triggers and the percentage of images with animal content. These data were then used to specify how often a randomly placed node should capture an observation per tick.

Upon each capture, images are created, given a random size based on the maximum and minimum size found in the 120,000 images collected from DG and the sum of the images is used to calculate the size of the archive. Using this size, a sensing node calculates how long the archive takes to send based on the size and the transmission rate. We assume that the rate stays constant for the duration of transmission.
When an archive is sent to the routing node, we used the average time for our image processing tool and Drools engine to run and attempt a classification, which is 143 seconds (ticks). To keep the classifications as general as possible, so that the simulation applies to any WSN for scientific observations, archives are not classified down to the species level, they are marked as \textit{interesting} or \textit{empty} and then forwarded to the central node.

\subsection{Routing}
The routing protocol used needs to be dynamic in order to adapt to nodes being added and removed during deployment, while minimising traffic in a resource constrained network. In our approach, we use a modification the Minimum Cost Forwarding Algorithm (MCFA), described in Section \ref{bg:rp}. A cost is assigned to each node, based on how far they are from the central node, with neighbouring nodes choosing to connect to the node with the lowest cost. However, in normal implementations of MCFA, all nodes are of the same type and simply need to connect to a base station. This protocol is used in all scenarios.

In our architecture, sensing nodes cannot connect directly to a central node because processing would not take place. Our implementation of MCFA works with a discovery phase and a transmission phase. The discovery phase is a scheduled event, taking place at the start of deployment but it can be run throughout deployment to react to nodes being added or removed. 

\subsubsection{Discovery}
	Discovery begins at each central node, scanning nodes in range for routing nodes and sending a broadcast packet, with a cost of 0, to inform them that they are within range of a central node. Links between Central and Routing nodes use W-Fi in all of our scenarios. Once received, routing nodes increment the count and forward the packet to any routing nodes within range of them, where we use the range of Zigbee. We found that this method overloaded the routing nodes and all sensing nodes within range would connect to the first routing node they receive the broadcast from. We then implemented a method, called \textit{load balancing}, which uses the sensing nodes connected to a routing node to calculate whether it should offload new nodes to a neighbouring routing node.
	
	The maximum connections a routing node can have is determined by the total number of sensing nodes in the network divided by the total number of routing nodes, which is held in the knowledge base of the central node. Once a routing node has the maximum number of connections allowed, it starts to offload to a neighbouring routing node that is also in range of the sensing node requesting a connection. If there are no neighbouring nodes then it the node exceeds the maximum number of connections allowed, to save sensing nodes being left with nowhere to send their data.
	
	If the sensing node that receives the broadcast does not have an existing route to a central node, or the cost of the current route is higher than the received route, it adds an edge to the routing node, increments the count and forwards it to all nodes in range. This process continues until the broadcast reaches the edge of the network. Nodes do not have global knowledge of the route to the central node, only of their neighbour with the lowest cost.
	
	This phase can be repeated throughout the course of the deployment, simply by scheduling it as an event to occur every \textit{n} ticks. However, the simulation currently only uses the discovery phase at the beginning of the deployment.
	
\subsubsection{Transmission}
	Once the discovery phase has been completed, providing nodes are within range of the central node, the transmission phase begins where only DwC archives are then sent across the network. Observations are captured based on the mode of the simulation and sent to the lowest cost neighbour.
	
	In order to manage transmissions, sensing nodes have a \textit{SendState} object that contains the next archive to send, the time to send it and whether it is currently sending. This is used to determine what operations to perform, once an archive has been sent, it is delete from the SendState and the sending flag is set to false. A new archive is then added and sent when the opportunity arises.
	
	When a routing node receives the archive, it begins processing. Routing nodes use the SendState as well, but they only add an archive once it has been processed and they then select the oldest archive that has been classified as interesting, providing an archive is not already waiting to be sent. The archive stores information about the route it takes, recording every hop, as well as the time it took from capture to central node.
	
	Scheduled sending events run every thousand ticks, which is configurable, to check the sending state of the node and send any archives in the SendState. The node then waits for the number of ticks that it will take in order to transmit the archive.
	
	Once the simulation is completed, either manually or through a defined number of ticks, the archives in each central node are iterated over and written to a CSV file, with details such as the path it took, total transmission time and time of capture.
	
\subsection{Capture}
	Using the existing data collected from Danau Girang, we calculated how often a camera triggers in a six month deployment, as well as how often the observation contained interesting content. 
	
	To calculate the count of interesting images, we processed every directory of images to extract the largest object in the foreground, using our Triton program. Once processed, we iterated through every directory, counted the total number of images and the total number of extracted images. This gave us a 20.7\% chance of an image being interesting, across every camera.
	
	The chance of a camera being triggered each second was calculated by the total number of observations (13,399) divided by the number of seconds in six months (15,552,000). This gives a chance of 0.000861561. A random number is generated every 1000 ticks and compared to this, if the value is less than or equal to the chance, then an observation is captured.
	
\subsection{Processing}
	The types of knowledge processing capabilities that we outlined in Section \ref{sim:imp} are used in the simulation to determine which type of processing to perform on observations. The result of processing is that an observation is marked as interesting or empty. The outcome can be any of the following:
		\begin{description}
			\item True positive (TP): A region of interest is extracted that matches the object of interest in the images.
			\item False positive (FP): A region of interest is extracted that contains nothing of interest.
			\item True negative (TN): A camera is triggered with nothing of interest in the image and no region of interest is extracted.
			\item False negative (FN): An image that is not empty is captured but no region of interest extracted.
		\end{description}
	
	Using the results of the image processing application we developed (explained in Section \ref{tech:sf:triton}) for the properties of HK processing, we found that it performed with an 82\% accuracy at detecting TP images, with a 98\% accuracy for finding TN images. Nodes with LK do not have the ability to mark an image as empty, but they can mark an image as interesting. However, the results we have from our rule base are not as extensive as the results we have for Triton, so instead we use a predefined 10\% accuracy for detecting TPs.
	

\section{Results}
Each scenario, outlined in Section \ref{sim:imp}, runs to simulate a 6 month deployment. Using our motivating scenario, we modelled each scenario on a fixed number of nodes: 1 central node, 4 routing nodes and 20 sensing nodes. The implementation of our architecture has been limited to using Zigbee as the transmission medium between all sensing nodes, with a Wi-Fi connection between routing nodes and a central node. Sensing nodes then send an observation when it is captured or, if it has not captured anything then it checks for a backlog every ten minutes. Sensing nodes check for new observations to process every five minutes. We ran each scenario a hundred times, with each run simulating a 6 month duration. The time to process an observation using LK has been simulated to take 5 seconds compared to the 90 seconds when a node has HK capabilities; these values have been chosen based on the average processing time we derived using existing data.

\subsection{Scenarios}

This section details the results from each scenario of increasingly capable knowledge processing nodes. NK-ALL represents a 'standard' WSN where no nodes have knowledge processing capabilities and forward to a single endpoint. NK-LK simulates a network where sensing nodes have no knowledge processing capabilities, but routing nodes have LK, allowing them to detect whether an observation is interesting, but lack the knowledge to determine, with any confidence, whether an observation is of no interest. LK-ALL simulates a network where both sensing and routing nodes have LK capabilities. LK-HK is the Repast implementation of our K-HAS architecture; sensing nodes have LK and routing nodes have HK. HK-ALL has sensing and routing nodes that have HK capabilities. 

Table \ref{tab:observ_int} shows the time for average transmission time, in hours, for interesting observations. Interesting observations consist of \textit{true positives} and \textit{false negatives}, which means that the spread of transmission times would be quite varied, because TPs would be prioritised but FPs would be treated as empty images. However, we can see that the total of interesting images almost doubles when all nodes have HK. More importantly, the difference between HK-ALL and LK-HK is less than 2\%, while still providing a battery advantage. With LK-HK and HK-ALL, the median is much lower when compared to LK-ALL and NK-LK, due to the higher levels of knowledge processing capabilities.

With the results broken down further, Table \ref{tab:observ_tp} shows the average duration for true positives and we can see that, while the median stays much the same for each scenario, average duration varies considerably. The average time for a TP to be sent is lower for LK-HK than for LK-ALL, this could be due to the extra processing time for HK nodes or a processing backlog with a large number of observations. Although the duration of LK-HK is approximately twice that of LK-ALL, it does deliver more TPs, with 36.46\% of all images received being TP, compared with 8.45\% for LK-ALL.

\begin{table}[h]\footnotesize
\begin{tabularx}{\textwidth}{ |X|X|X|X|X|}
\hline
Scenario & Median & Mean & Standard Deviation & \% Total\\
\hline
NKLK & 608 & 841.95 & 854.18 & 22.11\\
LKALL & 127 & 652.33 & 868.09 & 24.35\\
LKHK & 3 & 255.49 & 591.63 & 38.86\\
HKALL & 2 & 59.57 & 320.48 & 40.78\\
\hline
\end{tabularx}
\caption{Transmission Time for Interesting Observations}\label{tab:observ_int}
\end{table}

\begin{table}[h]\footnotesize
\begin{tabularx}{\textwidth}{ |X|X|X|X|X|}
\hline
Scenario & Median & Mean & Standard Deviation & \% Total\\
\hline
NK-LK & 3 & 277.01 & 543.22 & 3.88\\
LK-ALL & 2 & 99.73 & 353.16 & 8.45\\
LK-HK & 3 & 207.09 & 518.11 & 36.46\\
HK-ALL & 2 & 2.77 & 2.75 & 38.16\\
\hline
\end{tabularx}
\caption{Transmission Time for True Positive Observations}\label{tab:observ_tp}
\end{table}

\begin{table}[h]\footnotesize
\begin{tabularx}{\textwidth}{ |X|X|X|X|X|}
\hline
Scenario & Median & Mean & Standard Deviation & \% Total\\
\hline
LK-ALL & 675.5 & 945.94 & 916.11 & 15.9\\ 
LK-HK & 683 & 989.02 & 1006.4 & 2.41\\
HK-ALL & 553 & 885.65 & 931.19 & 2.6\\
\hline
\end{tabularx}
\caption{Transmission Time for False Negative Observations}\label{tab:observ_fn}
\end{table}

\begin{figure}[!h]
\centering
\includegraphics[width=\textwidth]{Chap7/figures/all_int.png}
\caption{Mean Transmission Time for Interesting Observations in All Scenarios}
\label{fig:all_int}
\end{figure}

\begin{figure}[!h]
\centering
\includegraphics[width=\textwidth]{Chap7/figures/all_int_percent.png}
\caption{Percentage of Interesting Observations in All Scenarios}
\label{fig:all_int_percent}
\end{figure}

Figure \ref{fig:all_int} shows the average transmission time for interesting observations across every scenario. NK-ALL has no points for TP and FN because of its lack of processing capabilities. The drop in transmission time for interesting images is clearly visible here, but there is an increase in TN transmission time when comparing LK-ALL and LK-HK. This spike is better explained with the use of Figure \ref{fig:all_int_percent}, showing that, although it takes longer for TP images to be delivered in LK-HK, the percentage of TP images increased in LK-HK by almost four times that of LK-ALL. The faster transmission times in HK-ALL can be explained by the fact that all nodes can accurately detect empty images from the edge of the network and also prioritise interesting images with a greater accuracy than any other scenario. However, nodes with HK processing power have much higher power requirements and require battery changes every 3 weeks, this power trade off shows that nodes of this type can deliver interesting observations significantly faster than nodes with LK. A number of WSNs will place nodes in places that are not easily accessible by humans and not expected to be visited so regularly. In our motivating scenario, some nodes could be rendered inaccessible by river floods for weeks at a time and regular human traffic can prevent animals from using those sites. The percentage of TPs delivered by LK-HK and HK-ALL are not that different, but the delivery time when using both HK and LK nodes does affect the delivery time. 

\subsection{LK-HK Scenarios}

The main difference between HK-ALL and LK-HK is that all sensing nodes in HK-ALL have the ability to perform more intense data processing and determine which observations to prioritise with greater accuracy. Whereas LK-HK focusses on HK capabilities for the routing nodes and maximising the lifetime of sensing nodes by limiting them to LK capabilities. Therefore, we tested different proportions of LK and HK sensing nodes within the LK-HK scenario in order to determine the best ratio between performance and network lifetime. Table \ref{tab:khas_int} shows the average duration for an interesting observation when 4, 8, 12, 16 and 20 of the 20 sensing nodes have HK, with the rest having LK processing capabilities. Figure \ref{fig:khas_int_percent} illustrates how pushing knowledge further towards the edge of the network increases the delivery time and the percentage of interesting images. LK-HK 16 and LK-HK 20 have an average difference of three hours for interesting images, but LK-H K20 means that all nodes in the network have HK processing capabilities and, thus, a shorter battery life. While the difference in transmission time is not that significant, the extra LK nodes with a longer battery life means that they can be placed in areas that may not be easily accessible or need to remain undisturbed.

\begin{table}[h]\footnotesize
\begin{tabularx}{\textwidth}{ |X|X|X|X|}
\hline
Scenario & Median & Mean & Standard Deviation \\
\hline
LK-HK 4 & 2 & 153.21 & 462.75 \\
LK-HK 8 & 2 & 146.04 & 428.79 \\
LK-HK 12 & 2 & 156.95 & 461.67 \\
LK-HK16 & 2 & 88.86 & 349.68 \\
LK-HK 20 & 2 & 63.13 & 330.67 \\
\hline
\end{tabularx}
\caption{Transmission Time Results for Interesting Observations}\label{tab:khas_int}
\end{table}

\begin{table}[h]\footnotesize
\begin{tabularx}{\textwidth}{ |X|X|X|X|}
\hline
Scenario & Median & Mean & Standard Deviation \\
\hline
LK-HK 4 & 2 & 104.74 & 393.06 \\
LK-HK 8 & 2 & 94.56 & 357.04 \\
LK-HK 12 & 2 & 86.85 & 343.69 \\
LK-HK16 & 2 & 29.15 & 191.82 \\
LK-HK 20 & 2 & 2.08 & 1.93 \\
\hline
\end{tabularx}
\caption{Transmission Time Results for True Positive Observations}\label{tab:khas_tp}
\end{table}

\begin{figure}[!h]
\centering
\includegraphics[width=\textwidth]{Chap7/figures/khas_int_percent.png}
\caption{Percentage of Interesting Observations in LK-HK Variations}
\label{fig:khas_int_percent}
\end{figure}

\begin{figure}[!h]
\centering
\includegraphics[width=\textwidth]{Chap7/figures/all_total.png}
\caption{Mean Duration for All Observations in Zigbee Scenarios}
\label{fig:all_total}
\end{figure}

\subsection{Unrestricted Transmission Speeds}
In this section, we manipulated the variables set in out original simulations, that were based on existing sensed data, to identify bottlenecks that prevented observations from being received immediately. Zigbee has a low transmission rate, of 2.77 Kbps, therefore we expected increasing the transmission rate of all nodes in the network to a rate in the thousands (61920 Kbps) to determine whether this would reduce the mean transmission time for all images. Figure \ref{fig:unres_int} shows that, when compared with Figure \ref{fig:all_int}, the transmission time is reduced in all scenarios, by up to half in some cases. HK-ALL shows that this increase allows for interesting observations to be delivered with almost no delay, in near real-time. However, with a rate that should allow almost all observations to be sent in a matter of seconds, many scenarios still have a delay of 600 hours. Figure \ref{fig:unres_total} shows the average duration for all observations received at the central node and, with Wi-Fi transmission rates, we expected this to be much higher. However, when compared with Figure \ref{fig:all_total}, which shows the average duration for all observations using Zigbee as the transmission medium, we can see that the transmission time for most scenarios reduces to fewer than 500 hours. HK-ALL, remains much the same, despite the faster transmission rate. 

Another bottleneck that we identified was the duration that each node would check for new observations, and process them. In the current solution, sensing nodes check every 10 minutes and routing nodes check every 5. This delay could cause an observation to not be picked up immediately, as well as the transmission delay. To test this, we reduced the Sensing node check to every minute and the Routing node to check every 30 seconds for new observations. Using Zigbee, the mean transmission time for all observations, in the HK-ALL scenario, was 501.02 hours. Using the increased transmission rate, it was 354.34 hours and with the reduced checking delay, the time dropped to 1.58 hours. Therefore, in order to create a real time implementation of these scenarios, one would need to use a fast transmission medium, such as Wi-Fi, and increase the time delay to check for new data; both of which would reduce the battery life of the node. This solution would ensure that all sensed data would be received almost as soon as it was captured, however, our HK-ALL Zigbee implementation would deliver interesting images within 3 hours but conserve battery life by using a radio with less power consumption and checking for new data less often. This faster solution would be primarily suited to deployments where power is not a constraint, but the real-time receipt of all sensed data is.

\begin{figure}[!h]
\centering
\includegraphics[width=\textwidth]{Chap7/figures/unres_all_int.png}
\caption{Mean Duration for Interesting Observations in Unrestricted Scenarios}
\label{fig:unres_int}
\end{figure}

\begin{figure}[!h]
\centering
\includegraphics[width=\textwidth]{Chap7/figures/unres_all_total.png}
\caption{Mean Duration for All Observations in Unrestricted Scenarios}
\label{fig:unres_total}
\end{figure}

\section{Conclusion}
	
In this chapter, we have detailed the development of simulations to show the different scenarios for pushing knowledge further out into a network. Using different levels of knowledge processing capabilities on nodes, we have shown that a network with HK processing capabilities can detect and prioritise interesting images, while simultaneously delaying empty images for a time when the network is not busy. However, using a network that solely comprises of HK nodes results in a battery life that lasts for 3 weeks on each node. The K-HAS network architecture we have proposed provides a combination of HK and LK nodes, distributed based on their role in the network, for example, nodes tasked with sensing and forwarding images only have LK processing capabilities. These simulations have shown that the HK-LK scenario results in a delay of interesting image delivery, when compared to LK-ALL, but the percentage of interesting images delivered is significantly increased.

Bridging the gap between HK-LK and HK-ALL, we recorded the results when the number of HK nodes versus LK nodes was incremented by 4 on each run, until there were 20 in total. The results of these showed that LK-HK with all HK nodes was very similar in performance to HK-ALL, but with the same issues with node lifetime. However, with 16 HK and 4 LK nodes, the performance was similar, while allowing for 4 nodes with a longer battery life to be deployed.

While the simulation is not feature complete, we believe it is accurate enough to show how LK-HK utilises the knowledge-processing capabilities at each tier to process, and prioritise, sensed data based on knowledge gained from the environment, previously sensed data and from humans using the network. We also show that LK-HK is not the ideal solution for every network; less accessible networks are better suited to LK-ALL and WSNs with constant power supplies would benefit from HK-ALL. LK-HK simply acts as the compromise between these scenarios that allow for the network lifetime to be maximised, while ensuring fast delivery of interesting sensed data. The number of HK nodes used in this scenario can also be varied, based on how accessible the location of each node is and the required lifetime of the network. For example, a WSN to handle the security in a block of apartments would not have power constraints but may need real time reporting on break-ins. However, a WSN based in a remote woodland with no access to solar power may not require real-time updates, but network lifetime will need to be maximised. In summary, our results show that an increase in HK nodes in the network creates an increase in the the percentage of interesting images delivered, as well as a reduction in the transmission time.



\chapter{Conclusion and Future Work}
In this thesis, we aimed to show that utilising the local knowledge of an environment in a WSN improves the efficiency of the network by giving it the ability to prioritise sensed data based on the results of in-network processing.  We believe that pushing knowledge out farther towards the edges of the network improves the overall performance. To show this, we have developed a three-tier WSN architecture that uses knowledge-processing capabilities to process sensed data as it is forwarded through the network. Most current WSN implementations deliver data chronologically, or store it on the node to be retrieved by queries. We believe that this knowledge can be used to infer how valuable sensed data is and prioritise that through the network, delivering the most interesting data first. However, resources are still limited in WSNs and our architecture had to utilise these resources effectively, such as battery life and bandwidth, to maximise the network lifetime. Using Data Collection (DC) nodes at the edge of the network, they capture observations and use their limited knowledge-processing capabilities to enrich the sensed data before sending it on. Data Processing (DP) nodes use more powerful knowledge-processing features to attempt to classify observations and prioritise the sending of them to Data Aggregation (DA) nodes. DA node make the data available to users and use their classifications, and input, to dynamically update its knowledge base.

We used a collaboration with a research centre in Malaysia with the aim to implement K-HAS in the Malaysian rainforest, in an area that had experienced logging and contained a diverse range of rare wildlife. Using K-HAS, we wanted to deploy a network that would prioritise images of rare wildlife and only send images of common wildlife when bandwidth was available.  Over the course of 3 visits, we gathered knowledge from the area, researchers and locals to build a knowledge base and create rules that could be used to classify data. We also collected images taken from their current, manual solution to infer patterns and use existing classifications for new sensed data.

Current sensing technology means that K-HAS is not ready to be implemented as the architecture dictates. DC nodes are not a type of sensor that is readily available, and this is especially true for image capturing sensor nodes. Because of this, we modified the architecture to use commercial hardware with fewer capabilities that would allow us to actually deploy a sensor network in the Malaysian rainforest that does use local knowledge. We call this architecture LORIS, Local-knowledge Ontology-based Remote-Sensing Informatics System, and this solution combines the DA and DP node to hold all of the knowledge-processing capabilities.

LORIS has shown that even using local knowledge at the base station of a WSN means that sensed data is processed and organised within minutes of being received. This method also means that users are alerted to data that they have subscribed automatically.

Our simulations of K-HAS show that the bandwidth in a WSN is used more effectively when knowledge is pushed out towards the edge of the network, allowing nodes to perform knowledge-processing to make inferences about the contents of the data and prioritise, or delay, its delivery appropriately. We have no reason to think that this would not work in practice.


\section{Summary of Contributions}
In this section, we summarise the contributions detailed in this thesis, focussing on our deployment in Malaysia, the tiered architecture designed for general use and the alignment ontology created to formalise the architecture of K-HAS and the data standard that it uses. 
\subsection{K-HAS}
In Chapter 4, we present a novel tiered architecture, K-HAS, for WSNs that uses local knowledge. We explored existing networks, those related to our motivating scenario in Malaysia and also those that use knowledge or context-awareness, and routing protocols that use the sensed data to determine how it is routed. We explain the purpose of each tier in the network and show how sensed data is enriched and routed as it progresses through the network. Using Darwin Core as a data standard, each node communicates in a common format and metadata is packaged with data in archives that can be read by any node in the network. We also show how rules can be used to infer the content of sensed data and the ability to run rule engines in the network allows sensed data to be prioritised based on its value; not just the time it was captured.
\subsection{Ontology}
In Chapter 5 we explain the aligning ontology created to formally represent the K-HAS architecture and the data standard used. We show how no ontology current joins observation-centric and sensor-centric ontologies to completely represent the hardware used in a WSN as well as the sensed data. We also show how the ontology is extensible and in no way specific to K-HAS; it can be used with any WSN that deals with scientific observations. The modular nature of the ontology means that parts of the ontology can be extracted and re-used in a network that the entire ontology may not be suitable for.
\subsection{LORIS}
In Chapter 6, we present the Local-knowledge Ontology-based Remote-Sensing Informatics System (LORIS), a system developed for our motivating scenario when we discovered that current sensing technology means that K-HAS is not ready to be implemented in its current form. DC nodes are not a type of sensor that is readily available, and this is especially true for image capturing sensor nodes. Because of this, we modified the architecture to use commercial hardware with fewer capabilities that would allow us to deploy a sensor network in the Malaysian rainforest that does use local knowledge. This solution combines the DA and DP node to hold all of the knowledge-processing capabilities and then uses commercially available sensors to replace DC nodes. 

This architecture is easier to implement but does not provide in-network processing, although it automates the delivery of sensed data and uses the increased knowledge-processing capabilities of modern PCs to process sensed data on arrival and inform users. We show how our deployment of LORIS was successful and highlighted how sensed data was delivered within minutes of being captured in some cases, and processed shortly after. 
\subsection{Simulations}
In Chapter 7, we explain the implementation and results of our simulations to model an ideal deployment of K-HAS. We model every variable of the network on existing data collected from our motivating scenario and show that the delivery of observations can be reduced by more than four hundred hours, when compared to the current manual solution. We also outline how the network is able to prioritise data that it believes to be interesting, using a priority queue mechanism that delays data it believes to be empty. Comparing K-HAS to different network implementations, where nodes have different levels of knowledge processing capabilities, shows that our network implementation is more power efficient than a network where every node has higher knowledge processing capabilities, yet its performance is similar.

\section{Future Work}
The focus of this thesis is to show that local knowledge can improve the timeliness of interesting data by making inferences based on previously sensed data and knowledge of the environment. We have shown, using simulations, that this can be done with our tiered approach. However, another aim was to deploy such an architecture for our collaborators. We have explained that current technology means that this is not feasible, but sensing, and microcomputer technology has moved forward significantly since the beginning of this PhD and it would be possible with a longer time period. Our deployment in Malaysia was limited due to time constraints and we would haved liked to leave a functional deployment active in Danau Girang for a six month period. 

With more time, we could create custom DC nodes using webcams and micro-computers, like the Raspberry Pi, encased in a watertight enclosure; allowing us to use higher levels of knowledge-processing at the edge of the network. A deployment of this length would also allow nodes to act on sensed data classified by users and update their knowledge base, responding to changes in the network dynamically.

K-HAS uses a combination of open source projects and some software created during the course of this work, but it does require specialist knowledge to be deployed. We would like to create an installation candidate that would be usable by those without any expertise that could provide basic information on the purpose of the network and the installation of all necessary packages would be automated.

On top of this, our software's user interfaces have been tested by researchers at Danau Girang but we have no metrics on the usability of the software. Testing the software on users to determine how they would score the different areas of the software, such as usability, response time and learning curve, would help us to improve the software and ensure that it can be used by those without any technical knowledge of K-HAS or its underlying architecture.

Experiments have shown us that the range of wireless transmissions is often hard to predict and can be heavily influenced by weather and obstacles. If sensors were placed at the edge of range for a neighbouring sensor, then it is not guaranteed that transmissions could be made every time. We believe that using humans as an intermediate hop could be an interesting research opportunity. Using a mobile app to transfer data between nodes as they are within walking range could speed up data transfers and their knowledge could be used to prioritise data without the need to process it.

If a human views a recent observation when they are in range of a node, they could make an instant assessment on whether it is interesting or not. If they mark it as such, then it could be passed through the network with a higher priority and reach a DA node within a short period as it would not require processing, updating knowledge bases for future, similar observations as it is forwarded.

One important goal is to create a deployment of K-HAS for a different purpose than our motivating scenario in Malaysia. There is a need in Malaysia to track hunters in the forest and alert authorities, this would not require any change in how K-HAS is currently implemented but would show how it can dynamically adapt based to changes in its sensing requirements. We would also like to test K-HAS in a situation where power is no longer a limiting factor, but delivery time of sensed data would be. A building security network would be one such example, deploying K-HAS across a number of floors and processing video feeds to alert users within seconds about suspicious activity.

However, using K-HAS for a WSN that, for example, uses text based sensed data to monitor the temperature and lava level of a volcano is a completely different implementation, but the local knowledge could be used to prevent an emergency and predict eruptions. A deployment such as this would show that the benefit of using local knowledge in a WSN is not limited to our motivating scenario, but is versatile enough to benefit almost any network.

Using heterogenous sensor nodes within a K-HAS network would also show how local knowledge can be used in different streams, as well as how they can be combined to make more detailed inferences.  Using motion sensors with microphones could be used to determine what person/animal is near the sensor and this information can then be used to infer patterns as fallback sources when an image based classification cannot be made.

Our simulations in Chapter 7 shows how K-HAS can use local knowledge, but it is by no means a complete implementation.  We need to implement a more modular simulation that allows K-HAS to be applied to any form of WSN. We would like to simulate individual knowledge bases on every node and experiment with networks that contain multiple central nodes to visualise the flow of sensed data through the network and see if there is an effect, positive or negative, on the speed of interesting sensed data. A more immediate goal is to test K-HAS when all of the network uses a communication medium with lower range but a higher transfer rate, such as Wi-Fi, to determine how much Zigbee slows the delivery of sensed data when compared with processing.

We would also like to experiment with different ratios of knowledge processing capabilities on nodes to determine if there is an ideal ratio that maximises the flow of sensed data, delivering interesting data first and quickly but also delivering data that has been classified as not interesting within a shorter time period that would allow humans to act on the data if it had been misclassified.

\bibliographystyle{plainnat}
\bibliography{refs/library}
\begin{appendices}
\chapter{Example Darwin Core Archive}\label{appendix:dwc}
\lstinputlisting[caption={EML File},language=XML]{Chap4/listings/dwc_arch/eml.xml}
\lstinputlisting[caption={Metadata File},language=XML]{Chap4/listings/dwc_arch/meta.xml}
\lstinputlisting[caption={Example Set File (set.csv)}]{Chap4/listings/dwc_arch/set.csv}
\lstinputlisting[caption={File Describing Image Locations (images.csv)}]{Chap4/listings/dwc_arch/images.csv}

\chapter{K-HAS Ontology}\label{appendix:ontology}
\lstinputlisting[language=XML, caption={K-HAS Ontology Source Code}]{App/listings/align.owl}

\chapter{Example Drools Rules Used in LORIS}\label{appendix:drools}
\lstinputlisting[caption={Example Drools File}]{Chap4/listings/rules.drl}

\chapter{Interview Transcript: Gill Bolongon}\label{appendix:interview}
\lstinputlisting[caption={Sample Intrview Transcript}]{App/listings/Gill.txt}

\chapter{Qualitative Analysis Extract}\label{appendix:interview:extract}
\includegraphics[width=\textwidth]{App/figures/roshan_extract}
\end{appendices}
\backmatter


\end{document}
