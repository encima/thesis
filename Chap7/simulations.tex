\chapter{Simulations}
In this chapter, we detail simulations that we developed to test our hypothesis with a complete implementation of K-HAS. LORIS was implemented because current technology does not fully support the requirements of the K-HAS architecture, because of this we implemented a simulation of K-HAS that satisfies all of its requirements.

Using RePast, a java-based network simulation tool, we created a network to emulate K-HAS. Repast is an agent-based modelling system that allows for agents to be created and placed on a grid. Ticks denote a period of time and simulations can run for a fixed number of ticks, or until stopped. Ticks can also be used to schedule events, such as searching for neighbours, by calling methods that last for a set number of ticks, or begin at a particular tick. Similar to the chaining of rules we described with Drools, ticks are the core of Repast's scheduling mechanism which can be used to schedule single events, as well as chaining events. Consider a train simulation. A train may take three hundred ticks to arrive at its destination, from the train arrival a scheduled event to open doors and make announcements which could schedule other events and so on.

Agents are created, using the RePast SDK and Java classes are used to manipulate their behaviour. Simple networks may only contain basic agents with only a few variations from those provided by RePast. However, for more complex networks, a hierarchy of agents is required and Java's inheritance can then be used create subclasses of an agent.

A 2D (or 3D) space is used to display the grid and the simulation is run within Repast's own GUI, that provides functionality such as editing the properties of classes, integrating with Matlab, taking screenshots and saving different configurations of the same network.

The rest of this chapter is structured as follows. Section 2 describes the implementation of the network. Section 3 outlines the results and Section 4 compares these with LORIS and the current solution in our motivating scenario. Section 5 concludes our findings and highlights areas that require further experimentation.

\section{K-HAS Implementation}
As described in Chapter \ref{chap:arch}, K-HAS consists of three tiers: Data Collection, Data Processing and Data Aggregation. The DC tier focusses on capturing sensed data, performing basic processing and routing sensed data to the DP tier. The DP tier has more knowledge-processing capabilities and nodes are responsible for processing sensed data. Using a rule engine and a dynamic knowledge base, DP nodes process more than just the metadata of sensed data in order to create a classification. DP nodes have a direct connection to the DA node, which is the the endpoint in the K-HAS network, an Internet connected node that stores all sensed data from the network and acts as the central knowledge base.

K-HAS was originally intended to be deployed around Danau Girang and, over the course of three years, we have collected several thousand images and local knowledge interviews in order to configure K-HAS before deployment. Because of this, the data we have collected was used to create a deployment of K-HAS using scientific observations, in order to fit with our motivating scenario. LORIS had already been deployed to capture scientific observations, so this made a comparison of the three (current manual solution, LORIS and K-HAS simulation) possible and because we already had metrics on areas of the network such as battery life, transmission time of a Darwin Core archive and processing time.

Before implementing, we designed the agents required based largely on the existing ontology. Using that, we created a hierarchy of nodes inheriting common properties from a node object. As previously mentioned, we had metrics on range and transmission times from previous experiments and the deployment of LORIS, we used these to create properties for each transmission medium that could be used by each node object.

We also needed to create an object to represent the DwC archives, a Drools REST API had been implemented to work with the LORIS web interface, and much of the DwC archive code was reusable within RePast. 

The structure of the simulation is shown below. Network builder instantiates all the nodes, places them on the grid and schedules necessary events. 

\begin{itemize}
\item Network Builder
\item Node
	\begin{itemize}
	\item Data Collection
	\item Data Processing
	\item Data Aggregation
	\end{itemize}
\item Darwin Core
	\begin{itemize}
	\item Identification
	\item Location
	\item Occurrence
	\item Image
	\item Species
	\end{itemize}	 
\end{itemize}

\subsection{Darwin Core}
The Darwin Core class represents a DwC archive, encapsulation Identification, Location, Occurrence, Image and Species. The images we have collected from Danau Girang were processed to find details such as the average size when taking at night and day, how often an average camera triggers and the percentage of images with animal content. These data were then used to specify how often a randomly placed node should capture an observation per tick.

Upon each capture, and based on the 'time' of day, images are created, given a random size based on the maximum and minimum size found in the 120,000 images collected from DG and the sum of the images is used to calculate the size of the archive. Using this size, a DC node calculates how long the archive takes to send based on the size and the transmission rate, we assume that the rate stays constant for the duration of transmission.

When an archive is sent to the DP node, we used the average time for our image processing tool and Drools engine to run and attempt a classification, which is 143 seconds (ticks). 
\subsection{Routing}

